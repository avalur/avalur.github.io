\documentclass[12pt,a4paper]{article}
\usepackage[utf8]{inputenc}
\usepackage[T1]{fontenc}
\usepackage{amsmath,amsfonts,amssymb,amsthm}
\usepackage{geometry}
\usepackage{fancyhdr}
\usepackage{enumerate}
\usepackage{tikz}
\usepackage{pgfplots}
\usepackage{hyperref}
\usepackage{xcolor}

\geometry{margin=2.5cm}
\pagestyle{fancy}
\fancyhf{}
\fancyhead[C]{GPT-5 Solutions}
\fancyfoot[C]{\thepage}

\title{GPT-5 Solutions}
\author{Generated from JSON Data}
\date{\today}

\theoremstyle{definition}
\newtheorem{problem}{Problem}
\newtheorem{solution}{Solution}

\begin{document}

    \maketitle
    \tableofcontents
    \newpage

    \section{Problem 1, Day 1}
    \subsection{Variant 1}
    Let $P \in \mathbb{R}[x]$ with $\deg P \geq 2$. For $x \in \mathbb{R}$, the tangent line at $(x, P(x))$ has equation
    $$y = P(x) + P'(x)(t - x).$$
    A point $(a, b)$ lies on this line if and only if
    $$b = P(x) + P'(x)(a - x).$$
    Thus, for fixed $a$, the set of values of $b$ that are attained by tangent lines at abscissa $a$ is the image of the function
    $$G_a(x) := P(x) + P'(x)(a - x).$$

    \textbf{(a)} $\deg P$ odd implies $\bigcup_x \ell_x = \mathbb{R}^2$.

    Let $\deg P = n$ be odd and let $c$ be the leading coefficient of $P$, so $P(x) = cx^n + \text{lower order terms}$. Then
    $$G_a(x) = P(x) + aP'(x) - xP'(x).$$
    The coefficient of $x^n$ in $G_a$ is
    $$\text{coeff}(x^n) = c - nc = (1-n)c \neq 0$$
    (since $n \geq 2$). Therefore $\deg G_a = n$ is odd. For any $b \in \mathbb{R}$, the polynomial equation
    $$G_a(x) - b = 0$$
    has odd degree, hence at least one real root. Thus for every $(a,b) \in \mathbb{R}^2$ there exists $x$ with $b = P(x) + P'(x)(a-x)$, i.e., $(a,b)$ lies on the tangent at $x$. Hence the union of all tangent lines is all of $\mathbb{R}^2$.

    \textbf{(b)} No even-degree polynomial has this property.

    Suppose $\deg P = n$ is even. As above,
    $$G_a(x) = P(x) + aP'(x) - xP'(x)$$
    has leading term $(1-n)cx^n$ with $(1-n)c \neq 0$, so $\deg G_a = n$ is even. Hence as $x \to \pm\infty$, $G_a(x) \to \pm\infty$ with the same sign, so the image of $G_a$ is an interval of the form $(-\infty, M]$ or $[m, \infty)$. In particular, $G_a$ is not surjective onto $\mathbb{R}$. Therefore, for this fixed $a$, there exist $b$ for which the equation $b = G_a(x)$ has no real solution, i.e., some points $(a,b)$ do not lie on any tangent line. Consequently, no polynomial of even degree can have $\bigcup_x \ell_x = \mathbb{R}^2$.

    A concrete illustration: for $P(x) = x^2$, $G_a(x) = 2ax - x^2$, whose range is $(-\infty, a^2]$, not all of $\mathbb{R}$.

    \textbf{Final answer.}
    \begin{enumerate}
        \item[(a)] True for any odd-degree polynomial $P$: the union of its tangent lines is all of $\mathbb{R}^2$.
        \item[(b)] No. There is no polynomial of even degree whose tangent lines cover $\mathbb{R}^2$.
    \end{enumerate}

    \subsection{Variant 2}
    Let $P$ be a real polynomial of degree $n \geq 2$, with leading coefficient $c \neq 0$. For each $x \in \mathbb{R}$, the tangent line at $(x, P(x))$ is
    $$\ell_x = \{ (t, P(x) + P'(x)(t - x)) : t \in \mathbb{R} \}.$$

    A point $(a, b) \in \mathbb{R}^2$ lies on $\ell_x$ iff
    $$b = P(x) + P'(x)(a - x).$$

    Define, for each fixed $a \in \mathbb{R}$,
    $$H_a(x) = P(x) + (a - x) P'(x).$$
    Then $(a, b)$ lies on some tangent line iff the equation $H_a(x) = b$ has a real solution $x$.

    Compute the degree and leading coefficient of $H_a$. If $P(x) = c x^n + \text{lower terms}$, then $P'(x) = n c x^{n-1} + \text{lower}$, and hence
    \begin{align}
        H_a(x) &= c x^n + (a - x)(n c x^{n-1}) + \text{lower} \\
        &= (1 - n) c x^n + n a c x^{n-1} + \text{lower}.
    \end{align}
    Thus $\deg(H_a) = n$ and its leading coefficient is $(1 - n)c \neq 0$.

    \textbf{(a)} If $n$ is odd, then $H_a$ is an odd-degree polynomial, hence
    $$\lim_{x\to+\infty} H_a(x) \text{ and } \lim_{x\to-\infty} H_a(x) \text{ have opposite signs}.$$
    By continuity, $H_a(\mathbb{R}) = \mathbb{R}$. Therefore for every $a, b \in \mathbb{R}$ there exists $x$ with $H_a(x) = b$, which means every point $(a, b)$ lies on some tangent line. Hence $\bigcup_{x\in\mathbb{R}} \ell_x = \mathbb{R}^2$.

    \textbf{(b)} If $n$ is even, then $H_a$ has even degree with leading coefficient $(1 - n)c$. There are two cases:
    \begin{itemize}
        \item If $c > 0$, then $(1 - n)c < 0$, so $H_a(x) \to -\infty$ as $x \to \pm\infty$. Thus $H_a$ is bounded above on $\mathbb{R}$. For any $b$ larger than $\sup H_a(\mathbb{R})$, the equation $H_a(x) = b$ has no real solution, so points $(a, b)$ with such $b$ are not covered by the union of tangents.
        \item If $c < 0$, then $(1 - n)c > 0$, so $H_a(x) \to +\infty$ as $x \to \pm\infty$, and $H_a$ is bounded below. For any $b$ smaller than $\inf H_a(\mathbb{R})$, there is no solution, hence such points are not covered.
    \end{itemize}

    In either case, for even $n$ there exist points $(a, b)$ not lying on any tangent line. Therefore no polynomial of even degree has $\bigcup_{x\in\mathbb{R}} \ell_x = \mathbb{R}^2$.

    \textbf{Final answer.}
    \begin{enumerate}
        \item[(a)] True: for odd $\deg(P)$, $\bigcup_{x\in\mathbb{R}} \ell_x = \mathbb{R}^2$.
        \item[(b)] No: for even $\deg(P)$, the union of tangent lines cannot be all of $\mathbb{R}^2$.
    \end{enumerate}

    \subsection{Variant 3}
    Let $P$ be a real polynomial of degree $n \geq 2$, and for each $x \in \mathbb{R}$ let $\ell_x$ be the tangent line to the graph of $P$ at $(x, P(x))$. A convenient representation of $\ell_x$ is
    $$y = P(x) + P'(x)(t - x),$$
    so that for a fixed abscissa $t$, the intersection of $\ell_x$ with the vertical line $t = \text{const}$ has ordinate
    $$L_x(t) = P(x) - xP'(x) + tP'(x).$$

    \textbf{(a) Degree of $P$ odd $\Rightarrow$ $\bigcup_x \ell_x = \mathbb{R}^2$.}

    Fix an arbitrary point $(u, v) \in \mathbb{R}^2$. Consider the polynomial in $x$
    $$F(x) = P(x) + P'(x)(u - x) - v = [P(x) - xP'(x)] + uP'(x) - v.$$
    If $\deg P = n$ with leading coefficient $a_n \neq 0$, then
    $$P(x) - xP'(x) = a_n x^n - n a_n x^n + \text{lower degree terms} = (1 - n) a_n x^n + (\text{lower}),$$
    while $uP'(x)$ has degree at most $n - 1$. Hence $\deg F = n$ and its leading coefficient is $(1 - n) a_n \neq 0$. In particular, if $n$ is odd, then $F$ is an odd-degree real polynomial, so
    $$\lim_{x \to \infty} F(x) \quad \text{and} \quad \lim_{x \to -\infty} F(x)$$
    have opposite signs. By the intermediate value theorem there exists $x_0$ with $F(x_0) = 0$, i.e.,
    $$v = P(x_0) + P'(x_0)(u - x_0),$$
    so $(u, v)$ lies on the tangent line at $x_0$. Since $(u, v)$ was arbitrary, $\bigcup_x \ell_x = \mathbb{R}^2$.

    \textbf{(b) No even-degree polynomial has this property.}

    Assume $\deg P = n$ is even with leading coefficient $a_n$. Fix $u \in \mathbb{R}$ and consider the function of $x$
    $$H_u(x) = L_x(u) = P(x) - xP'(x) + uP'(x).$$
    As above, $H_u$ is a polynomial in $x$ of degree $n$ with leading coefficient $(1 - n) a_n$. Since $n \geq 2$, $1 - n < 0$.

    \begin{itemize}
        \item If $a_n > 0$, then $(1 - n) a_n < 0$, so $H_u(x) \to -\infty$ as $x \to \pm\infty$. Hence $H_u$ attains a finite global maximum $M(u)$, and the set $\{H_u(x) : x \in \mathbb{R}\}$ is contained in $(-\infty, M(u)]$. Thus no tangent line meets the vertical line $t = u$ above height $M(u)$, so points $(u, v)$ with $v > M(u)$ are not covered.

        \item If $a_n < 0$, the same reasoning shows $H_u(x) \to +\infty$ as $x \to \pm\infty$, so $H_u$ attains a finite global minimum $m(u)$, and no tangent line meets $t = u$ below height $m(u)$. Points $(u, v)$ with $v < m(u)$ are not covered.
    \end{itemize}

    Therefore for any even-degree polynomial $P$, the union of tangent lines cannot be all of $\mathbb{R}^2$.

    \textbf{Final answer:}
    \begin{itemize}
        \item[(a)] True: for odd degree $P$, $\bigcup_x \ell_x = \mathbb{R}^2$.
        \item[(b)] No: for every even-degree polynomial $P$, the union of tangent lines misses some points, so the equality fails.
    \end{itemize}

    \subsection{Final Solution}
    Let $P \in \mathbb{R}[x]$ with $\deg P = n \geq 2$, and for $x \in \mathbb{R}$ let $\ell_x$ be the tangent line to the graph of $P$ at $(x, P(x))$:
    $$\ell_x = \{ (t, P(x) + P'(x)(t - x)) : t \in \mathbb{R} \}.$$

    A point $(a, b)$ lies on $\ell_x$ iff
    $$b = P(x) + P'(x)(a - x).$$

    For fixed $a$, define
    $$H_a(x) = P(x) + (a - x)P'(x).$$
    Then $(a, b)$ lies on some tangent line iff the equation $H_a(x) = b$ has a real solution $x$.

    Compute the leading term of $H_a$. If $P(x) = c x^n + \text{(lower terms)}$ with $c \neq 0$, then $P'(x) = n c x^{n-1} + \text{(lower terms)}$, and hence
    \begin{align}
        H_a(x) &= c x^n + a(n c x^{n-1}) - x(n c x^{n-1}) + \text{(lower terms)}\\
        &= (1 - n)c x^n + \text{(lower terms)}.
    \end{align}
    Because $n \geq 2$, $(1 - n)c \neq 0$, so $\deg H_a = n$.

    \textbf{(a)} If $n$ is odd, $H_a$ has odd degree, so $\lim_{x\to+\infty} H_a(x)$ and $\lim_{x\to-\infty} H_a(x)$ have opposite signs. By continuity, $H_a(\mathbb{R}) = \mathbb{R}$. Thus for every $(a, b) \in \mathbb{R}^2$ there exists $x$ with $H_a(x) = b$, i.e., every point lies on some tangent line. Hence $\bigcup_{x\in\mathbb{R}} \ell_x = \mathbb{R}^2$.

    \textbf{(b)} If $n$ is even, then $H_a$ has even degree with leading coefficient $(1 - n)c$. Consequently, $H_a(x) \to -\infty$ as $x \to \pm\infty$ when $c > 0$, and $H_a(x) \to +\infty$ as $x \to \pm\infty$ when $c < 0$. In either case, $H_a$ is bounded on one side (has a finite global maximum if $c > 0$, or a finite global minimum if $c < 0$). Therefore, for this fixed $a$, there exist $b$ for which $H_a(x) = b$ has no real solution. Hence the vertical line $t = a$ contains points not lying on any tangent line, and $\bigcup_{x\in\mathbb{R}} \ell_x \neq \mathbb{R}^2$ for every even-degree polynomial.

    \textbf{Final answer.}
    \begin{itemize}
        \item[(a)] True: if $\deg P$ is odd, then $\bigcup_{x\in\mathbb{R}} \ell_x = \mathbb{R}^2$.
        \item[(b)] No: for every even-degree polynomial $P$, $\bigcup_{x\in\mathbb{R}} \ell_x \neq \mathbb{R}^2$.
    \end{itemize}

    \section{Problem 2}
    \subsection{Variant 1}
    Let $J(f) = \int_{-1}^{1} (f''(x))^2 dx$. We minimize $J(f)$ under the constraints
    \begin{itemize}
        \item $f(1) = f(-1) = 1$ (fixed endpoint values),
        \item $\int_{-1}^{1} f(x) dx = 0$ (integral constraint).
    \end{itemize}

    \textbf{1) Euler–Lagrange equation and natural boundary conditions}

    Introduce a Lagrange multiplier $\lambda$ for the integral constraint and consider
    $$I(f) = \int_{-1}^{1} [(f''(x))^2 + \lambda f(x)] dx,$$
    with $f(\pm 1)$ fixed. For variations $h$ with $h(\pm 1) = 0$ we have
    $$0 = \frac{d}{dt}\bigg|_{t=0} I(f + t h) = 2 \int f'' h'' + \lambda \int h.$$

    Integrating by parts twice,
    $$\int f'' h'' = [f'' h']_{-1}^{1} - \int f''' h' = [f'' h']_{-1}^{1} - [f''' h]_{-1}^{1} + \int f^{(4)} h.$$
    Since $h(\pm 1) = 0$, the middle boundary term vanishes. Because $h'(\pm 1)$ is arbitrary, the boundary term $[f'' h']_{-1}^{1}$ vanishes for all such $h$ only if
    $$f''(-1) = f''(1) = 0.$$
    Then the stationarity condition reduces to
    $$\int (2 f^{(4)} + \lambda) h = 0 \text{ for all } h \text{ with } h(\pm 1) = 0,$$
    hence $2 f^{(4)} + \lambda = 0$ on $[-1,1]$. Therefore $f^{(4)}$ is a constant, and so $f$ is a quartic polynomial.

    \textbf{2) Solving for the minimizer}

    Write $f(x) = a x^4 + b x^3 + c x^2 + d x + e$.
    From $f''(x) = 12 a x^2 + 6 b x + 2 c$ and $f''(\pm 1) = 0$ we get
    \begin{align}
        12 a + 6 b + 2 c &= 0,\\
        12 a - 6 b + 2 c &= 0
    \end{align}
    $\Rightarrow b = 0, c = -6 a$.

    From $f(\pm 1) = 1$ we get $d = 0$ and $e = 1 + 5 a$.

    The integral constraint:
    \begin{align}
        0 &= \int_{-1}^{1} f(x) dx = 2 \int_{0}^{1} [a(x^4 - 6 x^2) + e] dx\\
        &= 2 \left[a\left(\frac{1}{5} - 2\right) + e\right] = 2 \left(-\frac{9a}{5} + e\right),
    \end{align}
    so $e = \frac{9 a}{5}$. Comparing with $e = 1 + 5 a$ yields
    $$1 + 5 a = \frac{9 a}{5} \Rightarrow 5 + 25 a = 9 a \Rightarrow 16 a = -5 \Rightarrow a = -\frac{5}{16},$$
    and then $e = -\frac{9}{16}$. Hence
    $$f(x) = \left(-\frac{5}{16}\right) x^4 + \frac{30}{16} x^2 - \frac{9}{16} = \frac{-5 x^4 + 30 x^2 - 9}{16}.$$

    \textbf{3) Computing the minimum value}

    For this $f$,
    $$f''(x) = 12 a (x^2 - 1) = 12\left(-\frac{5}{16}\right)(x^2 - 1) = \frac{15}{4}(1 - x^2).$$
    Therefore
    \begin{align}
        \int_{-1}^{1} (f'')^2 dx &= \frac{225}{16} \int_{-1}^{1} (1 - 2 x^2 + x^4) dx\\
        &= \frac{225}{16} \left[2 - 2 \cdot \frac{2}{3} + \frac{2}{5}\right]\\
        &= \frac{225}{16} \cdot \frac{16}{15} = 15.
    \end{align}

    \textbf{4) Optimality and uniqueness}

    Let $f_0$ be the quartic above, and let $h = f - f_0$. Then $h(\pm 1) = 0$ and $\int h = 0$. We have
    $$J(f) = \int (f_0'' + h'')^2 = \int (f_0'')^2 + 2 \int f_0'' h'' + \int (h'')^2.$$
    Integrating by parts twice as before, using $f_0''(\pm 1)=0$ and $f_0^{(4)}$ constant, we get
    $$\int f_0'' h'' = \int f_0^{(4)} h = \text{const} \cdot \int h = 0.$$
    Hence
    $$J(f) = J(f_0) + \int (h'')^2 \geq J(f_0) = 15,$$
    with equality iff $h'' \equiv 0$, i.e., $h$ is linear. Since $h(\pm 1) = 0$, $h \equiv 0$, so $f \equiv f_0$. Thus the minimizer is unique.

    Therefore,
    $$\int_{-1}^{1} (f''(x))^2 dx \geq 15,$$
    with equality if and only if
    $$f(x) = \frac{-5 x^4 + 30 x^2 - 9}{16}.$$

    \textbf{Final answer:}
    \begin{itemize}
        \item The inequality holds: $\int_{-1}^{1} (f''(x))^2 dx \geq 15$.
        \item Equality occurs precisely for $f(x) = \frac{-5 x^4 + 30 x^2 - 9}{16}$.
    \end{itemize}
    \subsection{Variant 2}
    Define $w(x) = 1 - x^2$. For any twice continuously differentiable $f$ with $\int_{-1}^1 f = 0$ and $f(\pm 1) = 1$, compute
    $$\int_{-1}^1 f''(x) w(x) dx$$
    by integrating by parts twice:
    \begin{align}
        \int f'' w &= [f' w]_{-1}^1 - \int f' w'\\
        &= -[f w']_{-1}^1 + \int f w''\\
        &= -(f(1)w'(1) - f(-1)w'(-1)) + \int f (-2),
    \end{align}
    where we used $w(\pm 1) = 0$ and $w'' = -2$. Since $w'(1) = -2$, $w'(-1) = 2$, $f(\pm 1) = 1$, and $\int f = 0$, we obtain
    $$\int_{-1}^1 f''(x) (1 - x^2) dx = -(1 \cdot (-2) - 1 \cdot 2) + (-2) \cdot 0 = 4.$$

    By Cauchy--Schwarz,
    $$\left(\int f'' w\right)^2 \leq \int (f'')^2 \int w^2,$$
    hence
    $$\int_{-1}^1 (f'')^2 dx \geq \frac{(\int f'' w)^2}{\int w^2} = \frac{16}{\int_{-1}^1 (1 - x^2)^2 dx}.$$
    Compute $\int_{-1}^1 (1 - x^2)^2 dx = \int (1 - 2x^2 + x^4) dx = 2 - \frac{4}{3} + \frac{2}{5} = \frac{16}{15}$.
    Therefore,
    $$\int_{-1}^1 (f'')^2 dx \geq \frac{16}{16/15} = 15.$$

    Equality holds in Cauchy--Schwarz iff $f''$ is proportional to $w$, i.e., $f''(x) = c(1 - x^2)$. Integrating twice gives
    $$f(x) = c\left(\frac{x^2}{2} - \frac{x^4}{12}\right) + \alpha x + \beta.$$
    From $f(\pm 1) = 1$ we get $\alpha = 0$ and $\beta = 1 - \frac{5c}{12}$. The condition $\int_{-1}^1 f = 0$ yields
    $$c\left(\frac{3}{10}\right) + 2\beta = 0 \Rightarrow c = \frac{15}{4}, \text{ hence } \beta = -\frac{9}{16}.$$
    Thus the equality cases are exactly
    $$f(x) = -\frac{5}{16} x^4 + \frac{15}{8} x^2 - \frac{9}{16},$$
    for which $f''(x) = \frac{15}{4}(1 - x^2)$ and $\int (f'')^2 = 15$.

    \textbf{Final answer:}
    $$\int_{-1}^{1} (f''(x))^2 dx \geq 15,$$
    with equality if and only if $f(x) = -\frac{5}{16} x^4 + \frac{15}{8} x^2 - \frac{9}{16}$.

    \subsection{Variant 3}
    Let $g(x) = 1 - x^2$. For any twice continuously differentiable $f$ with $f(\pm 1) = 1$ and $\int_{-1}^1 f = 0$, integrate by parts twice:

    \begin{align}
        \int_{-1}^1 f''(x) g(x) \, dx &= [f'(x) g(x)]_{-1}^1 - \int_{-1}^1 f'(x) g'(x) \, dx \\
        &= 0 + 2\int_{-1}^1 x f'(x) \, dx \\
        &= [2x f(x)]_{-1}^1 - 2\int_{-1}^1 f(x) \, dx \\
        &= 2(f(1) + f(-1)) - 2\int_{-1}^1 f(x) \, dx \\
        &= 4.
    \end{align}

    By Cauchy–Schwarz,
    $$\left(\int_{-1}^1 f'' g\right)^2 \leq \int_{-1}^1 (f'')^2 \int_{-1}^1 g^2,$$
    so
    $$\int_{-1}^1 (f'')^2 \geq \frac{16}{\int_{-1}^1 (1 - x^2)^2 \, dx}.$$

    Compute
    \begin{align}
        \int_{-1}^1 (1 - x^2)^2 \, dx &= \int_{-1}^1 (1 - 2x^2 + x^4) \, dx \\
        &= 2\left(1 - \frac{2}{3} + \frac{1}{5}\right) = \frac{16}{15}.
    \end{align}

    Hence $\int_{-1}^1 (f'')^2 \geq \frac{16}{16/15} = 15$.

    Equality in Cauchy–Schwarz occurs iff $f''$ is a constant multiple of $g$, i.e., $f''(x) = \lambda(1 - x^2)$. Using $\int f'' g = 4$ and $\int g^2 = \frac{16}{15}$ gives $\lambda = \frac{15}{4}$. Integrating twice,
    $$f'(x) = \frac{15}{4}\left(x - \frac{x^3}{3}\right) + A,$$
    $$f(x) = \frac{15}{8}x^2 - \frac{15}{48}x^4 + A x + B.$$

    The boundary conditions $f(1) = f(-1) = 1$ force $A = 0$ and $B = -\frac{9}{16}$. Thus
    $$f(x) = -\frac{5}{16} x^4 + \frac{15}{8} x^2 - \frac{9}{16} = \frac{-5 x^4 + 30 x^2 - 9}{16},$$
    which also satisfies $\int_{-1}^1 f = 0$. This function yields $\int_{-1}^1 (f'')^2 = 15$.

    Therefore the inequality holds, and equality occurs precisely for $f(x) = \frac{-5 x^4 + 30 x^2 - 9}{16}$.

    \textbf{Final answer:}
    $$\int_{-1}^{1} (f''(x))^2 \, dx \geq 15, \text{ with equality iff } f(x) = \frac{-5 x^4 + 30 x^2 - 9}{16}.$$

    \subsection{Final Solution}
    Let $w(x) = 1 - x^2$. For any twice continuously differentiable $f$ with $f(\pm 1) = 1$ and $\int_{-1}^{1} f = 0$, integrate by parts twice:

    $$\int_{-1}^{1} f''(x) w(x) \, dx = [f'(x) w(x)]_{-1}^1 - \int_{-1}^{1} f'(x) w'(x) \, dx$$

    $$= 0 + 2\int_{-1}^{1} x f'(x) \, dx$$

    $$= [2x f(x)]_{-1}^1 - 2\int_{-1}^{1} f(x) \, dx$$

    $$= 2(f(1) + f(-1)) - 2 \cdot 0 = 4$$

    By Cauchy--Schwarz,
    $$\left(\int_{-1}^{1} f'' w\right)^2 \leq \int_{-1}^{1} (f'')^2 \int_{-1}^{1} w^2$$

    so
    $$\int_{-1}^{1} (f'')^2 \geq \frac{16}{\int_{-1}^{1} (1 - x^2)^2 \, dx}$$

    Since $\int_{-1}^{1} (1 - x^2)^2 \, dx = \int_{-1}^{1} (1 - 2x^2 + x^4) \, dx = 2 - \frac{4}{3} + \frac{2}{5} = \frac{16}{15}$, we get

    $$\int_{-1}^{1} (f'')^2 \, dx \geq \frac{16}{16/15} = 15$$

    Equality in Cauchy--Schwarz holds if and only if $f''$ is a constant multiple of $w$, say $f''(x) = \lambda(1 - x^2)$. Then
    $$4 = \int f'' w = \lambda \int w^2 = \lambda \cdot \frac{16}{15}$$

    so $\lambda = \frac{15}{4}$. Integrating twice,
    $$f(x) = \lambda\left(\frac{x^2}{2} - \frac{x^4}{12}\right) + Ax + B = -\frac{5}{16}x^4 + \frac{15}{8}x^2 + Ax + B$$

    The conditions $f(1) = f(-1) = 1$ force $A = 0$ and $B = 1 - \frac{5\lambda}{12} = -\frac{9}{16}$. This $f$ also satisfies $\int_{-1}^{1} f = 0$. Thus equality occurs exactly for
    $$f(x) = \frac{-5x^4 + 30x^2 - 9}{16}$$

    \textbf{Final answer:}
    $$\int_{-1}^{1} (f''(x))^2 \, dx \geq 15$$
    with equality if and only if $f(x) = \frac{-5x^4 + 30x^2 - 9}{16}$.

    \section{Problem 3}
    \subsection{Variant 1}
    Let $n = 2025$. We first characterize the set $S$.

    \textbf{Claim.} A real symmetric rank-1 matrix with entries in $\{-1, +1\}$ must be of the form
    $A = s \varepsilon \varepsilon^T$, where $s \in \{-1, +1\}$ and $\varepsilon \in \{-1, +1\}^n$.

    \textbf{Proof.} Any real symmetric rank-1 matrix has the form $A = \lambda xx^T$ for some $\lambda \neq 0$ and $x \in \mathbb{R}^n$. Then $a_{ii} = \lambda x_i^2 \in \{\pm 1\}$. Since $x_i^2 \geq 0$ and $\lambda$ is fixed, the signs of the diagonal entries cannot vary with $i$; hence all $a_{ii}$ are equal, say $a_{ii} = s \in \{\pm 1\}$ for all $i$. Thus $\lambda x_i^2 = s$ is constant, so $|x_i|$ is constant and nonzero. Write $x_i = a \varepsilon_i$ with $a > 0$ and $\varepsilon_i \in \{\pm 1\}$. Then $A_{ij} = \lambda x_i x_j = \lambda a^2 \varepsilon_i \varepsilon_j = s \varepsilon_i \varepsilon_j$. Hence $A = s \varepsilon \varepsilon^T$. Conversely, any such matrix has rank 1, is symmetric, and has entries in $\{\pm 1\}$. This proves the claim.

    Thus every $A \in S$ can be written as $A = s_a u u^T$ with $s_a \in \{\pm 1\}$ and $u \in \{-1, +1\}^n$, and similarly $B = s_b v v^T$ with $s_b \in \{\pm 1\}$ and $v \in \{-1, +1\}^n$. Note that replacing $u$ by $-u$ (or $v$ by $-v$) does not change the matrix.

    Compute
    \begin{align}
        AB &= s_a s_b u (u^T v) v^T, \\
        BA &= s_a s_b v (v^T u) u^T.
    \end{align}

    Let $c = u^T v = \sum_{i=1}^n u_i v_i$. Then $AB = s_as_b c u v^T$ and $BA = s_as_b c v u^T$. Therefore $AB = BA$ if and only if either:
    \begin{itemize}
        \item $c = 0$, in which case $AB = BA = 0$; or
        \item $u v^T = v u^T$, which holds if and only if $u$ and $v$ are proportional, i.e., $v = \pm u$ (since $u, v$ have $\pm 1$ entries).
    \end{itemize}

    Now $n = 2025$ is odd, so $c = \sum u_i v_i$ is a sum of an odd number of $\pm 1$ terms and cannot be zero. Hence the only way $AB = BA$ is when $v = \pm u$, i.e., when $B$'s sign-pattern equals $A$'s up to a global sign. The scalar signs $s_a, s_b$ do not affect commutativity.

    \textbf{Counting/probability.} The set $S$ has size $|S| = 2^n$: there are 2 choices for $s$ and $2^{n-1}$ choices for the sign-pattern class $[u] = \{u, -u\}$. For a fixed $A$, the commuting $B$ must have the same class $[v] = [u]$, and there are exactly 2 such $B$ ($s_b = \pm 1$). Thus, for each $A$, the number of commuting $B$ is 2, out of $|S| = 2^n$ possible $B$. Therefore
    $$P(AB = BA) = \frac{2}{2^n} = 2^{1-n} = 2^{-(n-1)}.$$

    With $n = 2025$, this is $2^{-2024}$.

    \textbf{Final answer.}
    $2^{-2024}$
    \subsection{Variant 2}
    Let $n=2025$. Any real symmetric rank-1 matrix with entries $\pm 1$ must be of the form
    $A = \varepsilon u u^T$, where $\varepsilon \in \{\pm 1\}$ and $u \in \{\pm 1\}^n$.
    Indeed, for a symmetric rank-1 $A = \lambda x x^T$, the condition that all entries are $\pm 1$ forces $|x_i|$ to be constant, so $x = c s$ with $s_i \in \{\pm 1\}$, and then $A = \text{sign}(\lambda) s s^T$.

    Thus choose $A = \alpha u u^T$ and $B = \beta v v^T$ with $\alpha, \beta \in \{\pm 1\}$ and $u, v \in \{\pm 1\}^n$, independently and uniformly (this yields the uniform distribution on $S$).

    Compute
    \begin{align}
        AB &= \alpha\beta (u^T v) u v^T\\
        BA &= \alpha\beta (u^T v) v u^T
    \end{align}
    Hence $AB = BA$ iff either
    \begin{itemize}
        \item $u^T v = 0$, or
        \item $u v^T = v u^T$, which holds iff $u$ and $v$ are linearly dependent, i.e., $u = \pm v$ (since entries are $\pm 1$).
    \end{itemize}

    Because $n = 2025$ is odd, $u^T v = \sum$ of 2025 terms each $\pm 1$ cannot be 0. Therefore the only commuting case is $u = \pm v$.

    With $u$ and $v$ independent uniform in $\{\pm 1\}^n$,
    \begin{align}
        P(u = v) &= 2^{-n}\\
        P(u = -v) &= 2^{-n}
    \end{align}
    so $P(u = \pm v) = 2 \cdot 2^{-n} = 2^{1-n}$.

    For $n = 2025$ this is $2^{-2024}$.

    Final answer: $2^{-2024}$

    \subsection{Variant 3}
    Let $n = 2025$. Any real symmetric rank-1 matrix $A$ can be written as $A = \lambda uu^T$ for some vector $u$ and scalar $\lambda$. If all entries of $A$ are $\pm 1$, then in particular $A_{ii} = \lambda u_i^2 \in \{\pm 1\}$ for all $i$. Since $u_i^2 > 0$, all diagonal entries must have the same sign, so $u_i^2$ is constant across $i$. Hence $|u_i|$ is constant; writing $\varepsilon_i = \text{sign}(u_i) \in \{\pm 1\}$, we get
    \[A = s \varepsilon \varepsilon^T,\]
    where $\varepsilon \in \{\pm 1\}^n$ and $s \in \{\pm 1\}$. Thus
    \[S = \{ s v v^T : v \in \{\pm 1\}^n, s \in \{\pm 1\} \},\]
    with $v$ and $-v$ giving the same matrix for fixed $s$. Therefore $|S| = 2 \cdot 2^{n-1} = 2^n$.

    Take $A = s_a a a^T$ and $B = s_b b b^T$ with $a,b \in \{\pm 1\}^n$ and $s_a,s_b \in \{\pm 1\}$. Then
    \[AB = s_as_b (a^T b) a b^T, \quad \text{and} \quad BA = s_as_b (a^T b) b a^T.\]
    Hence $AB = BA$ if and only if either $a^T b = 0$ or $a b^T = b a^T$. The latter holds if and only if $a$ and $b$ are proportional, and since $a,b$ have $\pm 1$ entries, that means $b = \pm a$.

    Now, because $n = 2025$ is odd, the inner product $a^T b$ cannot be $0$ (it equals $n - 2d$ for some integer $d$, hence is odd). Therefore $A$ and $B$ commute if and only if $b = \pm a$.

    Fix $A$. The matrices $B$ that commute with $A$ are exactly $B = \pm a a^T$, i.e., $2$ choices out of $|S| = 2^n$. Thus the desired probability is
    \[\frac{2}{2^n} = 2^{1-n} = 2^{-2024}.\]

    Final answer: $2^{-2024}$

    \subsection{Final Solution}
    Let $n = 2025$. We first characterize the set $S$ of real symmetric rank-1 matrices with entries in $\{-1, +1\}$.

    Any real symmetric rank-1 matrix has the form $A = \lambda xx^T$ with $\lambda \neq 0$. The diagonal entries are $A_{ii} = \lambda x_i^2 \in \{\pm 1\}$. Since $x_i^2 \geq 0$ and $\lambda$ is fixed, all $A_{ii}$ must have the same sign $s \in \{\pm 1\}$, so $\lambda x_i^2 = s$ for all $i$, implying $|x_i|$ is constant. Write $x = a\varepsilon$ with $a > 0$ and $\varepsilon \in \{\pm 1\}^n$. Then
    $$A = \lambda a^2 \varepsilon \varepsilon^T,$$
    and because $\lambda a^2 = s$, we get $A = s \varepsilon \varepsilon^T$.

    Conversely, any $s \varepsilon \varepsilon^T$ is symmetric, rank 1, and has entries $\pm 1$. Thus
    $$S = \{ s \varepsilon \varepsilon^T : s \in \{\pm 1\}, \varepsilon \in \{\pm 1\}^n \},$$
    with $\varepsilon$ and $-\varepsilon$ yielding the same matrix. Hence $|S| = 2 \cdot 2^{n-1} = 2^n$.

    Let $A = s_a u u^T$ and $B = s_b v v^T$ with $u, v \in \{\pm 1\}^n$ and $s_a, s_b \in \{\pm 1\}$. Then
    \begin{align}
        AB &= s_a s_b (u^T v) u v^T, \\
        BA &= s_a s_b (u^T v) v u^T.
    \end{align}
    Therefore $AB = BA$ if and only if either:
    \begin{itemize}
        \item $u^T v = 0$, or
        \item $u v^T = v u^T$, which holds if and only if $u$ and $v$ are proportional. Since $u, v$ have entries $\pm 1$, this means $v = \pm u$.
    \end{itemize}

    But $u^T v$ is the sum of $n = 2025$ odd many terms each $\pm 1$, so it cannot be zero. Thus $AB = BA$ if and only if $v = \pm u$.

    Fix $A$. The $B$ that commute with $A$ are precisely those with the same sign-pattern class $[v] = [u]$, and there are exactly two such matrices ($s_b = \pm 1$). Since $|S| = 2^n$, the desired probability is $\frac{2}{2^n} = 2^{1-n}$.

    For $n = 2025$, this equals $2^{-2024}$.

    \textbf{Final answer:} $2^{-2024}$

    \section{Problem 4}
    \subsection{Variant 1}
    Let $a$ be a fixed even positive integer, and let $b$ be any positive integer. Set
    $t = \lfloor x/a \rfloor$, so $x = at + r$ with $0 \leq r < a$.

    Rewrite the left-hand side:
    $$\left\lfloor (b^a + x)^{1/a} b^{a-1} \right\rfloor = \left\lfloor b^a \left(1 + \frac{x}{b^a}\right)^{1/a} \right\rfloor.$$

    We want
    $$\left\lfloor b^a \left(1 + \frac{x}{b^a}\right)^{1/a} \right\rfloor = b^a + t$$
    for all $b$. This is equivalent to the double inequality
    $$b^a + t \leq b^a \left(1 + \frac{x}{b^a}\right)^{1/a} < b^a + t + 1,$$
    or, after dividing by $b^a$ and raising to the power $a$ (note monotonicity since $a$ is even),
    $$\left(1 + \frac{t}{b^a}\right)^a \leq 1 + \frac{x}{b^a} < \left(1 + \frac{t+1}{b^a}\right)^a.$$

    Multiplying by $b^a$ and expanding, this becomes, for all $b \geq 1$,
    $$at + \sum_{k=2}^a \binom{a}{k} t^k b^{-a(k-1)} \leq x < a(t+1) + \sum_{k=2}^a \binom{a}{k} (t+1)^k b^{-a(k-1)}.$$

    Taking the intersection over all $b$, the left endpoints increase to their maximum at $b=1$, while the right endpoints decrease to their minimum as $b \to \infty$. Hence the condition is exactly
    $$(1+t)^a - 1 \leq x < a(t+1). \quad (1)$$

    Since $(1+t)^a - 1 \geq at$ for $t \geq 0$ (by the binomial theorem), any $x$ in (1) does satisfy $\lfloor x/a \rfloor = t$, and the above inequalities yield the desired floor identity for all $b$.

    Thus, for each integer $t \geq -1$, the admissible $x$ form the interval
    $$[(1+t)^a - 1, a(t+1)),$$
    but this interval is nonempty only when
    $$(1+t)^a - 1 < a(t+1) \iff (t+1)^a \leq a(t+1).$$

    Let $s = t+1 \geq 0$. The condition becomes $s^a \leq a s$. Therefore the solution set is the union, over all integers $s \geq 0$ with $s^a \leq a s$, of the intervals
    $$[s^a - 1, a s).$$

    Now determine which $s$ work:
    \begin{itemize}
        \item $s = 0$ gives $[-1, 0)$.
        \item $s = 1$ gives $[0, a)$.
        \item For $s \geq 2$, the condition $s^a \leq a s$ is equivalent to $s^{a-1} \leq a$. For even $a \geq 4$, we have $2^{a-1} > a$, so no $s \geq 2$ works. For $a = 2$, we have $s^{a-1} = s \leq 2$, so $s = 2$ also works, yielding $[2^2 - 1, 2 \cdot 2) = [3, 4)$.
    \end{itemize}

    Therefore:
    \begin{itemize}
        \item If $a = 2$, the solutions are $x \in [-1, 2) \cup [3, 4)$.
        \item If $a \geq 4$ is even, the solutions are $x \in [-1, a)$.
    \end{itemize}

    \textbf{Final answer:}
    \begin{itemize}
        \item For $a = 2$: all real $x$ in $[-1, 2) \cup [3, 4)$.
        \item For even $a \geq 4$: all real $x$ in $[-1, a)$.
    \end{itemize}
    \subsection{Variant 2}
    Let $a \geq 2$ be even. Write
    $x = ak + r$ with $k = \lfloor x/a \rfloor$ and $r \in [0, a)$.

    The equation is
    $$\lfloor b^{a-1} \sqrt[a]{b^a + x} \rfloor = b^a + \lfloor x/a \rfloor$$
    for every positive integer $b$. Using $\sqrt[a]{b^a + x} = b(1 + x/b^a)^{1/a}$, the left-hand side is
    $$\lfloor b^a (1 + (ak + r)/b^a)^{1/a} \rfloor.$$

    Thus the equality is equivalent to, for all $b \in \mathbb{N}$,
    $$b^a + k \leq b^a (1 + (ak + r)/b^a)^{1/a} < b^a + k + 1. \qquad (1)$$

    \textbf{Upper bound.}
    For $u \geq -1$ and $a \geq 1$, the function $t \mapsto (1 + t)^{1/a}$ is concave, so
    $$(1 + u)^{1/a} \leq 1 + u/a.$$
    Applying this with $u = (ak + r)/b^a$ yields
    $$b^a (1 + (ak + r)/b^a)^{1/a} \leq b^a + k + r/a < b^a + k + 1$$
    since $r < a$. Hence the right inequality in (1) always holds.

    \textbf{Lower bound.}
    The left inequality in (1) is equivalent (by monotonicity of $x \mapsto x^a$) to
    $$1 + (ak + r)/b^a \geq (1 + k/b^a)^a,$$
    i.e.
    $$r \geq \sum_{i=2}^a \binom{a}{i} k^i / b^{a(i-1)}. \qquad (2)$$

    Therefore the condition for all $b$ is $r \geq \sup_{b \geq 1} T_b(k)$, where
    $$T_b(k) = \sum_{i=2}^a \binom{a}{i} k^i / b^{a(i-1)}.$$

    \begin{itemize}
        \item If $k \geq 0$, each term is $\geq 0$ and decreases with $b$, so $\sup_{b \geq 1} T_b(k) = T_1(k) = (1 + k)^a - 1 - a k$.

        \item If $k = -1$, then
        $$T_b(-1) = \sum_{i=2}^a \binom{a}{i} (-1)^i / b^{a(i-1)}.$$
        Let $u = 1/b^a \in (0,1]$. Define
        $$\Phi(u) = \frac{(1 - u)^a - 1 + a u}{u} = \sum_{i=2}^a \binom{a}{i}(-1)^i u^{i-1}.$$
        Then
        $$\Phi'(u) = \frac{1 - (1 - u)^{a-1}(1 + (a - 1)u)}{u^2} \geq 0,$$
        because $g(u) = (1 - u)^{a-1}(1 + (a - 1)u)$ is decreasing on $[0,1]$ and $g(0)=1$. Hence $\Phi$ is increasing on $(0,1]$, so $\Phi(u) \leq \Phi(1) = a - 1$. Therefore $\sup_{b \geq 1} T_b(-1) = a - 1$.

        \item If $k \leq -2$, one checks directly
        $$(1 + k)^a - 1 - a k = (-1 - t)^a - 1 + a(1 + t) \text{ with } t \geq 1$$
        equals $t^a - 1 + a + a t \geq 2a$, so it already exceeds $a$.
    \end{itemize}

    Thus, a necessary and sufficient condition for (1) is
    $$r \geq F(k) := (1 + k)^a - 1 - a k, \text{ with } 0 \leq r < a. \qquad (3)$$

    Since $r < a$, this forces $F(k) < a$. We now solve $F(k) < a$ for integer $k$:

    \begin{itemize}
        \item $k = 0$: $F(0) = 0 < a$, allowed; $r \in [0, a)$, i.e. $x \in [0, a)$.

        \item $k = -1$: $F(-1) = a - 1 < a$, allowed; $r \in [a - 1, a)$, i.e. $x = -a + r \in [-1, 0)$.

        \item $k = 1$: $F(1) = 2^a - 1 - a$. This is $< a$ only when $a = 2$, giving $F(1) = 1 < 2$. Then $r \in [1, 2)$, i.e. $x \in [3, 4)$. For even $a \geq 4$, $F(1) \geq 11 \geq a$, so $k = 1$ is not allowed.

        \item $k \geq 2$ or $k \leq -2$: $F(k) \geq 2a \geq a$, not allowed.
    \end{itemize}

    \textbf{Conclusion.}
    \begin{itemize}
        \item If $a = 2$: the solutions are $x \in [-1, 2) \cup [3, 4)$.
        \item If $a \geq 4$ even: the solutions are $x \in [-1, a)$.
    \end{itemize}

    \textbf{Final answer.}
    \begin{itemize}
        \item For $a = 2$: all real $x$ in $[-1, 2) \cup [3, 4)$.
        \item For even $a \geq 4$: all real $x$ in $[-1, a)$.
    \end{itemize}

    \subsection{Variant 3}
    Let $a$ be an even positive integer and let $x$ be real. For the expression to be defined for every positive integer $b$, we must have $b^a + x \geq 0$ for $b = 1$, hence $x \geq -1$.

    Rewrite the left-hand side:
    $$\sqrt[a]{b^a + x} \cdot b^{a-1} = b^a (1 + x/b^a)^{1/a}.$$
    Therefore the equality is equivalent to
    $$\lfloor b^a (1 + x/b^a)^{1/a} \rfloor = b^a + \lfloor x/a \rfloor.$$
    Subtract $b^a$ from both sides (noting $b^a$ is an integer):
    $$\lfloor b^a((1 + x/b^a)^{1/a} - 1) \rfloor = \lfloor x/a \rfloor. \quad (1)$$

    Define
    $$F_b(x) = b^a((1 + x/b^a)^{1/a} - 1).$$
    By concavity of $t \mapsto t^{1/a}$ on $(0,\infty)$ (since $1/a \in (0,1]$), we have for $y \geq -1$:
    $$(1 + y)^{1/a} \leq 1 + y/a,$$
    so $F_b(x) \leq x/a$. Thus $\lfloor F_b(x) \rfloor \leq \lfloor x/a \rfloor$. To obtain equality in (1), it suffices to ensure $F_b(x) \geq \lfloor x/a \rfloor$ for all $b$.

    \textbf{Monotonicity in $b$.} Write
    $$F_b(x) = x \cdot H(1 + x/b^a),$$
    where $H(s) = (s^{1/a} - 1)/(s - 1)$ for $s > 0$, $s \neq 1$, and $H(1) = 1/a$.
    Since $t \mapsto t^{1/a}$ is concave, the chord slope $H(s)$ is decreasing in $s$ on $(0,\infty)$. As $b$ increases, $1 + x/b^a$ moves monotonically to 1, and from the monotonicity of $H$ we get that $F_b(x)$ is increasing in $b$ (for both signs of $x$). Hence
    $$\min_{b\geq 1} F_b(x) = F_1(x) = (1 + x)^{1/a} - 1,$$
    and $F_b(x) \leq x/a$ for all $b$.

    Therefore, (1) holds for all $b$ if and only if
    $$(1 + x)^{1/a} - 1 \geq \lfloor x/a \rfloor, \text{ with } x \geq -1. \quad (2)$$

    Let $k = \lfloor x/a \rfloor$, so $x \in [ak, ak + a)$. We analyze (2) by cases.

    \begin{itemize}
        \item If $k \leq -1$: then $1 + k \leq 0$ and since $(1 + x)^{1/a} \geq 0$, the inequality in (2) holds automatically. Intersecting with $x \geq -1$ forces $k = -1$ and $x \in [-1, 0)$.

        \item If $k \geq 0$: then (2) is equivalent to $(1 + x)^{1/a} \geq 1 + k$, i.e.
        $$x \geq (1 + k)^a - 1,$$
        together with $x \in [ak, ak + a)$. Thus we need the interval
        $$[(1 + k)^a - 1, ak + a)$$
        to be nonempty.
    \end{itemize}

    For $k = 0$, this gives $x \in [0, a)$.

    For $k \geq 1$, note that
    $$(1 + k)^a - 1 = ak + C,$$
    where $C = \sum_{j=2}^a \binom{a}{j} k^j \geq \binom{a}{2} k^2$.
    Hence for $k \geq 1$:
    $$(1 + k)^a - 1 \geq ak + \binom{a}{2} \geq ak + a \text{ if } a \geq 4,$$
    so the interval is empty for all $k \geq 1$ when $a \geq 4$.

    For $a = 2$, the condition becomes $(1 + k)^2 - 1 < 2(k + 1)$, i.e. $k^2 < 2$, so $k = 0$ or $k = 1$. We already have $k = 0$ giving $[0, 2)$; for $k = 1$ we get $x \in [3, 4)$.

    Collecting all cases:
    \begin{itemize}
        \item If $a = 2$: $x \in [-1, 2) \cup [3, 4)$.
        \item If $a \geq 4$ even: $x \in [-1, a)$.
    \end{itemize}

    It is straightforward to check that these $x$ indeed satisfy the original identity for all $b$, using the monotonicity of $F_b(x)$ and the bounds $F_b(x) \in [F_1(x), x/a)$.

    \textbf{Final answer.}
    \begin{itemize}
        \item For $a = 2$: all real $x$ in $[-1, 2) \cup [3, 4)$.
        \item For even $a \geq 4$: all real $x$ in $[-1, a)$.
    \end{itemize}

    \subsection{Final Solution}
    Let $a$ be a fixed even positive integer and $b$ a positive integer. Write $x = ak + r$ with $k = \lfloor x/a \rfloor$ and $0 \leq r < a$. Note first that $b^a + x \geq 0$ for $b = 1$ forces $x \geq -1$.

    We have
    $$\lfloor b^{a-1}(b^a + x)^{1/a} \rfloor = \lfloor b^a(1 + x/b^a)^{1/a} \rfloor = b^a + k$$
    if and only if, for all $b \in \mathbb{N}$,
    $$b^a + k \leq b^a(1 + (ak + r)/b^a)^{1/a} < b^a + k + 1. \quad (1)$$

    \textbf{Upper bound.} Since $t \mapsto (1 + t)^{1/a}$ is concave on $[-1, \infty)$,
    $$(1 + u)^{1/a} \leq 1 + u/a \quad (u \geq -1).$$
    With $u = (ak + r)/b^a$ we get
    $$b^a(1 + u)^{1/a} \leq b^a + k + r/a < b^a + k + 1,$$
    so the right inequality in $(1)$ always holds.

    \textbf{Lower bound.} The left inequality in $(1)$ is equivalent to
    \begin{align}
        1 + (ak + r)/b^a &\geq (1 + k/b^a)^a \\
        &\Leftrightarrow r \geq \sum_{i=2}^a \binom{a}{i} k^i b^{-a(i-1)} =: T_b(k). \quad (2)
    \end{align}

    Thus $(1)$ holds for all $b$ iff $r \geq \sup_{b \geq 1} T_b(k)$. We evaluate this supremum by cases.

    \begin{itemize}
        \item \textbf{$k \geq 0$.} All terms in $T_b(k)$ are $\geq 0$ and decrease with $b$, hence
        $$\sup_{b \geq 1} T_b(k) = T_1(k) = (1 + k)^a - 1 - ak =: F(k).$$
        So we need $r \geq F(k)$ with $0 \leq r < a$, i.e. $F(k) < a$. Since $F$ is increasing on $k \geq 0$,
        $F(0) = 0$ and $F(1) = 2^a - 1 - a$. For $a = 2$, $F(1) = 1 < 2$, while for even $a \geq 4$, $F(1) \geq 11 \geq a$. Hence:
        \begin{itemize}
            \item $k = 0$ is always allowed ($r \in [0, a) \Rightarrow x \in [0, a)$);
            \item $k = 1$ is allowed only if $a = 2$ (then $r \in [1, 2) \Rightarrow x \in [3, 4)$);
            \item $k \geq 2$ is never allowed ($F(k) \geq F(1) \geq a$).
        \end{itemize}

        \item \textbf{$k = -1$.} Here
        $$T_b(-1) = \sum_{i=2}^a \binom{a}{i}(-1)^i b^{-a(i-1)}.$$
        Let $u = 1/b^a \in (0, 1]$, and define
        $$\Phi(u) = \frac{(1 - u)^a - 1 + au}{u} = \sum_{i=2}^a \binom{a}{i}(-1)^i u^{i-1} = T_b(-1).$$
        Compute
        $$\Phi'(u) = \frac{1 - (1 - u)^{a-1}(1 + (a - 1)u)}{u^2} \geq 0$$
        since $(1 - u)^{a-1}(1 + (a - 1)u)$ decreases on $[0,1]$. Thus $\Phi$ is increasing, so
        $$\sup_{b \geq 1} T_b(-1) = \Phi(1) = a - 1.$$
        Therefore $r \geq a - 1$, i.e. $r \in [a - 1, a)$, giving $x \in [-1, 0)$.

        \item \textbf{$k \leq -2$.} Then $T_1(k) = (1 + k)^a - 1 - ak \geq 2a \geq a$ (since $a$ is even), which cannot be $\leq r < a$. Hence no solutions.
    \end{itemize}

    Collecting all possibilities:
    \begin{itemize}
        \item For any even $a \geq 2$: $x \in [-1, 0)$ (from $k = -1$) and $x \in [0, a)$ (from $k = 0$).
        \item Additionally, when $a = 2$: $x \in [3, 4)$ (from $k = 1$).
    \end{itemize}

    It remains to note that for these $x$ the lower bound $r \geq \sup_b T_b(k)$ ensures the left inequality in $(1)$, and the concavity estimate ensures the right inequality; hence the original floor identity holds for every $b$.

    \textbf{Final answer.}
    \begin{itemize}
        \item If $a = 2$: all real $x$ in $[-1, 2) \cup [3, 4)$.
        \item If $a \geq 4$ is even: all real $x$ in $[-1, a)$.
    \end{itemize}

    \section{Problem 5}
    \subsection{Variant 1}
    \textbf{Step 1. The eventual shape of a self-map and the exact value of} $\mathrm{ord}(\tau)$.

    Let $\tau \in T_n$. Its functional digraph is a disjoint union of directed cycles, with rooted in-trees feeding into the cycle vertices. Let:
    \begin{itemize}
        \item $c =$ number of vertices that lie on cycles,
        \item $h =$ maximal distance (height) from a vertex to its cycle,
        \item $\ell_1, \ldots, \ell_r$ be the lengths of the cycles, and $L = \mathrm{lcm}(\ell_1, \ldots, \ell_r)$.
    \end{itemize}

    Then the sequence of maps $\tau, \tau^2, \ldots$ is eventually periodic with preperiod length $h$ and period $L$, and in fact
    $$\mathrm{ord}(\tau) = h + L.$$

    \textbf{Proof of} $\mathrm{ord}(\tau) = h + L$:
    \begin{itemize}
        \item For $0 \leq k < h$, the maps $\tau^k$ are all distinct, since for each $k$ there exists a vertex at depth $\geq k + 1$ whose image under $\tau^k$ is still in the tree, whereas $\tau^{k+1}$ pushes it one step further.
        \item For $h \leq k < h + L$, the maps $\tau^k$ are all distinct and $\tau^{k+L} = \tau^k$, because once all points have reached their cycles, the action depends on $k$ modulo $L$.
        \item No $\tau^k$ with $k < h$ equals any $\tau^m$ with $m \geq h$, since the image of a vertex of maximum depth under $\tau^k$ is not on a cycle, while under $\tau^m$ it is on a cycle.
    \end{itemize}

    Thus $\mathrm{ord}(\tau) = h + L$.

    \textbf{Step 2. A reduction: an exact formula for} $g(n)$.

    From Step 1, for any $\tau$ we have $\mathrm{ord}(\tau) = h + L$, where $L$ is the lcm of cycle lengths (over the $c$ cycle vertices), and $h \leq n - c$. For fixed $c$, the maximal possible $L$ among permutations on $c$ letters is $f(c)$. Hence
    $$\mathrm{ord}(\tau) \leq (n - c) + f(c).$$

    Conversely, this bound is attained: choose a permutation on $c$ with order $f(c)$, and let the remaining $t = n - c$ vertices form a single directed path feeding into one vertex on some cycle. Then $h = t$ and $L = f(c)$, so $\mathrm{ord}(\tau) = t + f(c)$.

    Therefore
    $$g(n) = \max_{0 \leq t \leq n} [t + f(n - t)].$$

    \textbf{Step 3. A number-theoretic lemma.}

    We prove that for sufficiently large $n$, and for any integer $s$ with $s \geq n^{0.501}$,
    $$f(n) - f(n - s) \geq s - n^{0.501} - 1. \quad (\star)$$

    \textbf{Proof.} Let $c = n - s$, and let $L^* = f(c)$, realized by some permutation on $c$ points. Let $u = \lceil n^{0.501} \rceil$ and consider the set $P$ of primes in the interval $(s - u, s]$. A standard consequence of the prime number theorem (via partial summation) is that the sum $S(x)$ of primes $\leq x$ satisfies $S(x) \sim x^2/(2 \log x)$. In particular, there exists a constant $K > 0$ such that for all sufficiently large $x$ and all $1 \leq y \leq x$,
    $$\sum_{x-y < p \leq x} p \geq K x y / \log x.$$

    Applying this with $x = s$ and $y = u$, for large $n$ we get
    $$\sum_{p \in P} p \geq K s u / \log s \gg n^{1.002}/\log n > n \geq c.$$

    Now, if every prime $p \in P$ divides $L^*$, then in the permutation of size $c$ we would need to include each such prime as a factor of some cycle length. Covering a prime $p$ requires at least $p$ points in the cycles (because some cycle length must be a multiple of $p$, hence $\geq p$). Covering several primes with a single cycle does not reduce this cost (a multiple of distinct primes $p$ and $q$ has length at least $pq \geq p + q$). Therefore the total number of cycle vertices needed to include all primes in $P$ is at least $\sum_{p \in P} p$, which exceeds $c$ for large $n$. This is impossible. Hence there exists a prime $m \in P$ that does not divide $L^*$.

    Using the remaining $s$ points, form one additional cycle of length $m$ (and make the other $s - m$ points fixed). Then on $n$ points we obtain a permutation whose order is $\mathrm{lcm}(L^*, m) = L^* \cdot m$ (because $\gcd(L^*, m) = 1$). Thus
    $$f(n) \geq L^* \cdot m \geq L^* + m - 1 \geq f(n - s) + (s - u) - 1,$$
    since $m \geq s - u$ and $L^* \geq 1$, which proves $(\star)$.

    \textbf{Step 4. Conclusion.}

    Recall $g(n) = \max_{0 \leq t \leq n} [t + f(n - t)]$.
    \begin{itemize}
        \item If $t \leq n^{0.501}$, then $t + f(n - t) \leq f(n) + n^{0.501}$.
        \item If $t \geq n^{0.501}$, apply $(\star)$ with $s = t$ to get
        $$f(n - t) \leq f(n) - (t - n^{0.501} - 1),$$
        so $t + f(n - t) \leq f(n) + n^{0.501} + 1$.
    \end{itemize}

    Therefore, for sufficiently large $n$,
    $$g(n) \leq f(n) + \lfloor n^{0.501} \rfloor + 1 < f(n) + n^{0.501}.$$

    This proves the desired inequality.

    \textbf{Final answer.}
    For sufficiently large $n$,
    $$g(n) < f(n) + n^{0.501}.$$

    \subsection{Variant 2}
    Let $\tau \in T_n$. View $\tau$ as a functional digraph on $[n]$, i.e., a disjoint union of directed cycles with rooted trees feeding into those cycles.

    Write:
    \begin{itemize}
        \item $C =$ set of cyclic points of $\tau$, $m = |C|$.
        \item $t = n - m =$ number of non-cyclic points.
        \item $L = \text{lcm}$ of the cycle lengths on $C$ (so $L$ is the order of the restriction of $\tau$ to $C$). In particular $L \leq f(m)$.
        \item $h =$ maximum distance (height) of a non-cyclic point to its eventual cycle. Then $h \leq t$.
    \end{itemize}

    Two basic facts about the iterates $\tau^k$:
    \begin{enumerate}
        \item After $h$ steps, every point has reached a cycle, so from that time on the restriction to $C$ is a permutation of $C$ with period $L$. Hence the sequence of maps $\tau, \tau^2, \ldots$ becomes periodic after at most $h$ steps with period dividing $L$.
        \item Therefore the number of distinct maps among $\{\tau, \tau^2, \tau^3, \ldots\}$ is at most $h + L$.
    \end{enumerate}

    Thus
    $$\text{ord}(\tau) \leq h + L \leq t + f(m). \quad (1)$$

    We will compare $f(m)$ with $f(n)$ using a simple prime-counting argument.

    A prime in $(t/2, t]$ cannot appear together with a distinct prime in the same cycle length (for large $n$) because the product of two distinct primes $p, q > t/2$ is $> t^2/4$, which exceeds $m$ for $t \geq n^{0.501}$ and all sufficiently large $n$ (since then $t^2/4 \geq n^{1.002}/4 > n \geq m$). Hence, each distinct prime $p \in (t/2, t]$ dividing the order $L$ must be supported by a different cycle of length at least $p \geq t/2$, consuming at least $t/2$ points. Therefore, the number $r$ of primes in $(t/2, t]$ dividing $L$ satisfies
    $$r \leq 2m/t \leq 2n/t. \quad (2)$$

    On the other hand, by standard prime number estimates, for all sufficiently large $x$,
    $$\pi(x) - \pi(x/2) \geq c \frac{x}{\log x}$$
    for some absolute constant $c > 0$. Taking $x = t$ (and recalling $t \geq n^{0.501}$), we get
    $$\pi(t) - \pi(t/2) \gg t / \log t.$$
    For large $n$ this quantity exceeds $2n/t$ (because $t/\log t \gg n^{0.501} / \log n \gg n^{0.499}$), so by (2) there exists a prime $p \in (t/2, t]$ that does not divide $L$.

    Now extend any permutation on $m$ points of order $f(m)$ to a permutation on $n$ points by using the remaining $t$ points to form one $p$-cycle. Since $p \nmid f(m)$, the new order is $\text{lcm}(f(m), p) = p f(m)$. Hence
    $$f(n) \geq p f(m) \geq (t/2) f(m),$$
    so
    $$f(m) \leq 2 f(n) / t. \quad (3)$$

    Combining (1) and (3), we obtain for $t \geq n^{0.501}$ (and hence $t \geq 2$ for large $n$),
    $$\text{ord}(\tau) \leq t + f(m) \leq t + 2 f(n)/t \leq t + f(n) \leq f(n) + n^{0.501}.$$

    If instead $t < n^{0.501}$, then from (1) and monotonicity $f(m) \leq f(n)$,
    $$\text{ord}(\tau) \leq t + f(m) \leq f(n) + t < f(n) + n^{0.501}.$$

    In either case, for sufficiently large $n$,
    $$\text{ord}(\tau) < f(n) + n^{0.501}.$$

    Taking the maximum over $\tau \in T_n$ yields
    $$g(n) < f(n) + n^{0.501}$$
    for all sufficiently large $n$.

    \textbf{Final answer.}
    For all sufficiently large $n$, $g(n) < f(n) + n^{0.501}$.

    \subsection{Variant 3}
    Let $[n] = \{1,2,\ldots,n\}$. For a self-map $\tau$ on $[n]$, its functional digraph decomposes into directed cycles with rooted trees feeding into the cycles. Denote

    \begin{itemize}
        \item $C(\tau)$ the set of cyclic points, $m = |C(\tau)|$.
        \item $h(\tau)$ the maximal distance (height) from a point to the cycles (i.e., the longest tail length).
        \item $L(\tau)$ the least common multiple of the cycle-lengths of $\tau$ (i.e., the order of $\tau$ restricted to $C(\tau)$, which is a permutation).
    \end{itemize}

    \textbf{1) The number of distinct iterates of a map}

    \textbf{Claim.} For any $\tau \in T_n$,
    $$\text{ord}(\tau) = L(\tau) + \max(0, h(\tau) - 1).$$

    \textbf{Proof.} Consider the sequence $\tau, \tau^2, \tau^3, \ldots$. For $k \geq h := h(\tau)$, every point is on some cycle after applying $\tau^k$, hence $\tau^k$ restricted to $C(\tau)$ is a power of the corresponding permutation, and $\tau^k$ depends only on $k$ modulo $L := L(\tau)$. Thus among $k \geq h$ there are exactly $L$ distinct maps: $\tau^h, \tau^{h+1}, \ldots, \tau^{h+L-1}$, and thereafter they repeat with period $L$.

    For $1 \leq k \leq h - 1$, these maps are all distinct and none coincides with any $\tau^{h+r}$. Indeed, pick $x$ at maximal distance $h$ from the cycles; then $\tau^k(x)$ has distance $h - k \geq 1$ from the cycles, whereas for any $r \geq 0$, $\tau^{h+r}(x)$ lies on a cycle. Hence $\tau^k \neq \tau^{h+r}$. Therefore the total number of distinct maps is $(h - 1) + L$ if $h \geq 1$, and $L$ if $h = 0$. This is exactly $L + \max(0, h - 1)$. $\square$

    \textbf{2) A formula for $g(n)$}

    If $\tau$ has $m$ cyclic points, then $h(\tau) \leq n - m$. By the claim,
    $$\text{ord}(\tau) \leq L(\tau) + \max(0, n - m - 1) \leq f(m) + \max(0, n - m - 1).$$

    Conversely, for any $m$ between $0$ and $n$, choose a permutation $\sigma \in S_m$ of order $f(m)$ on some $m$-element subset $C$, and attach to one chosen cycle-vertex a directed chain of length $n - m$ using the other $n - m$ points. The resulting map $\tau$ has $L(\tau) = f(m)$ and $h(\tau) = n - m$, hence $\text{ord}(\tau) = f(m) + \max(0, n - m - 1)$.

    Therefore
    $$g(n) = \max_{0 \leq m \leq n} \left[ f(m) + \max(0, n - m - 1) \right].$$

    \textbf{3) Bounding $g(n)$ in terms of $f(n)$}

    Let $t = \lfloor n^{0.501} \rfloor$.

    \begin{itemize}
        \item If $m \geq n - t$, then $f(m) + \max(0, n - m - 1) \leq f(n) + t \leq f(n) + n^{0.501}$.

        \item If $m \leq n - t - 1$, we will show that for all sufficiently large $n$,
        $$f(m) + (n - m - 1) \leq f(n).$$
    \end{itemize}

    It is a classical result (Landau) that
    $$\log f(k) = (1 + o(1)) \sqrt{k \log k}.$$
    In particular, for any $\alpha > 1/2$ we have $f(n - n^\alpha) = o(f(n))$ as $n \to \infty$. Taking $\alpha = 0.501$, there exists $N_1$ such that for $n \geq N_1$,
    $$f(n - t) \leq \frac{1}{2} f(n).$$
    Also, since $f(n)$ grows super-polynomially, there exists $N_2$ such that for $n \geq N_2$, $f(n) \geq 2n$. For $m \leq n - t$ we then have
    $$f(m) + (n - m - 1) \leq f(n - t) + n \leq \frac{1}{2} f(n) + \frac{1}{2} f(n) = f(n).$$

    Combining both cases, for $n \geq \max(N_1, N_2)$,
    $$g(n) \leq \max\{ f(n), f(n) + t \} = f(n) + t \leq f(n) + n^{0.501}.$$

    Since $g(n)$ is an integer, we in fact have
    $$g(n) \leq f(n) + \lfloor n^{0.501} \rfloor < f(n) + n^{0.501}.$$

    This proves the desired inequality for all sufficiently large $n$.

    \textbf{Final answer.}
    For sufficiently large $n$, $g(n) < f(n) + n^{0.501}$.

    \subsection{Final Solution}
    \textbf{Solution.}

    \textbf{1) Dynamics of a self-map and} $\mathbf{\text{ord}(\tau)}$

    Let $\tau \in T_n$. Its functional digraph is a disjoint union of directed cycles with rooted trees feeding into the cycles. Let:
    \begin{itemize}
        \item $L(\tau)$ be the least common multiple of the cycle lengths;
        \item $h(\tau)$ be the maximal distance (height) from a vertex to its eventual cycle (i.e., the longest tail length).
    \end{itemize}

    \textbf{Claim.} For any $\tau \in T_n$,
    $$\text{ord}(\tau) = L(\tau) + \max(0, h(\tau) - 1).$$

    \textbf{Proof.} For $k \geq h := h(\tau)$, every point lies on a cycle after applying $\tau^k$, so $\tau^k$ depends only on $k$ modulo $L := L(\tau)$. Hence among $\{\tau^h, \tau^{h+1}, \ldots\}$ there are exactly $L$ distinct maps. For $1 \leq k \leq h - 1$, the maps $\tau, \tau^2, \ldots, \tau^{h-1}$ are pairwise distinct and none coincides with any $\tau^{h+r}$ ($r \geq 0$): if $x$ is a vertex at maximal distance $h$ from the cycles, then $\tau^k(x)$ is not yet on a cycle for $k \leq h - 1$, whereas $\tau^{h+r}(x)$ is on a cycle. Summing gives $\text{ord}(\tau) = L + \max(0, h - 1)$. $\square$

    \textbf{2) A formula for} $\mathbf{g(n)}$

    Let $m$ be the number of cyclic points of $\tau$, so $h(\tau) \leq n - m$, and $L(\tau) \leq f(m)$. By the claim,
    $$\text{ord}(\tau) \leq f(m) + \max(0, n - m - 1).$$

    Conversely, for each $m \in \{1, \ldots, n\}$, choose a permutation on an $m$-element set with order $f(m)$, and attach a directed chain of length $n - m$ feeding into one point on a cycle. This has $L(\tau) = f(m)$ and $h(\tau) = n - m$, hence $\text{ord}(\tau) = f(m) + \max(0, n - m - 1)$. Therefore
    $$g(n) = \max_{1 \leq m \leq n} \left[ f(m) + \max(0, n - m - 1) \right].$$

    \textbf{3) Bounding} $\mathbf{g(n)}$ \textbf{in terms of} $\mathbf{f(n)}$

    Let $t = \lfloor n^{0.501} \rfloor$. Split the maximization into two ranges of $m$.

    \begin{itemize}
        \item If $m \geq n - t$, then
        $$f(m) + \max(0, n - m - 1) \leq f(n) + t - 1 \leq f(n) + t.$$

        \item If $m \leq n - t - 1$, write $d = n - m \geq t + 1$. We use Landau's classical asymptotic for the maximal order of a permutation:
        $$\log f(k) = (1 + o(1)) \sqrt{k \log k} \text{ as } k \to \infty.$$
        In particular, for any $\alpha > 1/2$ we have $f(n - n^\alpha) = o(f(n))$. Taking $\alpha = 0.501$, there exists $N_1$ such that for $n \geq N_1$,
        $$f(n - t) \leq \frac{1}{2} f(n).$$
        Moreover, $f(n)$ grows super-polynomially, so there exists $N_2$ such that for $n \geq N_2$, $f(n) \geq 2n$. Hence for $n \geq \max(N_1, N_2)$ and all $m \leq n - t$ we have
        $$f(m) + (n - m - 1) \leq f(n - t) + n \leq \frac{1}{2} f(n) + \frac{1}{2} f(n) = f(n).$$
    \end{itemize}

    Combining the two cases, for all sufficiently large $n$,
    $$g(n) \leq f(n) + t < f(n) + n^{0.501}.$$

    \textbf{Final answer.}
    For sufficiently large $n$, $g(n) < f(n) + n^{0.501}$.

    \section{Problem 6, Day 2}
    \subsection{Variant 1}
    Define the function
    $$\varphi(x) = \frac{f(x) - k}{x}, \quad \text{for } x \in [a,b].$$

    Since $a, b > 0$ and $f$ is continuously differentiable on $(0,\infty)$, $\varphi$ is continuous on $[a,b]$ and differentiable on $(a,b)$. Also, using $f(a) = f(b) = k$, we have $\varphi(a) = \varphi(b) = 0$. By Rolle's theorem, there exists $\xi \in (a,b)$ such that $\varphi'(\xi) = 0$.

    Compute the derivative:
    $$\varphi'(x) = \frac{x f'(x) - (f(x) - k)}{x^2}.$$

    Thus $\varphi'(\xi) = 0$ implies
    $$x f'(\xi) - (f(\xi) - k) = 0,$$
    i.e.
    $$f(\xi) - \xi f'(\xi) = k.$$

    This proves the claim.

    \textbf{Final answer:}
    There exists $\xi \in (a,b)$ such that $f(\xi) - \xi f'(\xi) = k$.
    \subsection{Variant 2}
    Define the function
    $$g(x) = \frac{f(x) - k}{x}, \quad \text{for } x \in [a,b].$$

    Since $a, b > 0$ and $f$ is continuously differentiable on $(0, \infty)$, the function $g$ is continuous on $[a,b]$ and differentiable on $(a,b)$. Moreover,
    $$g(a) = \frac{f(a) - k}{a} = 0 \quad \text{and} \quad g(b) = \frac{f(b) - k}{b} = 0.$$

    By Rolle's theorem, there exists $\xi \in (a,b)$ such that $g'(\xi) = 0$. Compute
    $$g'(x) = \frac{x f'(x) - (f(x) - k)}{x^2}.$$

    Thus $g'(\xi) = 0$ implies $x f'(\xi) - (f(\xi) - k) = 0$ at $x = \xi$, i.e.,
    $$f(\xi) - \xi f'(\xi) = k.$$

    \textbf{Final answer:} There exists $\xi \in (a,b)$ with $f(\xi) - \xi f'(\xi) = k$.

    \subsection{Variant 3}
    Define $g(x) = \frac{f(x) - k}{x}$ on $[a,b]$. This is well-defined since $a,b > 0$, continuous on $[a,b]$, and differentiable on $(a,b)$ because $f$ is continuously differentiable.

    Given $f(a) = f(b) = k$, we have $g(a) = g(b) = 0$. By Rolle's theorem, there exists $\xi \in (a,b)$ such that $g'(\xi) = 0$.

    Compute
    \[g'(x) = \frac{x f'(x) - (f(x) - k)}{x^2}.\]

    Thus $g'(\xi) = 0$ implies $\xi f'(\xi) - f(\xi) + k = 0$, i.e.,
    \[f(\xi) - \xi f'(\xi) = k.\]

    \textbf{Final answer:} There exists $\xi \in (a,b)$ with $f(\xi) - \xi f'(\xi) = k$.
    \subsection{Final Solution}
    Define $g(x) = \frac{f(x) - k}{x}$ for $x \in [a,b]$. Since $a,b > 0$ and $f$ is continuously differentiable on $(0,\infty)$, $g$ is continuous on $[a,b]$ and differentiable on $(a,b)$. Using $f(a) = f(b) = k$, we have $g(a) = g(b) = 0$. By Rolle's theorem, there exists $\xi \in (a,b)$ such that $g'(\xi) = 0$.

    Compute
    $$g'(x) = \frac{x f'(x) - (f(x) - k)}{x^2}.$$
    Thus $g'(\xi) = 0$ implies
    $$\xi f'(\xi) - (f(\xi) - k) = 0,$$
    i.e.,
    $$f(\xi) - \xi f'(\xi) = k.$$

    Final answer: There exists $\xi \in (a,b)$ such that $f(\xi) - \xi f'(\xi) = k$.

    \section{Problem 7}
    \subsection{Variant 1}
    Write $v_2(n)$ for the exponent of 2 in $n$, and $\text{odd}(n) = n/2^{v_2(n)}$ for the odd part of $n$.

    We claim that the sets
    $$M_d = \{ n \in \mathbb{Z}_{>0} : d \text{ divides } \text{odd}(n) \}$$
    for odd $d \geq 1$ are precisely the nonempty subsets $M \subseteq \mathbb{Z}_{>0}$ satisfying (a) and (b).

    First, each $M_d$ satisfies (a) and (b):
    \begin{itemize}
        \item[(a)] If $d \mid \text{odd}(x)$, then $d \mid \text{odd}(2x)$ because doubling does not change the odd part.
        \item[(b)] Let $x = 2^a d u$, $y = 2^b d v$ with $u, v$ odd. If $x$ and $y$ have the same parity, then $(x+y)/2 = 2^{\min(a,b)-1} d (u 2^{a-\min} + v 2^{b-\min})$. Thus $\text{odd}((x+y)/2) = d \cdot \text{odd}(u 2^{a-\min} + v 2^{b-\min})$, so $d$ divides the odd part; hence $(x+y)/2 \in M_d$.
    \end{itemize}

    So each $M_d$ is a valid solution.

    Conversely, let $M$ be a nonempty subset satisfying (a) and (b). We prove $M = M_d$ for a suitable odd $d$.

    \textbf{1)} $M$ contains odd numbers. Let $t$ be the minimal $v_2$ among elements of $M$, and pick $x \in M$ with $v_2(x) = t$. If $t = 0$, we are done. If $t \geq 1$, then $x$ and $2x$ are both in $M$ and even, so $(x+2x)/2 = 3x/2 \in M$ and $v_2(3x/2) = t-1$. Iterating reduces $v_2$ until we obtain an odd element of $M$.

    Let $d$ be the smallest odd element of $M$.

    \textbf{2)} From $d$ we can generate all odd multiples of $d$. Let $O$ be the set of odd numbers in $M$. We show by induction on odd $m \geq 1$ that $dm \in O$:
    \begin{itemize}
        \item Base: $m = 1$ gives $d \in O$.
        \item Step: Write $m = 1 + 2^t m'$ with $m'$ odd and $m' < m$. By induction, $dm' \in O$. Using (a), $2d, 2^{t+1} dm' \in M$. They are even, so by (b),
        $$\frac{2d + 2^{t+1} dm'}{2} = d + 2^t dm' = dm \in M,$$
        and since it is odd, $dm \in O$.
    \end{itemize}

    Thus $O$ contains all odd multiples of $d$. Using (a), $M$ contains all $2^k$ times these, i.e., $M \supseteq M_d$.

    \textbf{3)} No element outside $M_d$ can lie in $M$. Suppose $x \in M$ with $\text{odd}(x)$ not divisible by $d$. Let $u = \text{odd}(x)$, and let $k = v_2(x)$. As in step 1's reduction trick but ``shifted'' by $d$, define a sequence
    $$s_0 = x, \quad \text{and for } j = 0,1,\ldots,k \text{ set } s_{j+1} = \frac{s_j + 2^{k-j+1} d}{2}.$$
    A direct computation shows $s_{j+1} = 2^{k-j-1} (u + 2^{j+1} d)$, hence $s_k = u + 2kd$, which is an odd element of $M$. In particular, $M$ contains an odd element $w \equiv u \pmod{d}$, hence $w$ is not divisible by $d$.

    Now consider the odd sum $d + w$. Let $t = v_2(d + w) \geq 1$ (since both are odd). By repeatedly applying (b) to suitable even multiples (which is allowed by (a)), we can divide out all 2's from $d + w$ and obtain the odd number
    $$r = \frac{d + w}{2^t} \in M.$$
    Note that $\gcd(d, w) = \gcd(d, r)$, and since $d \nmid w$, we have $\gcd(d, w) < d$, hence $r < d$. This contradicts the minimality of $d$ in the set of odd elements of $M$.

    Therefore every $x \in M$ must satisfy $d \mid \text{odd}(x)$, i.e., $M \subseteq M_d$.

    Combining 2) and 3), we get $M = M_d$.

    \textbf{Conclusion.} The nonempty subsets $M \subseteq \mathbb{Z}_{>0}$ that satisfy (a) and (b) are exactly the sets
    $$M_d = \{ n \in \mathbb{Z}_{>0} : d \text{ divides the odd part of } n \}$$
    for some odd integer $d \geq 1$.

    \textbf{Final answer.}
    For each odd $d \geq 1$, the set $M_d = \{ n \in \mathbb{Z}_{>0} : d \mid \text{odd}(n) \}$ is a solution, and these are all the solutions.
    \subsection{Variant 2}
    Let $M \subseteq \mathbb{Z}_{>0}$ be nonempty and satisfy:
    \begin{enumerate}
        \item[(a)] $x \in M \Rightarrow 2x \in M$.
        \item[(b)] If $x,y \in M$ and $x+y$ is even, then $\frac{x+y}{2} \in M$.
    \end{enumerate}

    \textbf{Step 1.} $M$ is closed under addition.

    Indeed, for any $a,b \in M$, we have $2a,2b \in M$ by (a). Then $2a+2b$ is even and
    $\frac{2a+2b}{2} = a+b \in M$ by (b).

    \textbf{Step 2.} $M$ contains an odd element.

    Let $t = \min\{v_2(x) : x \in M\}$, where $v_2$ is the 2-adic valuation. If $t \geq 1$, pick $x \in M$ with $v_2(x)=t$. Then $2x \in M$ and $x+2x$ is even, so $\frac{x+2x}{2} = \frac{3x}{2} \in M$ by (b). But $v_2\left(\frac{3x}{2}\right) = t-1$, contradicting the minimality of $t$. Hence $t=0$, so $M$ contains an odd element.

    Let $r$ be the least odd element of $M$.

    \textbf{Step 3.} $r\mathbb{Z}_{>0} \subseteq M$.

    By Step 1, $M$ is closed under addition; hence all multiples $nr$ $(n \in \mathbb{Z}_{>0})$ lie in $M$.

    \textbf{Step 4.} Every element of $M$ is a multiple of $r$.

    Suppose, toward a contradiction, that there exists $y \in M$ not divisible by $r$. By Step 2 we may assume $y$ is odd (if $y$ is even, combine it with sufficiently large powers of 2 times $r$ using (b) to reduce its 2-adic valuation until an odd element is obtained; membership in $M$ is preserved at each step).

    Write $y = qr + a$ with $0 < a < r$ and $a$ even (since $y$ and $r$ are odd).

    Define a sequence $(y_0, y_1, y_2, \ldots)$ in $M$ by $y_0 := y$ and, for $n \geq 0$,
    \begin{align}
        y_{n+1} := \frac{y_n + r}{2} &\text{ if } y_n \text{ is odd}, \\
        y_{n+1} := \frac{y_n + 2r}{2} &\text{ if } y_n \text{ is even}.
    \end{align}
    Each step is valid by (b) (pairing $y_n$ with $r$ when $y_n$ is odd, and with $2r$ when $y_n$ is even).

    A simple induction shows that
    $$y_n \equiv 2^{-n} y \pmod{r},$$
    where $2^{-n}$ is the inverse of $2^n$ modulo $r$ (valid since $r$ is odd). In particular, none of the residues $y_n \pmod{r}$ is 0, because $y$ is not divisible by $r$. Moreover, for $n$ large enough, $y_n < 2r$ (since each step roughly halves the current term, up to an additive $O(r)$). Once $y_n < 2r$, its residue modulo $r$ equals either $y_n$ (if $y_n < r$) or $y_n - r$ (if $y_n > r$).

    Because the residue $y_n \pmod{r}$ never vanishes and takes only values in $\{1,2,\ldots,r-1\}$, eventually some $y_m$ is odd and less than $r$ (this uses that the orbit under division by 2 modulo $r$ contains both even and odd residues, and the sequence comes below $2r$). But then $y_m$ is an odd element of $M$ strictly smaller than $r$, contradicting the minimality of $r$.

    Therefore no such $y$ exists; every element of $M$ is divisible by $r$. Hence $M \subseteq r\mathbb{Z}_{>0}$.

    Combining with Step 3 we conclude $M = r\mathbb{Z}_{>0}$.

    \textbf{Step 5.} Verification.

    For any odd $r$, the set $M = r\mathbb{Z}_{>0}$ satisfies (a) and (b): doubling preserves multiples of $r$, and if $x = ra$, $y = rb$ have the same parity, then $\frac{x+y}{2} = r\frac{a+b}{2}$ is a multiple of $r$ because $a$ and $b$ have the same parity.

    Therefore, the complete list of solutions is exactly the sets of all positive multiples of an odd integer.

    \textbf{Final answer.}
    All such subsets are $M = r \mathbb{Z}_{>0}$ for some odd integer $r \geq 1$.
    \subsection{Variant 3}
    Let $v_2(n)$ be the exponent of 2 in $n$, and write $n = 2^{v_2(n)} \cdot n_{\text{odd}}$ with $n_{\text{odd}}$ odd.

    \textbf{Step 1.} Reduce to the ``odd gcd = 1'' case.

    Let $d := \gcd\{x_{\text{odd}} : x \in M\}$, i.e., the gcd of the odd parts of the elements of $M$. Then $d$ is odd, and every element of $M$ is divisible by $d$. Define
    $$S := \{x/d : x \in M\} \subseteq \mathbb{Z}_{>0}.$$
    Because $d$ is odd, $S$ also satisfies:
    \begin{itemize}
        \item[(a)] $s \in S \Rightarrow 2s \in S$,
        \item[(b)] $s,t \in S$ and $s+t$ even $\Rightarrow (s+t)/2 \in S$.
    \end{itemize}
    Moreover, the gcd of the odd parts of $S$ is 1.

    Thus it suffices to describe all such $S$; then $M = d \cdot S$.

    \textbf{Step 2.} $S$ contains an odd number.

    Let $m := \min S$. If $m$ is odd we are done. If $m$ is even, define $m_1 := (m+2m)/2 = 3m/2$. Since $m$ is even, $m_1 \in S$ and $v_2(m_1) = v_2(m) - 1$. Iterating, after finitely many steps we obtain an odd element $a \in S$.

    \textbf{Step 3.} From an odd $a \in S$, we can ``add'' odd parts.

    Let $a \in S$ be odd, and let $t \in S$ be arbitrary. Put $y_0 := 2t \in S$ (even) and define
    $$y_{k+1} := \frac{2a + y_k}{2} \quad \text{for } k \geq 0.$$
    Each $y_k \in S$, and a simple induction shows $y_k = a + y_0/2^k$. If we take $k = v_2(y_0) = v_2(t)+1$, then $y_k = a + \text{oddpart}(y_0) = a + \text{oddpart}(t)$. Hence:

    For every $t \in S$, we have $a + \text{oddpart}(t) \in S$.

    \textbf{Step 4.} $S$ contains all sufficiently large integers.

    Let $R$ be a finite subset of $S$ such that the gcd of $\{\text{oddpart}(r) : r \in R\}$ is 1 (possible since the gcd of odd parts in $S$ is 1). By Step 3, starting from $a$ we can add any $\text{oddpart}(r)$ with $r \in R$, and thus any nonnegative $\mathbb{Z}$-linear combination of those odd parts. Since the numerical semigroup generated by a finite set of positive integers with gcd 1 contains all sufficiently large integers, it follows that all sufficiently large odd integers belong to $S$. Then any sufficiently large even integer is the average of two consecutive large odd integers, so all sufficiently large integers lie in $S$. Therefore there exists $L_0$ with $[L_0, \infty) \subseteq S$.

    \textbf{Step 5.} $S$ is actually an upper ray from its minimum.

    Let $m = \min S$. Fix $n \geq m$. Choose $k$ large enough so that
    $$y := m + 2^k (n - m) \geq L_0.$$
    Then $y \in S$ (by Step 4), and $y \equiv m \pmod{2^k}$. Applying (b) $k$ times to the pair $(m, y)$ (which always has the same parity) gives
    $$m, y \Rightarrow (m+y)/2, \text{ then } (m + (m+y)/2)/2, \ldots, \text{ finally } m + (y-m)/2^k = n \in S.$$
    Hence every $n \geq m$ lies in $S$, i.e., $S = [m, \infty) \cap \mathbb{Z}$.

    \textbf{Step 6.} Scale back.

    Recall $M = d \cdot S$ with $d$ odd. Writing $m' := \min S$, we have
    $$M = \{d \cdot s : s \geq m'\} = \{n \in \mathbb{Z}_{>0} : n \geq d m' \text{ and } d \mid n\}.$$
    Thus $M$ is exactly the set of all multiples of an odd integer $d$, from some point onwards (with minimal element $dm'$).

    \textbf{Verification (sufficiency).}

    Given odd $d \geq 1$ and $r \geq 1$, define
    $$M = \{dn : n \geq r\}.$$
    Then:
    \begin{itemize}
        \item If $x = dn \in M$, $2x = d(2n) \in M$.
        \item If $x = da, y = db \in M$ and $x+y$ even, then $(x+y)/2 = d((a+b)/2)$ with $(a+b)/2 \geq r$; hence in $M$.
    \end{itemize}

    Therefore, the solutions are precisely these sets.

    \textbf{Final answer.}

    Exactly the sets
    $$M = \{n \in \mathbb{Z}_{>0} : n \equiv 0 \pmod{d} \text{ and } n \geq m\},$$
    where $d$ is an odd positive integer and $m$ is a positive multiple of $d$. Equivalently, for some odd $d$ and some $r \geq 1$,
    $$M = \{d \cdot n : n \geq r\}.$$
    \subsection{Final Solution}
    Let $v_2(n)$ be the exponent of 2 in $n$, and write $n = 2^{v_2(n)} \cdot \text{odd}(n)$ with $\text{odd}(n)$ odd.

    \textbf{Claim (closure under addition).} From (a) and (b), $M$ is closed under addition: if $a,b \in M$ then $2a,2b \in M$, hence $(2a+2b)/2 = a+b \in M$.

    \textbf{Claim (existence of an odd element).} Let $t$ be the minimum of $v_2(x)$ over $x \in M$, and pick $x$ with $v_2(x)=t$. If $t\geq 1$, then $x,2x \in M$ and $(x+2x)/2 = 3x/2 \in M$ has $v_2$ decreased by 1. Repeating, we reach an odd element of $M$. Thus $M$ contains an odd element.

    Now let $d$ be the gcd of the odd parts $\text{odd}(x)$ of elements $x \in M$. Then $d$ is odd, and $d \mid \text{odd}(x)$ for all $x \in M$, hence $d \mid x$ (because $d$ is odd). Thus $M \subseteq d\cdot\mathbb{Z}_{>0}$.

    Scale down by $d$: define $S = \{x/d : x \in M\} \subseteq \mathbb{Z}_{>0}$. Then $S$ is nonempty and satisfies (a) and (b), and the gcd of the odd parts in $S$ equals 1. We will show that $S$ is an upper ray: there exists $m$ such that $S = \{n \in \mathbb{Z}_{>0} : n \geq m\}$.

    1) $S$ contains an odd element (by the same $v_2$-reduction argument). Let $o$ be the least odd element of $S$. Because the gcd of odd parts in $S$ is 1, the gcd of all odd elements of $S$ is 1. Therefore, we can choose finitely many odd elements $a_1=o, a_2, \ldots, a_k \in S$ with $\gcd(a_1,\ldots,a_k)=1$.

    2) $S$ contains all sufficiently large integers. For each residue $r \in \{0,1,\ldots,a_1-1\}$, because the classes of $a_2,\ldots,a_k$ generate $\mathbb{Z}/a_1\mathbb{Z}$ (gcd is 1), there exist coefficients $t_2,\ldots,t_k$ with $0 \leq t_i < a_1$ such that
    \begin{align}
        t_2 a_2 + \cdots + t_k a_k \equiv r \pmod{a_1}.
    \end{align}
    Set $m_r = t_2 a_2 + \cdots + t_k a_k \in S$. Let $L = \max_r m_r$. Then for any $n \geq L$ with $n \equiv r \pmod{a_1}$,
    \begin{align}
        n = m_r + q a_1 \quad (q \geq 0),
    \end{align}
    and since $S$ is closed under addition and contains $a_1$, we have $n \in S$. Hence $[L, \infty) \subseteq S$.

    3) $S$ is actually the full upper ray from its minimum. Let $m = \min S$. Fix any $n \geq m$. Choose $k$ large enough so that $y := m + 2^k (n - m) \geq L$. Then $y \in S$ by step 2, and $y \equiv m \pmod{2^k}$. Applying (b) repeatedly to the pair $(m, y)$: set
    \begin{align}
        z_1 = \frac{m + y}{2}, \quad z_2 = \frac{m + z_1}{2}, \quad \ldots, \quad z_k = m + \frac{y - m}{2^k} = n.
    \end{align}
    At each step the two arguments have the same parity (since $y \equiv m \bmod 2^k$), so each averaging is allowed. Thus $n \in S$. Therefore $S = \{n \in \mathbb{Z}_{>0} : n \geq m\}$.

    Finally, scaling back, $M = d\cdot S = \{d n : n \geq m\}$. Writing $r = m$, we obtain
    \begin{align}
        M = \{n \in \mathbb{Z}_{>0} : n \equiv 0 \pmod{d} \text{ and } n \geq dr\},
    \end{align}
    with $d$ an odd integer and $r \geq 1$.

    \textbf{Verification (sufficiency).} For odd $d$ and $r \geq 1$, the set $M = \{d n : n \geq r\}$ satisfies:
    \begin{itemize}
        \item[(a)] If $x = d n \in M$ then $2x = d(2n) \in M$.
        \item[(b)] If $x = d a, y = d b \in M$ and $x + y$ is even, then $a$ and $b$ have the same parity, so $(x + y)/2 = d((a + b)/2)$ with $(a + b)/2 \geq r$, hence in $M$.
    \end{itemize}

    Therefore these are exactly the desired subsets.

    \textbf{Final answer.}
    The nonempty subsets $M \subseteq \mathbb{Z}_{>0}$ satisfying (a) and (b) are precisely
    \begin{align}
        M = \{d n : n \geq r\},
    \end{align}
    for some odd integer $d \geq 1$ and some integer $r \geq 1$.

    \section{Problem 8}
    \subsection{Variant 1}
    Let $A = (a_{ij}) \in M_n(\mathbb{R})$. The $90°$ counterclockwise rotation satisfies
    $(A^R)_{ij} = a_{j, n+1-i}$.

    Assume $A = A^R$. Then for all $i, j$,
    \begin{equation}
        a_{ij} = a_{j, n+1-i}.
    \end{equation}

    Applying the rotation twice gives the $180°$ rotation, whose effect on entries is
    $((A^R)^R)_{ij} = a_{n+1-i, n+1-j}$.
    Since $A = (A^R)^R$ as well, we also have
    \begin{equation}
        a_{ij} = a_{n+1-i, n+1-j}.
    \end{equation}

    We claim $A$ is symmetric, i.e. $a_{ij} = a_{ji}$. Using (2) with swapped indices gives
    $$a_{ij} = a_{n+1-j, n+1-i}.$$
    Applying (1) to the pair $(n+1-j, n+1-i)$ yields
    $$a_{n+1-j, n+1-i} = a_{n+1-i, j}.$$
    Thus $a_{ij} = a_{n+1-i, j}$. Applying (1) again to the pair $(n+1-i, j)$ gives
    $$a_{n+1-i, j} = a_{j, i}.$$
    Therefore $a_{ij} = a_{ji}$, proving $A$ is symmetric.

    Since $A$ is a real symmetric matrix, it is Hermitian, and hence all its eigenvalues are real. In particular, for any eigenvalue $\lambda$ of $A$, $\text{Im } \lambda = 0$ (and thus $\text{Re } \lambda = 0$ or $\text{Im } \lambda = 0$ holds).

    \textbf{Final answer:} All eigenvalues of $A$ are real; in particular, $\text{Im } \lambda = 0$ for every eigenvalue $\lambda$.
    \subsection{Variant 2}
    Let $J$ be the $n \times n$ reversal (anti-identity) matrix, $J = [\delta_{i,n+1-j}]$. For any $X \in M_n(\mathbb{R})$ one checks directly
    $(X^R)_{ij} = x_{j,n+1-i}$, hence $X^R = J X^T$.

    Thus $A = A^R$ is equivalent to
    \begin{equation}
        A = J A^T. \tag{1}
    \end{equation}

    From (1), taking transpose gives $A^T = A J$, and multiplying by $J$ on the left yields
    $J A^T = J A J$. Using (1) again, $J A^T = A$, so
    \begin{equation}
        J A J = A, \text{ i.e. } A \text{ commutes with } J. \tag{2}
    \end{equation}

    Since $J$ is real symmetric with eigenvalues $\pm 1$, there exists an orthogonal $Q$ such that
    $Q^T J Q = D := \text{diag}(I_r, -I_s)$, with $r + s = n$.
    Let $M := Q^T A Q$. Then (2) implies $D M D = M$, so $M$ commutes with $D$. Writing $M$ in $2 \times 2$ block form conformal with $D$,
    \[
        M = \begin{bmatrix} M_{11} & M_{12} \\ M_{21} & M_{22} \end{bmatrix},
    \]
    the commutation $D M = M D$ forces $M_{12} = 0$ and $M_{21} = 0$, i.e.
    $M = \text{diag}(B, C)$ for some real $B$ $(r \times r)$ and $C$ $(s \times s)$.

    Now transform (1) into this basis:
    $M = Q^T A Q = Q^T J A^T Q = (Q^T J Q) (Q^T A Q)^T = D M^T$.
    With $M = \text{diag}(B, C)$ and $D = \text{diag}(I_r, -I_s)$, this becomes
    $\text{diag}(B, C) = \text{diag}(B^T, -C^T)$,
    so
    $B = B^T$ (symmetric) and $C = -C^T$ (skew-symmetric).

    Therefore $A$ is orthogonally similar to a block-diagonal matrix with one real symmetric block and one real skew-symmetric block. The eigenvalues of a real symmetric matrix are real, while the eigenvalues of a real skew-symmetric matrix are purely imaginary (including possibly 0). Hence every eigenvalue $\lambda$ of $A$ is either real or purely imaginary.

    \textbf{Final answer.}
    For any $A$ with $A = A^R$, its spectrum is the disjoint union of the spectra of a real symmetric and a real skew-symmetric matrix; thus each eigenvalue $\lambda$ satisfies either $\text{Re } \lambda = 0$ or $\text{Im } \lambda = 0$.
    \subsection{Variant 3}
    Let $J$ denote the $n \times n$ reversal matrix (ones on the anti-diagonal, zeros elsewhere). Then for any $A$ we have
    $$A^R = J A^T.$$
    Indeed, $(J A^T)_{i,j} = A_{j,n+1-i}$, which is exactly the $90°$ counterclockwise rotation of entries.

    Hence the condition $A = A^R$ is equivalent to
    $$A = J A^T.$$

    From this,
    \begin{itemize}
        \item multiplying on the left by $J$ gives $J A = A^T$,
        \item taking transpose gives $A^T = A J$.
    \end{itemize}

    Combining, we get $J A = A J$; thus $A$ commutes with $J$.

    Since $J$ is real symmetric with $J^2 = I$, the spectral theorem yields an orthogonal matrix $Q$ such that
    $$Q^T J Q = \text{diag}(I_p, -I_q) =: D,$$
    where $p = \lceil n/2 \rceil$ and $q = \lfloor n/2 \rfloor$. Set
    $$S := Q^T A Q.$$

    Because $A$ commutes with $J$, $S$ commutes with $D$:
    $$D S = S D.$$
    Writing $S$ in $p \times p$, $p \times q$, $q \times p$, $q \times q$ blocks as $S = \begin{bmatrix} B & F \\ G & C \end{bmatrix}$, the commutation $D S = S D$ reads
    $$\begin{bmatrix} B & F \\ -G & -C \end{bmatrix} = \begin{bmatrix} B & -F \\ G & -C \end{bmatrix},$$
    so $F = 0$ and $G = 0$. Therefore $S$ is block diagonal:
    $$S = \begin{bmatrix} B & 0 \\ 0 & C \end{bmatrix}.$$

    Now translate $A = J A^T$ into this basis:
    $$S = Q^T A Q = Q^T J A^T Q = (Q^T J Q)(Q^T A Q)^T = D S^T.$$
    Thus
    $$\begin{bmatrix} B & 0 \\ 0 & C \end{bmatrix} = \begin{bmatrix} B^T & 0 \\ 0 & -C^T \end{bmatrix},$$
    so $B$ is symmetric ($B = B^T$) and $C$ is skew-symmetric ($C^T = -C$).

    Consequently, all eigenvalues of $B$ are real ($B$ real symmetric), and all eigenvalues of $C$ are purely imaginary ($C$ real skew-symmetric). Hence all eigenvalues of $S$, and thus of $A$ (being orthogonally similar to $S$), are either real or purely imaginary.

    \textbf{Final answer:} Every eigenvalue $\lambda$ of $A$ is real or purely imaginary; i.e., $\text{Re } \lambda = 0$ or $\text{Im } \lambda = 0$.
    \subsection{Final Solution}
    Let $J$ be the $n \times n$ reversal (anti-identity) matrix, with ones on the anti-diagonal and zeros elsewhere. A direct entrywise check shows
    $$(A^R)_{ij} = a_{j,\,n+1-i} = (J A^T)_{ij},$$
    so $A^R = J A^T$. Hence the condition $A = A^R$ is equivalent to
    $$A = J A^T. \quad {(1)}$$

    Transposing (1) gives $A^T = A J$ (since $J^T = J$). Left-multiplying (1) by $J$ and using $J^2 = I$ yields
    $$J A^T = A \text{ and thus } A = J A J,$$
    so $A$ commutes with $J$:
    $$A J = J A. \quad {(2)}$$

    Because $J$ is real symmetric with $J^2 = I$, there exists an orthogonal $Q$ such that
    $$Q^T J Q = D := \text{diag}(I_r, -I_s), \text{ with } r + s = n.$$
    Set $M := Q^T A Q$. From (2) we get $D M = M D$, so in a block form compatible with $D$ we have
    $$M = \begin{bmatrix} B & 0 \\ 0 & C \end{bmatrix}.$$
    Now rewrite (1) in this basis:
    $$M = Q^T A Q = Q^T J A^T Q = (Q^T J Q)(Q^T A Q)^T = D M^T.$$
    Therefore
    $$\text{diag}(B, C) = \text{diag}(B^T, -C^T),$$
    so $B = B^T$ (symmetric) and $C = -C^T$ (skew-symmetric).

    Hence $A$ is orthogonally similar to a block-diagonal matrix with one real symmetric block and one real skew-symmetric block. The eigenvalues of a real symmetric matrix are real, and those of a real skew-symmetric matrix are purely imaginary (possibly 0). Therefore every eigenvalue $\lambda$ of $A$ is either real or purely imaginary.

    \textbf{Final answer.}
    For any $A$ with $A = A^R$, each eigenvalue $\lambda$ satisfies $\text{Re } \lambda = 0$ or $\text{Im } \lambda = 0$.
    \section{Problem 9}
    \subsection{Variant 1}
    Let $R_j$ be the rank of $X_j$ among the remaining positive integers at step $j$ (i.e., $R_j=1$ means we pick the smallest remaining number, $R_j=2$ the second smallest, etc.). By the rule of the process, for every step $j$ and every $i \geq 1$ we have
    $$P(R_j=i) = 2^{-i},$$
    independently of the past. Hence the $R_j$ are i.i.d. with this distribution.

    \textbf{1) Distribution of $Y_n$}

    For $m \geq 0$, the event $Y_n \leq m$ means that all $n$ chosen numbers lie in $\{1, \ldots, m\}$. Equivalently, at step $j$ ($1 \leq j \leq n$) the rank $R_j$ must be at most the number of remaining elements in $\{1, \ldots, m\}$, which equals $m-(j-1)$. Thus, for $m \geq n-1$,
    $$P(Y_n \leq m) = \prod_{j=1}^n P(R_j \leq m-j+1) = \prod_{j=1}^n (1-2^{-(m-j+1)}).$$
    For $m \leq n-1$, this probability is 0, as one cannot fit $n$ distinct numbers into a set of size $m$.

    \textbf{2) Tail-sum for the expectation}

    Using $E[Y_n] = \sum_{m \geq 1} P(Y_n \geq m)$, and noting $P(Y_n \geq m) = 1$ for $m \leq n$, we get
    $$E[Y_n] = n + \sum_{m \geq n+1} \left[1 - \prod_{j=1}^n (1-2^{-(m-j)})\right].$$
    Reindex with $t = m-n \geq 1$ and set $q = 1/2$ to write
    $$E[Y_n] = n + \sum_{t \geq 1} \left[1 - \prod_{k=0}^{n-1} (1-q^{t+k})\right].$$

    \textbf{3) A $q$-identity}

    Define for $0 < q < 1$
    $$S_n(q) := \sum_{t \geq 1} \left[1 - \prod_{k=0}^{n-1} (1-q^{t+k})\right].$$
    We claim $S_n(q) = \sum_{i=1}^n \frac{q^i}{1-q^i}$. This can be proved by induction on $n$ by showing
    $$S_n(q) - S_{n-1}(q) = \frac{q^n}{1-q^n}.$$
    Indeed,
    \begin{align}
        S_n - S_{n-1} &= \sum_{t \geq 1} \prod_{k=0}^{n-2}(1-q^{t+k}) - \prod_{k=0}^{n-1}(1-q^{t+k})\\
        &= \sum_{t \geq 1} q^{t+n-1} \prod_{k=0}^{n-2}(1-q^{t+k}).
    \end{align}

    Let $A_t := \prod_{k=0}^{n-1}(1-q^{t+k})$. A simple difference identity gives
    $$A_t - A_{t-1} = q^{t-1}(1-q^n) \prod_{k=0}^{n-2}(1-q^{t+k}),$$
    hence
    $$q^t \prod_{k=0}^{n-2}(1-q^{t+k}) = \frac{q[A_t - A_{t-1}]}{1-q^n}.$$
    Summing over $t \geq 1$ and using telescoping ($A_0 = 0$, $\lim_{t \to \infty} A_t = 1$) yields
    $$\sum_{t \geq 1} q^t \prod_{k=0}^{n-2}(1-q^{t+k}) = \frac{q}{1-q^n},$$
    and therefore $S_n - S_{n-1} = \frac{q^n}{1-q^n}$. Since $S_1(q) = \frac{q}{1-q}$, the induction gives
    $$S_n(q) = \sum_{i=1}^n \frac{q^i}{1-q^i}.$$

    \textbf{4) Conclusion}

    With $q = 1/2$ we obtain
    $$E[Y_n] = n + \sum_{i=1}^n \frac{2^{-i}}{1-2^{-i}} = \sum_{i=1}^n \frac{2^i}{2^i-1}.$$

    \textbf{Final answer.}
    $$E[Y_n] = \sum_{i=1}^n \frac{2^i}{2^i-1}.$$
    \subsection{Variant 2}
    Let $Z_1 < Z_2 < \ldots < Z_n$ be the selected numbers in increasing order (the order statistics of $X_1, \ldots, X_n$). Set $Z_0 := 0$ and define the gaps $G_j := Z_j - Z_{j-1}$ for $j = 1, \ldots, n$. Then $Y_n = Z_n = G_1 + \cdots + G_n$.

    \textbf{Key observation (block-avoidance probability).}
    Fix $s \geq 1$. At any time, if none of the first $s$ remaining integers has been chosen yet, then the probability that the next pick avoids this block equals
    $$\sum_{i \geq s+1} 2^{-i} = 2^{-s}.$$
    Moreover, as long as the block remains untouched, this avoidance probability stays $2^{-s}$ at every step. Hence, for $r$ upcoming picks, the probability that all $r$ picks avoid this block is $(2^{-s})^r$.

    \textbf{Distribution of the first gap $G_1$.}
    For $k \geq 1$, the event $\{G_1 \geq k\}$ means that none of the first $k-1$ positive integers is chosen among the $n$ selections. By the observation with $s = k-1$ and $r = n$, we get
    $$P(G_1 \geq k) = (2^{-(k-1)})^n = 2^{-n(k-1)}.$$
    Thus $G_1$ is geometric (on $\{1, 2, \ldots\}$) with parameter $p = 1 - 2^{-n}$, i.e.
    $$P(G_1 = k) = (1 - 2^{-n}) 2^{-n(k-1)}, \quad \text{and} \quad E[G_1] = \frac{1}{p} = \frac{2^n}{2^n - 1}.$$

    \textbf{Distribution of general gaps $G_j$.}
    Condition on $Z_1, \ldots, Z_{j-1}$. There remain $r = n - j + 1$ selections to be made from the remaining integers, and consider the block of the next $s$ integers after $Z_{j-1}$, i.e., $\{Z_{j-1}+1, \ldots, Z_{j-1}+s\}$. The event $\{G_j \geq s+1\}$ is exactly that none of these $r$ remaining picks hits this block. By the same observation,
    $$P(G_j \geq s+1 \mid Z_1, \ldots, Z_{j-1}) = (2^{-s})^r = 2^{-s(n-j+1)}.$$
    Therefore, unconditionally,
    $$P(G_j \geq t) = 2^{-(t-1)(n-j+1)} \quad \text{for } t = 1, 2, \ldots,$$
    so $G_j$ is geometric with parameter $p_j = 1 - 2^{-(n-j+1)}$ and
    $$E[G_j] = \frac{1}{p_j} = \frac{2^{n-j+1}}{2^{n-j+1} - 1}.$$

    Taking expectations and summing:
    \begin{align}
        E[Y_n] &= E[Z_n] = \sum_{j=1}^{n} E[G_j]\\
        &= \sum_{j=1}^{n} \frac{2^{n-j+1}}{2^{n-j+1} - 1}\\
        &= \sum_{i=1}^{n} \frac{2^i}{2^i - 1},
    \end{align}
    as claimed.

    \textbf{Final answer.}
    $$E[Y_n] = \sum_{i=1}^{n} \frac{2^i}{2^i - 1}.$$
    \subsection{Variant 3}
    Let us realize the selection rule as follows: at each step, list the remaining integers in increasing order and, independently, flip a fair coin for each (in that order) until the first Head appears; choose that integer. This gives exactly
    \[P(\text{``choose the $i$-th smallest remaining''}) = \left(\frac{1}{2}\right)^i,\]
    as required.

    \textbf{1) Tail probabilities of the maximum.}

    For $k \geq 1$, the event $\{Y_n < k\}$ means that in each of the $n$ stages we select from the remaining numbers less than $k$. If before stage $j$ we have selected only numbers $< k$, then there remain exactly $k-j$ numbers less than $k$, which occupy the first $k-j$ positions in the ordered remaining list. Thus the probability that at stage $j$ we again choose a number $< k$ is
    \[\sum_{i=1}^{k-j} 2^{-i} = 1 - 2^{-(k-j)}.\]

    Therefore, for $k \geq n+1$,
    \[P(Y_n < k) = \prod_{j=1}^{n} (1 - 2^{-(k-j)}) = \prod_{r=k-n}^{k-1} (1 - 2^{-r}),\]
    and for $k \leq n$ we have $P(Y_n < k) = 0$.

    Hence
    \[E[Y_n] = \sum_{k=1}^{\infty} P(Y_n \geq k) = \sum_{k=1}^{n} 1 + \sum_{k=n+1}^{\infty} \left(1 - \prod_{r=k-n}^{k-1} (1 - 2^{-r})\right).\]

    \textbf{2) Increment of the expectation.}

    Consider the difference
    \[E[Y_n] - E[Y_{n-1}] = \sum_{k=1}^{\infty} (P(Y_n \geq k) - P(Y_{n-1} \geq k)) = \sum_{k=1}^{\infty} (P(Y_{n-1} < k) - P(Y_n < k)).\]

    For $k \geq n$,
    \begin{align}
        P(Y_{n-1} < k) &= \prod_{r=k-n+1}^{k-1} (1 - 2^{-r}), \text{ and}\\
        P(Y_n < k) &= (1 - 2^{-(k-n)}) \prod_{r=k-n+1}^{k-1} (1 - 2^{-r}),
    \end{align}
    so
    \[P(Y_{n-1} < k) - P(Y_n < k) = 2^{-(k-n)} \prod_{r=k-n+1}^{k-1} (1 - 2^{-r}).\]

    Let $m = k - n \geq 0$. Then
    \[E[Y_n] - E[Y_{n-1}] = \sum_{m=0}^{\infty} 2^{-m} \prod_{r=m+1}^{m+n-1} (1 - 2^{-r}).\]

    Define $Q_m = \prod_{r=m+1}^{m+n} (1 - 2^{-r})$ (with $Q_{-1} = 0$). A direct computation gives the telescoping identity
    \[Q_m - Q_{m-1} = (1 - 2^{-n}) 2^{-m} \prod_{r=m+1}^{m+n-1} (1 - 2^{-r}).\]

    Therefore
    \begin{align}
        E[Y_n] - E[Y_{n-1}] &= \frac{1}{1 - 2^{-n}} \sum_{m=0}^{\infty} (Q_m - Q_{m-1})\\
        &= \frac{1}{1 - 2^{-n}} \left(\lim_{M\to\infty} Q_M - Q_{-1}\right).
    \end{align}

    Since $Q_M = \prod_{r=M+1}^{M+n} (1 - 2^{-r}) \to 1$ as $M \to \infty$, we conclude
    \[E[Y_n] - E[Y_{n-1}] = \frac{1}{1 - 2^{-n}} = \frac{2^n}{2^n - 1}.\]

    Finally, with $E[Y_0] = 0$, summing these increments yields
    \[E[Y_n] = \sum_{i=1}^{n} (E[Y_i] - E[Y_{i-1}]) = \sum_{i=1}^{n} \frac{2^i}{2^i - 1}.\]

    \textbf{Final answer.}
    \[E[Y_n] = \sum_{i=1}^{n} \frac{2^i}{2^i - 1}.\]

    \subsection{Final Solution}
    \textbf{Solution.}

    Realize the selection rule by ranks: at each step, list the remaining integers in increasing order and pick the $i$-th smallest remaining with probability $2^{-i}$. Equivalently, one can flip a fair coin for each remaining integer in order until the first Head appears; the chosen integer is exactly the $i$-th remaining when the first Head occurs at flip $i$. Thus, at each step and independently of the past, the event ``we avoid the first $s$ remaining integers'' has probability
    $$\sum_{i \geq s+1} 2^{-i} = 2^{-s}.$$

    Let $Z_1 < Z_2 < \ldots < Z_n$ be the selected numbers in increasing order, set $Z_0=0$, and define the gaps $G_j := Z_j - Z_{j-1}$ ($j=1,\ldots,n$). Then $Y_n = Z_n = G_1 + \ldots + G_n$.

    Fix $j$ and condition on $Z_1,\ldots,Z_{j-1}$. There remain $r = n - j + 1$ selections. The event $\{G_j \geq t\}$ means that none of the $r$ remaining picks hits the block of the first $t-1$ integers after $Z_{j-1}$. As long as this block is untouched, at each of those $r$ steps the probability to avoid it equals $2^{-(t-1)}$, and these avoidance events are independent across steps. Hence
    $$P(G_j \geq t \mid Z_1,\ldots,Z_{j-1}) = (2^{-(t-1)})^r = 2^{-(t-1)(n-j+1)}.$$
    This does not depend on the conditioning, so unconditionally
    $$P(G_j \geq t) = 2^{-(t-1)(n-j+1)}, \quad t=1,2,\ldots,$$
    i.e. $G_j$ is geometric on $\{1,2,\ldots\}$ with parameter $p_j = 1 - 2^{-(n-j+1)}$. Therefore
    $$E[G_j] = \frac{1}{p_j} = \frac{2^{n-j+1}}{2^{n-j+1} - 1}.$$

    Summing expectations,
    \begin{align}
        E[Y_n] &= E[Z_n] = \sum_{j=1}^{n} E[G_j]\\
        &= \sum_{j=1}^{n} \frac{2^{n-j+1}}{2^{n-j+1} - 1}\\
        &= \sum_{i=1}^{n} \frac{2^{i}}{2^{i} - 1}.
    \end{align}

    \textbf{Final answer.}
    $$E[Y_n] = \sum_{i=1}^{n} \frac{2^{i}}{2^{i} - 1}.$$

    \section{Problem 10}
    \subsection{Variant 1}
    Let $t(n) = n(n+1)$. For a prime $p$, denote by $v_p(n)$ the $p$-adic valuation. Because $n$ and $n+1$ are coprime, $v_p(t(n)) = v_p(n) + v_p(n+1)$ equals the valuation of whichever of $n,n+1$ is divisible by $p$. Let $E_p(n)$ be the parity $\pmod{2}$ of $v_p(t(n))$. Then for fixed $p$, $E_p(n) = 1$ if and only if $v_p$ of the divisible one is odd.

    A local computation (using the geometric distribution of valuations) gives, for every prime $p$,
    $$P(E_p(n) = 1) = 2 \sum_{r \text{ odd} \geq 1} \left(\frac{1}{p^r} - \frac{1}{p^{r+1}}\right) = \frac{2}{p+1}.$$
    Thus $P(E_p(n) = 0) = \frac{p-1}{p+1}$.

    For two independent integers $a,b$ (uniformly in $[1,N]$), the condition that $t(a)t(b)$ is a perfect square is equivalent to $E_p(a) = E_p(b)$ for every prime $p$, and for large $N$ it suffices to check primes $p \leq N+1$ (since no $p > N+1$ divides $t(a)$ or $t(b)$). For a fixed $p$, the probability of a match is
    \begin{align}
        M_p &:= P(E_p(a) = E_p(b)) = P(1,1) + P(0,0) = \left[\frac{2}{p+1}\right]^2 + \left[\frac{p-1}{p+1}\right]^2\\
        &= 1 - \frac{4(p-1)}{(p+1)^2} = 1 - \frac{4}{p} + O\left(\frac{1}{p^2}\right).
    \end{align}

    We will extract a lower bound for the proportion of matching pairs using a truncation at a parameter $y$ with $2 \leq y \leq N$. Define the set
    $$A_y := \left\{n \leq N : E_p(n) = 0 \text{ for all primes } p \text{ with } y < p \leq N+1\right\}.$$
    By independence of local conditions across distinct primes (via the Chinese Remainder Theorem and standard density arguments), we have
    \begin{align}
        |A_y| &= N \prod_{y < p \leq N+1} P(E_p(n) = 0) + o(N)\\
        &= N \prod_{y < p \leq N+1} \left(1 - \frac{2}{p+1}\right) + o(N).
    \end{align}

    Moreover, among pairs $(a,b) \in A_y \times A_y$, the condition $t(a)t(b)$ is a square is already guaranteed at all primes $p > y$ (by the definition of $A_y$), and for primes $p \leq y$ the matching probability is $\prod_{p \leq y} M_p$. Therefore, the number of such matching pairs satisfies
    $$S_N \geq |A_y|^2 \prod_{p \leq y} M_p + o(N^2).$$

    We now estimate these products using Mertens' theorem for primes. Recall that
    $$\sum_{p \leq x} \frac{1}{p} = \log \log x + B + o(1),$$
    and hence for fixed $\alpha$ one has
    $$\prod_{p \leq x} \left(1 - \frac{\alpha}{p}\right) = \frac{C(\alpha)}{(\log x)^\alpha} \cdot (1 + o(1)),$$
    for some positive constant $C(\alpha)$. Since
    $$1 - \frac{2}{p+1} = 1 - \frac{2}{p} + O\left(\frac{1}{p^2}\right), \quad \text{and} \quad M_p = 1 - \frac{4}{p} + O\left(\frac{1}{p^2}\right),$$
    we obtain
    \begin{align}
        \prod_{y < p \leq N+1} \left(1 - \frac{2}{p+1}\right) &= C_1 \cdot \left(\frac{\log y}{\log N}\right)^2 \cdot (1 + o(1)),\\
        \prod_{p \leq y} M_p &= C_0 \cdot \frac{1}{(\log y)^4} \cdot (1 + o(1)),
    \end{align}
    for positive constants $C_0, C_1$.

    Plugging these into the lower bound gives
    \begin{align}
        S_N &\geq \left[N \cdot C_1 \left(\frac{\log y}{\log N}\right)^2\right]^2 \cdot \left[\frac{C_0}{(\log y)^4}\right] \cdot (1 + o(1))\\
        &= (C_0 C_1^2) \cdot \frac{N^2}{(\log N)^4} \cdot (1 + o(1)).
    \end{align}
    Since $y$ can be any function with $2 \leq y \leq N$ tending to infinity (e.g. $y = \lfloor\sqrt{N}\rfloor$), the constant $C_0 C_1^2$ is positive and independent of $N$, so for all large $N$,
    $$S_N \geq c \cdot \frac{N^2}{(\log N)^4},$$
    for some absolute $c > 0$. Consequently,
    $$\frac{S_N}{N} \geq c \cdot \frac{N}{(\log N)^4} \to \infty \quad \text{as } N \to \infty.$$

    In particular, the requested limit exists and is infinite.

    Final answer: $+\infty$
    \subsection{Variant 2}
    Let $s(n)$ denote the squarefree kernel of $n$ (the product of primes that appear to odd exponent in $n$). For $1 \leq a \leq N$, set
    $$r(a) = s(a(a+1)).$$

    Then $(a^2+a)(b^2+b)$ is a perfect square if and only if $s(a(a+1)) = s(b(b+1))$. Hence, if
    $$A_r(N) = \#\{1 \leq a \leq N : r(a) = r\},$$
    we have
    $$S_N = \sum_r A_r(N)^2 = N + 2\sum_r \binom{A_r(N)}{2}.$$

    Therefore it suffices to estimate
    $$P_N := \sum_r \binom{A_r(N)}{2},$$
    the number of unordered pairs $\{a,b\}$ with $1 \leq a < b \leq N$ and $r(a) = r(b)$. We will prove $P_N = o(N)$, which implies $S_N = N + o(N)$ and hence $S_N/N \to 1$.

    \textbf{1) Pell-type parameterization}

    Fix a squarefree integer $r$. The condition $r(a) = r$ is equivalent to
    $$a(a+1) = r t^2$$
    for some integer $t \geq 1$. Writing $u = 2a+1$, this becomes
    $$u^2 - 4r t^2 = 1, \quad \text{with } u \text{ odd}, \quad a = \frac{u-1}{2}.$$

    Thus for each fixed $r$, the $u$ that arise are the $u$-coordinates of the solutions to the Pell equation
    $$u^2 - D t^2 = 1, \quad \text{with } D = 4r \geq 8$$
    (since $r \geq 2$; note $a(a+1)$ cannot be a square as $a$ and $a+1$ are coprime).

    Let $u_1$ be the least $u > 1$ (necessarily odd) for which $u^2 - D t^2 = 1$ has an integer solution. All solutions are given by
    $$u_m + t_m\sqrt{D} = (u_1 + t_1\sqrt{D})^m, \quad m = 1,2,3,\ldots$$

    In particular, the sequence $(u_m)$ satisfies the recurrence $u_{m+1} = 2u_1 u_m - u_{m-1}$, and one has the identity
    $$u_m = T_m(u_1),$$
    where $T_m$ is the Chebyshev polynomial of the first kind. A simple induction using $T_{m+1}(x) = 2x T_m(x) - T_{m-1}(x)$ shows that for $x \geq 1$,
    $$T_m(x) \geq x^m \quad \text{for all } m \geq 1.$$

    Therefore, for each $r$, if $A_r(N) \geq m$ (i.e., there are at least $m$ values of $a \leq N$ in this class), then
    $$u_m \leq 2N + 1 \Rightarrow u_1^m \leq u_m \leq 2N+1 \Rightarrow u_1 \leq (2N+1)^{1/m}.$$

    \textbf{2) Counting $r$ with many solutions}

    Let $T_m(N)$ be the number of squarefree $r$ for which $A_r(N) \geq m$. From the conclusion above, for each such $r$ there exists an odd $u_1 \leq (2N+1)^{1/m}$ and integers $t_1 \geq 1$ such that
    $$4r t_1^2 = u_1^2 - 1.$$

    Thus, for a fixed odd $u$, the admissible $r$ are the squarefree parts of $(u^2 - 1)/4$. The number of such $r$ is at most the number of squarefree divisors of $(u^2 - 1)/4$, which is $\leq \tau((u^2 - 1)/4)$, where $\tau$ is the divisor function. Consequently,
    $$T_m(N) \leq \sum_{\substack{\text{odd } u \leq (2N+1)^{1/m}}} \tau\left(\frac{u^2 - 1}{4}\right).$$

    Using the well-known bound $\tau(n) \leq n^\varepsilon$ for any $\varepsilon > 0$ and all sufficiently large $n$, and noting that $(u^2 - 1)/4 \leq u^2$, we get, for any fixed $\varepsilon > 0$,
    $$T_m(N) \ll \sum_{u \leq c N^{1/m}} u^{2\varepsilon} \ll N^{1/m + 2\varepsilon}.$$

    Since $\varepsilon > 0$ is arbitrary, we can write succinctly
    $$T_m(N) \leq N^{1/m + o(1)}.$$

    \textbf{3) Bounding $P_N$}

    Write
    $$P_N = \sum_r \binom{A_r(N)}{2} = \sum_{m \geq 2} (m - 1) T_m(N).$$
    (The identity follows by counting, for each class with size $A$, the number of pairs as $\binom{A}{2} = \sum_{m=2}^A (m - 1)$.)

    Using the estimate for $T_m(N)$,
    $$P_N \leq \sum_{m \geq 2} (m - 1) N^{1/m + o(1)} = N^{1/2 + o(1)} + N^{1/3 + o(1)} + N^{1/4 + o(1)} + \ldots = N^{1/2 + o(1)}.$$

    Therefore $P_N = o(N)$, and hence
    $$S_N = N + 2P_N = N + o(N).$$

    It follows that the limit exists and equals 1:
    $$\lim_{N\to\infty} \frac{S_N}{N} = 1.$$

    \textbf{Final answer:} 1.
    \subsection{Variant 3}
    Let $r(n)$ denote the squarefree kernel of $n(n+1)$. Since $\gcd(n,n+1)=1$, $r(n)$ is the product of the primes that divide $n$ or $n+1$ to an odd exponent. Then the condition that $(a^2+a)(b^2+b)$ is a perfect square is equivalent to $r(a)=r(b)$. Hence, if we define
    \begin{itemize}
        \item $c_k(N) := \#\{1 \leq a \leq N : r(a)=k\}$ for squarefree $k$,
    \end{itemize}
    then
    $$S_N = \sum_k c_k(N)^2.$$

    In particular, $S_N \geq \sum_k c_k(N) = N$ (the diagonal pairs $a=b$). We will show that the off-diagonal contribution $S_N - N$ is $o(N)$, which will imply
    $$\lim_{N\to\infty} S_N/N = 1.$$

    \textbf{1) Reduction to Pell equations.} For a given squarefree $k$, the condition $r(a)=k$ is equivalent to
    $$a(a+1) = k t^2$$
    for some integer $t$. Multiplying by $4$ and setting $X=2a+1$, $Y=2t$, we obtain
    $$X^2 - k Y^2 = 1,$$
    with $X$ odd. Conversely, any solution $(X,Y)$ of $X^2 - k Y^2 = 1$ has $X$ odd and $Y$ even (for $k$ odd this is forced mod $2$; for $k$ even it holds mod $8$), so $a=(X-1)/2$ is an integer with $a(a+1)=k (Y/2)^2$. Thus
    $$c_k(N) = \#\{\text{solutions } (X,Y) \text{ of } X^2 - k Y^2 = 1 \text{ with } 1 \leq X \leq 2N+1\}.$$

    Let $\varepsilon_k = x_1 + y_1\sqrt{k} > 1$ be the fundamental unit of the Pell equation $X^2 - k Y^2 = 1$. All solutions are given by $x_n + y_n\sqrt{k} = \varepsilon_k^n$ $(n\geq 0)$. In particular, $x_n$ is increasing with $n$ and $x_n \asymp \varepsilon_k^n$. Hence
    $$c_k(N) \geq t \implies \varepsilon_k^t \lesssim 2N+1,$$
    more precisely
    $$c_k(N) \geq t \Rightarrow \varepsilon_k^t \leq 4N+1 \Rightarrow \varepsilon_k \leq (4N+1)^{1/t}. \quad (1)$$

    \textbf{2) Bounding the number of $k$ with small $\varepsilon_k$.} We use a simple counting lemma.

    \textbf{Lemma.} Let $A(X)$ be the number of squarefree $k$ for which the Pell equation $X^2 - k Y^2 = 1$ has a solution with $X \leq X$. Then $A(X) \ll X$.

    \textbf{Proof.} For each such $k$, choose one solution $(x,y)$ with $1 \leq x \leq X$. Then $k = (x^2 - 1)/y^2$ is integer, so $y^2 | x^2 - 1$. For fixed $y$, the congruence $y^2 | x^2 - 1$ means $x \equiv \pm 1 \pmod{y^2}$, giving at most $2\lceil X/y^2 \rceil$ choices of $x$ in $[1,X]$. Summing over $1 \leq y \leq X$,
    $$A(X) \leq \sum_{y=1}^X 2\lceil X/y^2 \rceil \ll X \sum_{y=1}^\infty 1/y^2 + \sum_{y=1}^X 1 \ll X.$$
    This proves the lemma.

    Combining $(1)$ with the lemma, for $t \geq 1$,
    $$A_t(N) := \#\{k : c_k(N) \geq t\} \leq \#\{k : \varepsilon_k \leq (4N+1)^{1/t}\} \ll N^{1/t}. \quad (2)$$

    \textbf{3) Bounding the off-diagonal.} Using the identity $m^2 = \sum_{t=1}^m (2t-1)$, we have
    $$S_N = \sum_k c_k(N)^2 = \sum_{t=1}^\infty (2t-1) A_t(N).$$
    Since $A_1(N) = \#\{k : c_k(N) \geq 1\} \leq N$, and using $(2)$ for $t \geq 2$,
    $$S_N \leq N + C \sum_{t=2}^\infty (2t-1) N^{1/t}$$
    for some absolute constant $C$. The series is dominated by its first term $t=2$, and one easily checks
    $$\sum_{t=2}^\infty (2t-1) N^{1/t} \ll N^{1/2}.$$
    Thus
    $$N \leq S_N \leq N + O(N^{1/2}),$$
    so $S_N = N + o(N)$, and consequently
    $$\lim_{N\to\infty} S_N/N = 1.$$

    \textbf{Final answer:} $1$

    \subsection{Final Solution}
    Let $r(n)$ denote the squarefree kernel of $n$, i.e., the product of all primes dividing $n$ to an odd power. Since $\gcd(a, a+1)=1$, the condition that $(a^2+a)(b^2+b)$ is a perfect square is equivalent to
    $$r(a(a+1)) = r(b(b+1)).$$

    Define, for each squarefree $r$,
    $$A_r(N) = \#\{1 \leq a \leq N : r(a(a+1)) = r\}.$$

    Then
    $$S_N = \sum_r A_r(N)^2 = \sum_r A_r(N) + 2\sum_r \binom{A_r(N)}{2} = N + 2 P_N,$$
    where $P_N$ counts unordered pairs $\{a,b\}$ with $1 \leq a < b \leq N$ and $r(a(a+1)) = r(b(b+1))$. We will prove $P_N = o(N)$, hence $S_N = N + o(N)$ and $S_N/N \to 1$.

    \textbf{1) Reduction to a Pell equation.} For fixed squarefree $r$, the condition $r(a(a+1)) = r$ is equivalent to
    $$a(a+1) = r t^2$$
    for some integer $t \geq 1$. Setting $u = 2a+1$ and noting $4a(a+1) = u^2 - 1$, we get the Pell equation
    $$u^2 - 4 r t^2 = 1,$$
    with $u$ odd and $u = 2a+1$. Thus $A_r(N)$ equals the number of (integer) solutions $(u,t)$ of $u^2 - 4 r t^2 = 1$ with $1 \leq u \leq 2N+1$ and $u$ odd.

    Let $u_1 > 1$ be the smallest $u$ for which $u^2 - 4 r t^2 = 1$ has a solution. Then all solutions are given by
    $$u_m + t_m\sqrt{4r} = (u_1 + t_1\sqrt{4r})^m, \quad m = 1,2,\ldots$$

    Let $\alpha = u_1 + t_1\sqrt{4r} > 1$. Since $(u_1 + t_1\sqrt{4r})(u_1 - t_1\sqrt{4r}) = 1$, we have $u_m = (\alpha^m + \alpha^{-m})/2 \geq \alpha^m/2$. Hence
    $$A_r(N) \geq m \Rightarrow u_m \leq 2N+1 \Rightarrow \alpha^m \leq 2 u_m \leq 4N+2 \Rightarrow u_1 < \alpha \leq (4N+2)^{1/m}. \quad (*)$$

    \textbf{2) Bounding how many $r$ have many $a$.} For $m \geq 2$, let
    $$T_m(N) := \#\{\text{squarefree } r : A_r(N) \geq m\}.$$

    By $(*)$, for each $r$ counted in $T_m(N)$ we have $u_1 \leq (4N+2)^{1/m}$. The map $r \mapsto u_1$ is injective: indeed, from $u_1^2 - 1 = 4 r t_1^2$ one sees $r$ is the squarefree kernel of $(u_1^2 - 1)/4$, uniquely determined by $u_1$. Therefore
    $$T_m(N) \leq \#\{\text{odd } u \leq (4N+2)^{1/m}\} \ll N^{1/m}.$$

    \textbf{3) Bounding the off-diagonal.} Using $\binom{A}{2} = \sum_{m=2}^A (m-1)$, we get
    $$P_N = \sum_r \binom{A_r(N)}{2} = \sum_{m=2}^{\infty} (m-1) T_m(N) \ll \sum_{m=2}^{\infty} (m-1) N^{1/m} = O(N^{1/2}),$$
    since the series is dominated by the $m=2$ term and the tail is $O(N^{1/3})$.

    Consequently, $S_N = N + 2P_N = N + O(N^{1/2}) = N + o(N)$, and hence
    $$\lim_{N\to\infty} S_N / N = 1.$$

    \textbf{Final answer:} $1$

\end{document}