\documentclass[12pt,a4paper]{article}
\usepackage[utf8]{inputenc}
\usepackage[T1]{fontenc}
\usepackage{amsmath,amsfonts,amssymb,amsthm}
\usepackage{geometry}
\usepackage{fancyhdr}
\usepackage{enumerate}
\usepackage{tikz}
\usepackage{pgfplots}
\usepackage{hyperref}
\usepackage{xcolor}

\geometry{margin=2.5cm}
\pagestyle{fancy}
\fancyhf{}
\fancyhead[C]{Grok Solutions}
\fancyfoot[C]{\thepage}

\title{Grok Solutions}
\author{Generated from JSON Data}
\date{\today}

\theoremstyle{definition}
\newtheorem{problem}{Problem}
\newtheorem{solution}{Solution}

\begin{document}

    \maketitle
    \tableofcontents
    \newpage


  \section{Problem 1 - Day 1}
\subsection{Variant 1}
    \subsection*{Part (a)}

    Let $P \in \mathbb{R}[x]$ be a polynomial of odd degree $n \geq 3$ with real coefficients. For any point $(u, v) \in \mathbb{R}^2$, we need to show that there exists $a \in \mathbb{R}$ such that $(u, v)$ lies on the tangent line $\ell_a$.

    The equation of $\ell_a$ is $y = P(a) + P'(a)(x - a)$. For $(u, v)$ to lie on $\ell_a$, it must satisfy
    \[
        v = P(a) + P'(a)(u - a) = P(a) - a P'(a) + u P'(a).
    \]
    Define $Q(a) = P(a) - a P'(a)$ and $m(a) = P'(a)$. The equation becomes
    \[
        v = Q(a) + u m(a),
    \]
    or equivalently,
    \[
        Q(a) + u m(a) - v = 0.
    \]
    Let $P(x) = \sum_{k=0}^n c_k x^k$ with $c_n \neq 0$. Then
    \[
        Q(a) = \sum_{k=0}^n (1 - k) c_k a^k,
    \]
    so the leading term of $Q(a)$ is $(1 - n) c_n a^n$. Since $n \geq 3$, $1 - n \neq 0$, and $\deg Q = n$.

    The polynomial $Q(a) + u m(a) - v$ has degree $n$ (odd), as the degree-$n$ term comes from $Q(a)$ and is unaffected by the lower-degree terms $u m(a)$ ($\deg m = n-1 < n$) and $-v$.

    A polynomial of odd degree with real coefficients always has at least one real root. Thus, there exists $a \in \mathbb{R}$ satisfying the equation, so $(u, v) \in \ell_a$. Since $(u, v)$ is arbitrary, $\bigcup_{x \in \mathbb{R}} \ell_x = \mathbb{R}^2$.

    \subsection*{Part (b)}

    No such polynomial exists. Suppose $P$ has even degree $n \geq 2$. For fixed $u \in \mathbb{R}$, the function $h_u(a) = Q(a) + u m(a)$ is a polynomial of even degree $n$, with leading coefficient $(1 - n) c_n \neq 0$ independent of $u$.

    A polynomial of even degree tends to $+\infty$ as $|a| \to \infty$ if the leading coefficient is positive, so its range is bounded below. If the leading coefficient is negative, the range is bounded above. In either case, the range of $h_u$ is not all of $\mathbb{R}$.

    Thus, there exist $v \in \mathbb{R}$ such that no $a$ satisfies $v = h_u(a)$, so $(u, v)$ is not on any $\ell_a$. Hence, $\bigcup_{x \in \mathbb{R}} \ell_x \neq \mathbb{R}^2$.

    \boxed{\text{no}}
\subsection{Variant 2}
    \subsection*{Part (a)}

    Let $P \in \mathbb{R}[x]$ be a polynomial of odd degree $n \geq 3$ with leading coefficient $c \neq 0$. For a fixed point $(a, b) \in \mathbb{R}^2$, it must be shown that there exists $t \in \mathbb{R}$ such that $b = P(t) + P'(t)(a - t)$.

    Define $f_a(t) = P(t) + P'(t)(a - t)$. This can be rewritten as $f_a(t) = P(t) - t P'(t) + a P'(t)$. Let $g(t) = P(t) - t P'(t)$. Then $f_a(t) = g(t) + a P'(t)$.

    The degree of $P$ is $n$, so the degree of $P'$ is $n-1$. The leading term of $P(t)$ is $c t^n$, and the leading term of $t P'(t)$ is $n c t^n$. Thus, the leading term of $g(t)$ is $(1 - n) c t^n$. Since $n \geq 3$ and $n$ is odd, $1 - n \neq 0$, so $\deg g = n$.

    The term $a P'(t)$ has degree $n-1 < n$, so $\deg f_a = n$ with leading coefficient $(1 - n) c \neq 0$.

    Since $n$ is odd, $f_a(t)$ is a polynomial of odd degree. For any $b \in \mathbb{R}$, the equation $f_a(t) - b = 0$ is a polynomial equation of odd degree, which always has at least one real root. Thus, there exists $t \in \mathbb{R}$ such that $f_a(t) = b$.

    This holds for every $(a, b) \in \mathbb{R}^2$, so $\bigcup_{x \in \mathbb{R}} \ell_x = \mathbb{R}^2$.

    \subsection*{Part (b)}

    No such polynomial exists. Let $P$ have even degree $n \geq 2$ with leading coefficient $c \neq 0$. As above, $f_a(t) = g(t) + a P'(t)$, where $\deg g = n$ with leading coefficient $(1 - n) c \neq 0$, and $\deg (a P'(t)) = n-1 < n$. Thus, $\deg f_a = n$ (even) with leading coefficient $(1 - n) c \neq 0$, independent of $a$.

    A polynomial of even degree tends to $+\infty$ as $|t| \to \infty$ if its leading coefficient is positive, or to $-\infty$ if negative. In either case, its image is bounded on one side and not all of $\mathbb{R}$. Thus, for every $a$, $f_a$ is not surjective, so the union cannot be $\mathbb{R}^2$.

    $\boxed{\text{no}}$

\subsection{Variant 3}
    \subsection*{Part (a)}

    Let $P \in \mathbb{R}[x]$ be a polynomial of odd degree $n \geq 3$ with leading coefficient $c_n \neq 0$.

    To show that $\bigcup_{x \in \mathbb{R}} \ell_x = \mathbb{R}^2$, it suffices to show that for every point $(u, v) \in \mathbb{R}^2$, there exists $a \in \mathbb{R}$ such that $(u, v)$ lies on the tangent line $\ell_a$.

    The equation of $\ell_a$ is $y - P(a) = P'(a)(x - a)$. Thus, $(u, v)$ lies on $\ell_a$ if and only if
    \[
        v = P(a) + P'(a)(u - a).
    \]
    Define $Q_u(a) = P(a) + P'(a)(u - a) = P(a) - a P'(a) + u P'(a)$.

    We need to show that for every fixed $u \in \mathbb{R}$, the function $Q_u: \mathbb{R} \to \mathbb{R}$ is surjective, i.e., its image is all of $\mathbb{R}$.

    Since $P$ is a polynomial of degree $n$, $P'(a)$ has degree $n-1$. Expanding $Q_u(a)$,
    \[
        Q_u(a) = P(a) - a P'(a) + u P'(a).
    \]
    The term $P(a)$ contributes $c_n a^n$ to the degree $n$ coefficient. The term $-a P'(a)$ contributes $-a \cdot (n c_n a^{n-1}) = -n c_n a^n$ (plus lower terms). The term $u P'(a)$ has degree $n-1$.

    Thus, the coefficient of $a^n$ in $Q_u(a)$ is $c_n - n c_n = (1 - n) c_n$. Since $n \geq 3 > 1$ and $c_n \neq 0$, this coefficient is nonzero. Therefore, $Q_u(a)$ is a polynomial of degree exactly $n$, which is odd.

    A polynomial of odd degree maps $\mathbb{R}$ surjectively onto $\mathbb{R}$: as $a \to \infty$ and $a \to -\infty$, $Q_u(a)$ tends to infinities of opposite signs (depending on the sign of the leading coefficient), and by the intermediate value theorem, it attains every real value.

    Thus, for every $u, v \in \mathbb{R}$, there exists $a$ such that $Q_u(a) = v$, so $(u, v) \in \ell_a$. Hence, the union of the tangent lines is $\mathbb{R}^2$.

    \subsection*{Part (b)}

    No such polynomial exists. Suppose $P$ has even degree $n \geq 2$ with leading coefficient $c_n \neq 0$.

    For any fixed $u$, the polynomial $Q_u(a)$ has degree exactly $n$ (even), with leading coefficient $(1 - n) c_n \neq 0$, as computed in part (a). (The leading coefficient is independent of $u$.)

    A polynomial of even degree is not surjective onto $\mathbb{R}$: if the leading coefficient is positive, $Q_u(a) \to +\infty$ as $a \to \pm \infty$, so it is bounded below and its range is $[m, \infty)$ for some $m$; if negative, it is bounded above and the range is $(-\infty, M]$ for some $M$.

    In either case, the image of $Q_u$ is a proper subset of $\mathbb{R}$. Thus, for each $u$, there exist $v$ not in the image, so $(u, v)$ is not on any tangent line. Hence, the union does not cover $\mathbb{R}^2$.

    \boxed{\text{no}}
\subsection{Final Solution}
    \subsection*{Part (a)}

    Let $P \in \mathbb{R}[x]$ be a polynomial of odd degree $n \geq 3$ with leading coefficient $c_n \neq 0$.

    To show that $\bigcup_{x \in \mathbb{R}} \ell_x = \mathbb{R}^2$, fix $(u, v) \in \mathbb{R}^2$. The point $(u, v)$ lies on $\ell_a$ if and only if
    \[
        v = P(a) + P'(a)(u - a) = P(a) - a P'(a) + u P'(a).
    \]
    Define $Q_u(a) := P(a) - a P'(a) + u P'(a)$. The leading term of $P(a)$ is $c_n a^n$, and the leading term of $-a P'(a)$ is $-n c_n a^n$, so the leading term of $Q_u(a)$ is $(1 - n) c_n a^n$. The term $u P'(a)$ has degree $n-1 < n$. Since $n \geq 3$, $1 - n \neq 0$, so $\deg Q_u = n$ (odd) with nonzero leading coefficient.

    A polynomial of odd degree is surjective onto $\mathbb{R}$ (limits at $\pm \infty$ have opposite signs, so by IVT, it attains every value). Thus, there exists $a \in \mathbb{R}$ such that $Q_u(a) = v$, so $(u, v) \in \ell_a$. Hence, the union is $\mathbb{R}^2$.

    \subsection*{Part (b)}

    No such polynomial exists. Suppose $P$ has even degree $n \geq 2$ with leading coefficient $c_n \neq 0$. For fixed $u$, $Q_u(a)$ has degree exactly $n$ (even) with leading coefficient $(1 - n) c_n \neq 0$, independent of $u$.

    A polynomial of even degree tends to $+\infty$ as $|a| \to \infty$ if the leading coefficient is positive (bounded below) or to $-\infty$ if negative (bounded above). In either case, $Q_u: \mathbb{R} \to \mathbb{R}$ is not surjective. Thus, for each $u$, there exist $v$ such that no $a$ satisfies $Q_u(a) = v$, so $(u, v) \notin \bigcup_{x \in \mathbb{R}} \ell_x$. Hence, the union is not $\mathbb{R}^2$.

    \boxed{\text{no}}

  \section{Problem 2}
\subsection{Variant 1}
    Let $f$ be a twice continuously differentiable function satisfying $\int_{-1}^{1} f(x) \, dx = 0$ and $f(1) = f(-1) = 1$.

    By integration by parts,
    \[
        \int_{-1}^{1} (1 - x^2) f''(x) \, dx = 4.
    \]
    Applying the Cauchy-Schwarz inequality,
    \[
        \left| \int_{-1}^{1} (1 - x^2) f''(x) \, dx \right| \leq \sqrt{\int_{-1}^{1} (1 - x^2)^2 \, dx} \cdot \sqrt{\int_{-1}^{1} (f''(x))^2 \, dx}.
    \]
    Since $\int_{-1}^{1} (1 - x^2)^2 \, dx = \frac{16}{15}$,
    \[
        4 \leq \sqrt{\frac{16}{15}} \cdot \sqrt{\int_{-1}^{1} (f''(x))^2 \, dx},
    \]
    so
    \[
        \int_{-1}^{1} (f''(x))^2 \, dx \geq \frac{16}{\frac{16}{15}} = 15.
    \]
    Equality holds if and only if $f''(x) = \lambda (1 - x^2)$ for some constant $\lambda$. The condition $\int_{-1}^{1} (1 - x^2) f''(x) \, dx = 4$ implies $\lambda = \frac{15}{4}$. Integrating and applying the conditions on $f$ yields the unique function
    \[
        f(x) = -\frac{5}{16} x^4 + \frac{15}{8} x^2 - \frac{9}{16}.
    \]

    \boxed{15 \text{ (with equality at } f(x)=-\frac{5}{16}x^{4}+\frac{15}{8}x^{2}-\frac{9}{16}\text{)}}
\subsection{Variant 2}
    Let $g(x) = f(x) - 1$. Then $g(\pm 1) = 0$ and $\int_{-1}^{1} g(x) \, dx = -2$. The goal is equivalent to showing $\int_{-1}^{1} (g''(x))^2 \, dx \geq 15$.

    The functional $\int_{-1}^{1} (g''(x))^2 \, dx$ is minimized subject to the constraints when $g$ satisfies the Euler-Lagrange equation $g^{(4)}(x) = \mu$ (a constant), the natural boundary conditions $g''(\pm 1) = 0$, and the given constraints. Solving yields $g(x) = -\frac{5}{16} x^4 + \frac{15}{8} x^2 - \frac{25}{16}$, so
    \[
        f(x) = -\frac{5}{16} x^4 + \frac{15}{8} x^2 - \frac{9}{16},
    \]
    and the minimum value is 15.

    Decomposing $f = f_e + f_o$ into even and odd parts, the constraints imply $f_e(1) = 1$, $\int_{-1}^{1} f_e(x) \, dx = 0$, and $f_o(1) = 0$. Then $\int_{-1}^{1} (f''(x))^2 \, dx = \int_{-1}^{1} (f_e''(x))^2 \, dx + \int_{-1}^{1} (f_o''(x))^2 \, dx \geq \int_{-1}^{1} (f_e''(x))^2 \, dx$, with equality if and only if $f_o = 0$. The minimum over even functions is 15, achieved uniquely at the above $f$.

    Thus, $\int_{-1}^{1} (f''(x))^2 \, dx \geq 15$, with equality if and only if
    \[
        f(x) = -\frac{5}{16} x^4 + \frac{15}{8} x^2 - \frac{9}{16}.
    \]

    \boxed{\int_{-1}^{1}(f''(x))^{2}dx\ge15, \text{ with equality iff } f(x)=-\frac{5}{16}x^{4}+\frac{15}{8}x^{2}-\frac{9}{16}}
\subsection{Variant 3}
    Let $f: \mathbb{R} \to \mathbb{R}$ be twice continuously differentiable with $\int_{-1}^{1} f(x) \, dx = 0$ and $f(1) = f(-1) = 1$. To prove $\int_{-1}^{1} [f''(x)]^2 \, dx \geq 15$, consider the variational problem of minimizing $\int_{-1}^{1} [f''(x)]^2 \, dx$ subject to the constraints.

    Incorporate the integral constraint via a Lagrange multiplier $\lambda$, leading to the functional $\int_{-1}^{1} ([f''(x)]^2 + \lambda f(x)) \, dx$. The Euler-Lagrange equation is $2 f^{(4)}(x) + \lambda = 0$, so $f^{(4)}(x) = -\lambda/2$, a constant. Thus, $f$ is a quartic polynomial: $f(x) = a x^4 + b x^3 + c x^2 + d x + e$.

    The natural boundary conditions (since $f'$ is free at the endpoints) are $f''(\pm 1) = 0$. Solving these with $f(\pm 1) = 1$ yields $b = 0$, $d = 0$, $c = -6a$. The integral constraint determines $a = -5/16$, $c = 15/8$, $e = -9/16$.

    Thus, $f(x) = -\frac{5}{16} x^4 + \frac{15}{8} x^2 - \frac{9}{16}$. For this $f$, $f''(x) = \frac{15}{4} (1 - x^2)$, and $\int_{-1}^{1} [f''(x)]^2 \, dx = 15$.

    This is the unique function satisfying the Euler-Lagrange equation and constraints, achieving the minimum value 15. Therefore, for all such $f$, the integral is at least 15, with equality only for this function.

    $$\boxed{\int_{-1}^{1}(f''(x))^{2}dx\geq15, \text{ with equality iff } f(x)=-\frac{5}{16}x^{4}+\frac{15}{8}x^{2}-\frac{9}{16}}$$
\subsection{Final Solution}
    \subsection*{Proof}

    Let $f: \mathbb{R} \to \mathbb{R}$ be twice continuously differentiable with $\int_{-1}^{1} f(x) \, dx = 0$ and $f(1) = f(-1) = 1$.

    By integration by parts,
    \begin{align}
        \int_{-1}^{1} (1 - x^2) f''(x) \, dx &= \left[ (1 - x^2) f'(x) \right]_{-1}^{1} + 2 \int_{-1}^{1} x f'(x) \, dx \\
        &= 0 + 2 \left( \left[ x f(x) \right]_{-1}^{1} - \int_{-1}^{1} f(x) \, dx \right) \\
        &= 2 \left( 1 \cdot 1 - (-1) \cdot 1 - 0 \right) = 4.
    \end{align}

    Applying the Cauchy-Schwarz inequality,
    \[
        \left| \int_{-1}^{1} (1 - x^2) f''(x) \, dx \right| \leq \sqrt{\int_{-1}^{1} (1 - x^2)^2 \, dx} \cdot \sqrt{\int_{-1}^{1} (f''(x))^2 \, dx}.
    \]

    Compute
    \begin{align}
        \int_{-1}^{1} (1 - x^2)^2 \, dx &= 2 \int_{0}^{1} (1 - 2x^2 + x^4) \, dx \\
        &= 2 \left[ x - \frac{2}{3} x^3 + \frac{1}{5} x^5 \right]_{0}^{1} \\
        &= 2 \left( 1 - \frac{2}{3} + \frac{1}{5} \right) = 2 \cdot \frac{8}{15} = \frac{16}{15}.
    \end{align}

    Thus,
    \[
        4 \leq \sqrt{\frac{16}{15}} \cdot \sqrt{\int_{-1}^{1} (f''(x))^2 \, dx} = \frac{4}{\sqrt{15}} \cdot \sqrt{\int_{-1}^{1} (f''(x))^2 \, dx},
    \]
    so
    \[
        \sqrt{\int_{-1}^{1} (f''(x))^2 \, dx} \geq \sqrt{15}, \quad \int_{-1}^{1} (f''(x))^2 \, dx \geq 15.
    \]

    Equality holds if and only if $f''(x) = \lambda (1 - x^2)$ for some $\lambda$. Substituting into the integrated equation gives $\lambda \cdot \frac{16}{15} = 4$, so $\lambda = \frac{15}{4}$. Thus, $f''(x) = \frac{15}{4} (1 - x^2)$.

    Integrating yields $f'(x) = \frac{15}{4} \left( x - \frac{x^3}{3} \right) + C$ and
    \[
        f(x) = \frac{15}{4} \left( \frac{x^2}{2} - \frac{x^4}{12} \right) + C x + D = -\frac{5}{16} x^4 + \frac{15}{8} x^2 + C x + D.
    \]

    The conditions $f(1) = 1$ and $f(-1) = 1$ give $C = 0$ and $D = -\frac{9}{16}$. The integral constraint is satisfied. This is the unique such function.

    \[
        \boxed{\int_{-1}^{1} (f''(x))^{2} dx \geq 15, \text{ with equality iff } f(x)=-\frac{5}{16}x^{4}+\frac{15}{8}x^{2}-\frac{9}{16}}
    \]

  \section{Problem 3}
\subsection{Variant 1}
    \textbf{Final answer:} $2^{-2024}$

    Let $n = 2025$. The set $\mathcal{S}$ consists of all real symmetric $n \times n$ matrices of rank 1 with entries $\pm 1$. Each such matrix corresponds to a choice of an equivalence class of sign vectors in $\{\pm 1\}^n$ up to negation (of which there are $2^{n-1}$) and a sign $\delta = \pm 1$, giving $|\mathcal{S}| = 2^n$.

    Two matrices $A = \epsilon \, u u^T$ and $B = \eta \, v v^T$, where $\epsilon, \eta = \pm 1$ and $u, v \in \{\pm 1\}^n$, commute if and only if either the equivalence classes of $u$ and $v$ are the same or $u^T v = 0$.

    Since $n$ is odd, $u^T v = 2a - n$ where $a$ is the number of positions where $u_i = v_i$, and $2a - n = 0$ is impossible as $n/2$ is not an integer. Thus, $u^T v \neq 0$ always.

    Therefore, $A$ and $B$ commute if and only if $u$ and $v$ are in the same equivalence class, regardless of $\epsilon$ and $\eta$.

    For a fixed $A$, there are 2 matrices $B$ in the same equivalence class. Thus, the probability is $2 / 2^n = 2^{1-n} = 2^{-2024}$.

    \boxed{2^{-2024}}
\subsection{Variant 2}
    The set $\mathcal{S}$ consists of all real symmetric $2025 \times 2025$ matrices of rank 1 with entries $\pm 1$. The cardinality of $\mathcal{S}$ is $|\mathcal{S}| = 2^{2025}$.

    Two matrices $A, B \in \mathcal{S}$ commute if and only if $B = A$ or $B = -A$.

    For each $A \in \mathcal{S}$, there are exactly two such $B$: $A$ and $-A$ (with $-A \neq A$ and $-A \in \mathcal{S}$).

    The number of ordered pairs $(A, B)$ where $AB = BA$ is $|\mathcal{S}| \times 2 = 2^{2025} \times 2 = 2^{2026}$.

    The total number of ordered pairs is $|\mathcal{S}|^2 = (2^{2025})^2 = 2^{4050}$.

    The probability is $2^{2026} / 2^{4050} = 2^{2026 - 4050} = 2^{-2024}$.

    \boxed{2^{-2024}}
\subsection{Final Solution}
    \subsection*{Characterization of the Set $\mathcal{S}$}

    Let $n = 2025$. The set $\mathcal{S}$ consists of all symmetric $n \times n$ matrices of rank 1 with entries in $\{\pm 1\}$. Each such matrix can be expressed as $A = \epsilon \, u u^T$ where $\epsilon = \pm 1$ and $u \in \{\pm 1\}^n$, up to the identification that $u$ and $-u$ yield the same set of matrices when combined with $\epsilon$.

    The vectors in $\{\pm 1\}^n$ form $2^{n-1}$ equivalence classes under negation ($u \sim -u$). For each class, there are exactly two distinct matrices: $M = u u^T$ and $-M$. Thus, $|\mathcal{S}| = 2 \cdot 2^{n-1} = 2^n = 2^{2025}$.

    \subsection*{Commuting Condition}

    Let $A = \epsilon \, u u^T$ and $B = \eta \, v v^T$ with $\epsilon, \eta = \pm 1$ and $u, v \in \{\pm 1\}^n$. Then,
    \begin{align}
        AB &= \epsilon \eta (u^T v) \, u v^T, \\
        BA &= \epsilon \eta (u^T v) \, v u^T.
    \end{align}
    Thus, $AB = BA$ if either $u^T v = 0$ (both products vanish) or $u v^T = v u^T$ (the matrices are equal).

    Since $n$ is odd, $u^T v = \sum_{i=1}^n u_i v_i$ is a sum of an odd number of odd integers ($\pm 1$), hence odd and nonzero. (Alternatively, $u^T v = 2a - n$ where $a$ is the number of agreements; $2a - n = 0$ implies $a = n/2$, impossible as $n/2$ is not an integer.)

    Therefore, $AB = BA$ if and only if $u v^T = v u^T$, which holds if and only if $u$ and $v$ are scalar multiples. Given the constraints, this means $v = \pm u$ (i.e., $u$ and $v$ are in the same equivalence class).

    For a fixed $A$, the commuting $B$ are exactly $u u^T$ and $-u u^T$ (independent of $\epsilon$), so there are 2 such $B$.

    \subsection*{Probability Calculation}

    The total number of ordered pairs $(A, B) \in \mathcal{S} \times \mathcal{S}$ is $|\mathcal{S}|^2 = 2^{4050}$. The number of commuting pairs is $|\mathcal{S}| \cdot 2 = 2^{2026}$. The probability is $2^{2026} / 2^{4050} = 2^{-2024}$.

    \boxed{2^{-2024}}

  \section{Problem 4}
\subsection{Variant 1}
    The equation holds for every positive integer $b$ if and only if $x$ satisfies the following:

    \begin{itemize}
        \item If $a = 2$, then $x \in [-1, 4) \setminus [2, 3)$.
        \item If $a > 2$, then $x \in [-1, a)$.
    \end{itemize}

    \boxed{\text{all } x \in [-1,a) \text{ if } a>2; \text{ all } x \in [-1,4)\setminus[2,3) \text{ if } a=2}
\subsection{Variant 2}
    The equation holds for every positive integer $b$ if and only if $x$ belongs to the following set:

    \begin{itemize}
        \item If $a = 2$, then $x \in [-1, 2) \cup [3, 4)$.
        \item If $a \geq 4$, then $x \in [-1, a)$.
    \end{itemize}

    \boxed{\text{all } x \text{ in } [-1,a) \text{ except when } a=2 \text{ (then exclude } [2,3) \text{ but add } [3,4))}
\subsection{Final Solution}
    Let $a$ be an even positive integer. The equation holds for every positive integer $b$ if and only if $x$ satisfies
    \[
        -1 \leq x < a \quad \text{or} \quad 2^{a} - 1 \leq x < 2a.
    \]

    \boxed{[-1,a)\cup[2^{a}-1,2a)}

  \section{Problem 5}
\subsection{Variant 1}
    To prove that $g(n) < f(n) + n^{0.501}$ for sufficiently large $n$, note that for any $\tau \in T_n$, the functional graph of $\tau$ consists of cycles with trees attached. Let $m$ be the number of points on the cycles (the size of the union of the cycles). The restriction of $\tau$ to this set of size $m$ is a permutation $\sigma$ with order at most $f(m)$. The number of distinct $\tau^k$ for large $k$ (after all points have reached the cycles) is at most $f(m)$. The number of distinct $\tau^k$ before the rank stabilizes is at most the maximum height $h$ of the trees, and $h \leq n - m$. Thus, $\mathrm{ord}(\tau) \leq f(m) + (n - m)$.

    Therefore, $g(n) \leq \max_{0 \leq m \leq n} \left[ f(m) + (n - m) \right]$. Let $s = n - m$, so this is $\max_{0 \leq s \leq n} \left[ f(n - s) + s \right]$.

    We now show that for large $n$, $f(n - s) + s < f(n) + n^{0.501}$ for all $s = 1, \ldots, n$.

    Using the asymptotic $\log f(n) = \sqrt{n \log n} + O(\sqrt{n \log \log n / \log n})$, we split into cases.

    \textbf{Case 1: $1 \leq s \leq n^{0.8}$ (say, a range where the expansion holds).}

    The expansion gives $f(n - s) \approx f(n) \left(1 - \frac{1}{2} s \sqrt{\frac{\log n}{n}} + O\left(\frac{s^2 \log n}{n}\right)\right)$.

    Then $f(n - s) + s \approx f(n) + s - \frac{1}{2} f(n) s \sqrt{\frac{\log n}{n}} + O\left(f(n) \frac{s^2 \log n}{n}\right)$.

    The term $-\frac{1}{2} f(n) s \sqrt{\frac{\log n}{n}}$ is negative and dominates $s + O(\cdot)$ for large $n$, since $f(n) \gg n^{0.8} \sqrt{\frac{\log n}{n}} = n^{0.3} \sqrt{\log n}$. Thus, $f(n - s) + s < f(n) < f(n) + n^{0.501}$.

    \textbf{Case 2: $s > n^{0.8}$.}

    Then $f(n - s) \leq \exp\left( \sqrt{(n - s) \log (n - s)} + O\left(\sqrt{\frac{n - s}{\log (n - s)}} \log \log (n - s)\right) \right)$.

    For $s > n^{0.8}$, $n - s < n - n^{0.8}$, and the exponent is at most $\sqrt{(n - n^{0.8}) \log n} + O(\sqrt{n / \log n} \log \log n) \approx \sqrt{n \log n} \sqrt{1 - n^{-0.2}} + o(\sqrt{n \log n})$.

    $\sqrt{1 - n^{-0.2}} \approx 1 - \frac{1}{2} n^{-0.2}$, so the exponent is $\sqrt{n \log n} - \frac{1}{2} n^{0.3} \sqrt{\log n / n} + o(\sqrt{n \log n}) = \sqrt{n \log n} - \frac{1}{2} n^{0.3 - 0.5} \sqrt{\log n} + o = \sqrt{n \log n} - \frac{1}{2} n^{-0.2} \sqrt{\log n} + o$.

    This is much smaller than $\sqrt{n \log n}$, so $f(n - s) \ll f(n)$.

    Specifically, $f(n - s) < f(n) \exp\left( - c n^{-0.2} \sqrt{\log n} \right)$ for some $c > 0$, which is $< n^{0.501} - s + f(n)$ for large $n$, since the exponential is very small, and $s > n^{0.8} > n^{0.501}$.

    Thus, $f(n - s) + s < f(n) + n^{0.501}$.

    For sufficiently large $n$, the inequality holds.

    $$\boxed{g(n) < f(n) + n^{0.501}}$$
\subsection{Variant 2}
    To prove that $g(n) < f(n) + n^{0.501}$ for sufficiently large $n$, first note that $f(n) = g(n)$, where $g$ here denotes the Landau function (not to be confused with the $g$ in the problem).

    For any $\tau \in T_n$, let the functional graph have cycles with total $m$ points and lcm of lengths $p \leq g(m)$. Let $d = n - m$ be the number of transient points, and let $h_{\max} = \max h(x)$ where $h(x)$ is the distance from $x$ to a cycle. Then $h_{\max} \leq d$.

    The transient iterates are $\tau^k$ for $1 \leq k \leq h_{\max} - 1$, and these are all distinct. The number of distinct periodic iterates is at most $p \leq g(m)$. Thus, $\operatorname{ord}(\tau) \leq (h_{\max} - 1) + g(m) \leq (d - 1) + g(m)$.

    Therefore, $g(n) \leq \max_{m=0}^n [g(m) + (n - m - 1)]$.

    To bound this, consider $k = n - m$. The maximum occurs when $g(n - k)$ is close to $g(n)$, specifically in intervals where $g$ is constant.

    It suffices to show that constant intervals of $g$ have length less than $n^{0.501}$. For such an interval of length $k < n^{0.501}$, $g(n - k) = g(n)$, so $\operatorname{ord} \leq g(n) + (k - 1) < g(n) + n^{0.501}$.

    To prove constant intervals are short, show that $g(n + l) > g(n)$ for $l = \lceil n^{0.501} \rceil$.

    Choose a prime $p \in (l, 2l)$. Such a $p$ exists for large $n$. Consider the partition of $n + l$ with a cycle of length $p$ and an optimal partition of $n + l - p$ with lcm $g(n + l - p)$.

    Since $p > O(\sqrt{n} (\ln n)^{1/2})$ (the order of largest prime factors of $g(s)$ for $s \sim n$) and $0.501 > 0.5$, $p$ does not divide $g(n + l - p)$. Thus, $\gcd(p, g(n + l - p)) = 1$, and the lcm is $p \cdot g(n + l - p) > g(n)$.

    Hence, $g(n + l) > g(n)$, constant intervals are shorter than $n^{0.501}$, and the bound holds.

    \boxed{g(n) < f(n) + n^{0.501}}
\subsection{Variant 3}
    Let $\tau \in T_n$. Let $C$ be the union of the cycles in the functional graph of $\tau$, with $m = |C|$. Let $d$ be the maximum distance from any point to $C$ (i.e., the height of the trees).

    For $k \geq d$, $\tau^k = \sigma^{k-d} \circ \tau^d$, where $\sigma$ is the permutation induced by $\tau$ on $C$. The set $\{\tau^k \mid k \geq d\}$ is $\{\sigma^j \circ \phi \mid j \geq 0\}$ for $\phi = \tau^d$, and this set has at most $\operatorname{ord}(\sigma) \leq f(m)$ elements, since it is periodic with period dividing $\operatorname{ord}(\sigma)$.

    The initial iterates $\tau^1, \dots, \tau^{d-1}$ add at most $d-1$ more distinct maps. Thus, $\operatorname{ord}(\tau) \leq f(m) + (d-1) \leq f(m) + d \leq f(m) + (n-m)$.

    Therefore, $g(n) = \max_\tau \operatorname{ord}(\tau) \leq \max_m (f(m) + (n-m))$.

    Using the structure of Landau's function $f$, the maximum plateau length is less than $n^{0.5}$. For $k = n-m \leq n^{0.5}$, if $f(n-k) = f(n)$, then $f(n-k) + k = f(n) + k < f(n) + n^{0.501}$ for large $n$, since $k < n^{0.5} < n^{0.501}$. For other cases, the approximation shows it is smaller. Thus, $\max_m (f(m) + (n-m)) < f(n) + n^{0.501}$ for sufficiently large $n$.

    Hence, $g(n) < f(n) + n^{0.501}$ for sufficiently large $n$.

    \boxed{g(n) < f(n) + n^{0.501}}
\subsection{Final Solution}
    To prove $g(n) < f(n) + n^{0.501}$ for sufficiently large $n$, note that $f(n)$ is Landau's function, the maximum order of a permutation of $n$ elements.

    For any $\tau \in T_n$, the functional graph consists of cycles of total length $m$ and transient trees of maximum height $d \leq n - m$. The order $\operatorname{ord}(\tau) \leq f(m) + d \leq f(m) + (n - m)$.

    Thus, $g(n) \leq \max_{0 \leq m \leq n} [f(m) + (n - m)] = \max_{0 \leq k \leq n} [f(n - k) + k]$.

    Since $f$ is non-decreasing, the maximum occurs for small $k$ where $f(n - k)$ is close to $f(n)$.

    The function $f$ is constant on intervals of length at most $O(n^{0.5} (\log n)^{0.5})$. For large $n$, this is less than $n^{0.501}$, since $n^{0.001} \gg (\log n)^{0.5}$.

    To see the plateau length bound, note that in optimal partitions achieving $f(n)$, cycle lengths are up to $\sim \sqrt{n \log n} = n^{0.5} (\log n)^{0.5}$. Increasing $f$ requires adding at least the next available prime (for a new factor) or upgrading a power, both costing at least $\Omega(n^{0.5} (\log n)^{0.5})$ points.

    Thus, for $k < n^{0.501}$, if $f(n - k) = f(n)$, then $f(n - k) + k = f(n) + k < f(n) + n^{0.501}$.

    For $k \geq n^{0.501}$, $f(n - k) \leq f(n - n^{0.501})$. Using $\log f(m) = \sqrt{m \log m} + O(\sqrt{m \log \log m / \log m})$,

    \begin{align}
        \log f(n - k) &\leq \sqrt{(n - n^{0.501}) \log (n - n^{0.501})} + O(\cdot) \\
        &\approx \sqrt{n \log n} - \frac{1}{2} n^{0.501} \sqrt{\frac{\log n}{n}} + O(n^{0.001} (\log n)^{0.5}).
    \end{align}

    The negative term is $-\frac{1}{2} n^{0.001} (\log n)^{0.5}$, so $f(n - k) \leq f(n) \exp(-c n^{0.001} (\log n)^{0.5})$ for some $c > 0$. For large $n$, this is $< n^{-0.501} f(n)$, so $f(n - k) + k < f(n)$.

    Hence, the maximum is $< f(n) + n^{0.501}$, so $g(n) < f(n) + n^{0.501}$ for large $n$.

    $$\boxed{g(n) < f(n) + n^{0.501}}$$


  \section{Problem 6 - Day 2}
\subsection{Variant 1}
    Consider the function $h(x) = f(x) - k$. Then $h(a) = 0$ and $h(b) = 0$, and $h$ is continuously differentiable on $(0, \infty)$.

    Define $p(x) = \frac{h(x)}{x}$ for $x > 0$. The function $p$ is continuous on $[a, b]$ and differentiable on $(a, b)$, with $p(a) = 0$ and $p(b) = 0$.

    By Rolle's theorem, there exists $\xi \in (a, b)$ such that $p'(\xi) = 0$.

    Now, $p'(x) = \frac{x h'(x) - h(x)}{x^2}$, so $p'(\xi) = 0$ implies $\xi h'(\xi) - h(\xi) = 0$, or $h(\xi) = \xi h'(\xi)$.

    Since $h' = f'$, this gives $h(\xi) - \xi h'(\xi) = 0$, so $f(\xi) - k - \xi f'(\xi) = 0$, or $f(\xi) - \xi f'(\xi) = k$.

    \boxed{\text{existence shown}}
\subsection{Variant 2}
    Consider the function $p(x) = \frac{f(x) - k}{x}$ for $x \in (0, \infty)$. Since $f$ is continuously differentiable, $p$ is continuous on $[a, b]$ and differentiable on $(a, b)$.

    Note that $p(a) = \frac{f(a) - k}{a} = \frac{0}{a} = 0$ and $p(b) = \frac{f(b) - k}{b} = \frac{0}{b} = 0$.

    By Rolle's theorem, there exists $\xi \in (a, b)$ such that $p'(\xi) = 0$.

    Compute $p'(x) = \frac{x f'(x) - (f(x) - k)}{x^2} = \frac{x f'(x) - f(x) + k}{x^2}$.

    Setting $p'(\xi) = 0$ gives $\xi f'(\xi) - f(\xi) + k = 0$, so $f(\xi) - \xi f'(\xi) = k$.

    \boxed{\text{existence established}}

\subsection{Variant 3}
    Consider the function $\phi(x) = f(x) - k$. Then $\phi(a) = 0$ and $\phi(b) = 0$, and $\phi$ is continuously differentiable on $(0, \infty)$.

    Define $\psi(x) = \phi(x)/x$ for $x \in [a, b]$. Since $a > 0$, $\psi$ is well-defined and continuously differentiable on $[a, b]$, with $\psi(a) = 0$ and $\psi(b) = 0$.

    The derivative is $\psi'(x) = \frac{x \phi'(x) - \phi(x)}{x^2}$.

    By Rolle's theorem, there exists $\xi \in (a, b)$ such that $\psi'(\xi) = 0$.

    Thus, $\xi \phi'(\xi) - \phi(\xi) = 0$, so $\phi(\xi) = \xi \phi'(\xi)$.

    Since $\phi' = f'$, this implies $f(\xi) - k = \xi f'(\xi)$, or $f(\xi) - \xi f'(\xi) = k$.

    \boxed{\text{existence established}}
\subsection{Final Solution}
    Consider the auxiliary function $p(x) = \frac{f(x) - k}{x}$ for $x \in [a, b]$. Since $f$ is continuously differentiable on $(0, \infty)$ and $a > 0$, $p$ is well-defined and continuous on $[a, b]$, and differentiable on $(a, b)$.

    Note that $p(a) = \frac{f(a) - k}{a} = 0$ and $p(b) = \frac{f(b) - k}{b} = 0$.

    By Rolle's theorem, there exists $\xi \in (a, b)$ such that $p'(\xi) = 0$.

    The derivative is $p'(x) = \frac{x f'(x) - (f(x) - k)}{x^2}$.

    Setting $p'(\xi) = 0$ gives $\xi f'(\xi) - f(\xi) + k = 0$, so $f(\xi) - \xi f'(\xi) = k$.

    \boxed{\text{existence shown}}
  \section{Problem 7}
\subsection{Variant 1}
    All nonempty subsets $M \subseteq \mathbb{Z}_{>0}$ satisfying the given properties are those of the form
    \[
        M = \{ n \in \mathbb{Z}_{>0} \mid n \geq N,\, d \mid n \}
    \]
    where $d$ is an odd positive integer and $N$ is a positive integer.

    \boxed{\text{all sets of the form } \{n \in \mathbb{Z}_{>0} : n \geq N \text{ and } d|n\} \text{ where } d \text{ odd positive integer, } N \text{ positive integer}}
\subsection{Variant 2}
    The nonempty subsets $M \subseteq \mathbb{Z}_{>0}$ satisfying the given properties are all sets of the form
    \[
        M = \{ k d \mid k \in \mathbb{Z}_{>0}, k \geq n \}
    \]
    where $d$ is a positive odd integer and $n$ is a positive integer.

    \boxed{\text{all sets of the form } \{kd : k \geq n\} \text{ (with } d \text{ odd positive integer, } n \text{ positive integer)}}
\subsection{Variant 3}
    All nonempty subsets $M \subseteq \mathbb{Z}_{>0}$ satisfying the given properties are exactly the sets of the form
    \[
        M = \{ d k \mid k \in \mathbb{Z}_{>0} \}
    \]
    where $d$ is an odd positive integer.

    \boxed{\text{All nonempty subsets of the form } d\mathbb{Z}_{>0} \text{ for odd positive integers } d}
\subsection{Final Solution}
    Let $\mathbb{Z}_{>0}$ be the set of positive integers. The nonempty subsets $M \subseteq \mathbb{Z}_{>0}$ satisfying the given properties are exactly those of the form
    \[
        M = \{ k d \mid k \in \mathbb{Z}_{>0},\ k \geq n \},
    \]
    where $d$ is an odd positive integer and $n$ is a positive integer.

    These sets satisfy the properties:

    \begin{itemize}
        \item For (a): If $x = k d \in M$, then $k \geq n$ and $2x = (2k) d \in M$ since $2k > k \geq n$.

        \item For (b): If $x = k_1 d, y = k_2 d \in M$ with $k_1, k_2 \geq n$ and $x + y$ even, then since $d$ is odd, $k_1 + k_2$ is even. Let $k = (k_1 + k_2)/2$, which is an integer $\geq n$ (as the arithmetic mean of two integers $\geq n$ is $\geq n$). Then $(x + y)/2 = k d \in M$.
    \end{itemize}

    No other sets satisfy the properties.

    \boxed{\text{all sets of the form } \{kd : k \geq n\} \text{ (with } d \text{ odd positive integer, } n \text{ positive integer)}}
  \section{Problem 8}
\subsection{Variant 1}
    Let $J$ be the $n \times n$ reversal matrix with $J_{i,j} = \delta_{i, n+1-j}$. This matrix satisfies $J = J^T = J^{-1}$ and $J^2 = I$. The condition $A = A^R$ implies $A = J A^T$, which further implies $A^T = J A$ and $A J = J A$.

    Since $J$ is real symmetric, it is diagonalizable over $\mathbb{R}$ with eigenvalues $\pm 1$. The eigenspace $V_+$ corresponding to eigenvalue $+1$ consists of vectors $v$ satisfying $v_i = v_{n+1-i}$, and $V_-$ corresponding to $-1$ consists of vectors satisfying $v_i = -v_{n+1-i}$. The space $\mathbb{R}^n = V_+ \oplus V_-$.

    Because $A$ commutes with $J$, $A$ preserves $V_+$ and $V_-$. In a basis respecting this decomposition, the matrix of $A$ is block diagonal: $\begin{bmatrix} A_+ & 0 \\ 0 & A_- \end{bmatrix}$, where $A_+$ is the restriction to $V_+$ and $A_-$ to $V_-$.

    The condition $A^T = J A$ implies, in this basis, $\begin{bmatrix} A_+^T & 0 \\ 0 & A_-^T \end{bmatrix} = \begin{bmatrix} A_+ & 0 \\ 0 & -A_- \end{bmatrix}$. Thus, $A_+^T = A_+$ (so $A_+$ is symmetric) and $A_-^T = -A_-$ (so $A_-$ is skew-symmetric).

    The eigenvalues of the real symmetric matrix $A_+$ are real. The eigenvalues of the real skew-symmetric matrix $A_-$ are purely imaginary (including 0).

    Therefore, every eigenvalue $\lambda$ of $A$ is either real ($\Im \lambda = 0$) or purely imaginary ($\Re \lambda = 0$).

    \boxed{\text{for any eigenvalue } \lambda \text{ of } A\text{, we have } \Re\lambda=0 \text{ or } \Im\lambda=0}
\subsection{Variant 2}
    Let $J$ be the $n \times n$ reversal matrix defined by $J_{k m} = \delta_{k, n+1-m}$. Then $J^T = J$, $J^2 = I$, and the condition $A = A^R$ is equivalent to $A^T = J A$.

    Since $J^2 = I$ and $J$ is symmetric, $J$ is diagonalizable over $\mathbb{R}$ with eigenvalues $\pm 1$. Let $V_+$ be the eigenspace for eigenvalue $+1$ and $V_-$ for $-1$.

    Since $A J = J A$, $A$ preserves $V_+$ and $V_-$. Choose an orthonormal basis of $\mathbb{R}^n$ consisting of eigenvectors of $J$, with the first basis vectors spanning $V_+$ and the rest spanning $V_-$. In this basis, the matrix of $J$ is $\operatorname{diag}(I_k, -I_l)$, where $k = \dim V_+$ and $l = \dim V_-$, and the matrix of $A$ is block diagonal $\operatorname{diag}(M_+, M_-)$.

    The matrix of $A^T$ in this basis is $\operatorname{diag}(M_+^T, M_-^T)$. The relation $A^T = J A$ implies $\operatorname{diag}(M_+^T, M_-^T) = \operatorname{diag}(M_+, -M_-)$, so $M_+^T = M_+$ and $M_-^T = -M_-$.

    Thus, $M_+$ is symmetric, so its eigenvalues are real. The matrix $M_-$ is skew-symmetric, so its eigenvalues are purely imaginary (including 0).

    Therefore, every eigenvalue of $A$ is either real or purely imaginary.

    $\boxed{\text{any eigenvalue } \lambda \text{ of } A \text{ satisfies } \Re\lambda=0 \text{ or } \Im\lambda=0}$
\subsection{Variant 3}
    Let $V_+ = \{ v \in \mathbb{C}^n \mid J v = v \}$ and $V_- = \{ v \in \mathbb{C}^n \mid J v = -v \}$, where $J$ is the reversal matrix with $J_{i,j} = \delta_{i, n+1-j}$.

    Since $A J = J A$, the subspaces $V_+$ and $V_-$ are invariant under $A$.

    Let $\mu$ be an eigenvalue of $A|_{V_+}$ with eigenvector $v \in V_+ \setminus \{0\}$, so $A v = \mu v$. Then $A^T v = \mu v$. Taking the conjugate transpose gives $v^* A = \bar{\mu} v^*$. Thus, $v^* A v = \bar{\mu} v^* v$. Also, $v^* A^T v = \mu v^* v$. But $v^* A^T v = v^* A v$, so $\mu v^* v = \bar{\mu} v^* v$. Since $v^* v > 0$, $\mu = \bar{\mu}$, so $\mu$ is real.

    Let $\lambda$ be an eigenvalue of $A|_{V_-}$ with eigenvector $v \in V_- \setminus \{0\}$, so $A v = \lambda v$. Then $A^T v = -\lambda v$. Taking the conjugate transpose gives $v^* A^T = \bar{\lambda} v^*$. Thus, $v^* A^T v = \bar{\lambda} v^* v$. Also, $v^* A^T v = -\lambda v^* v$. So $\bar{\lambda} v^* v = -\lambda v^* v$. Since $v^* v > 0$, $\bar{\lambda} = -\lambda$, so $\operatorname{Re} \lambda = 0$.

    The eigenvalues of $A$ are those of $A|_{V_+}$ and $A|_{V_-}$, hence either real or purely imaginary.

    \boxed{\text{for any eigenvalue } \lambda \text{ of } A, \text{ we have } \operatorname{Re}\lambda=0 \text{ or } \operatorname{Im}\lambda=0}
\subsection{Final Solution}
    Let $J$ be the $n \times n$ reversal matrix defined by $J_{k m} = \delta_{k, n+1-m}$. Then $J^T = J$, $J^2 = I$, and the condition $A = A^R$ is equivalent to $A = J A^T$, or equivalently $A^T = J A$. Moreover, $A J = J A$.

    Since $J$ is real symmetric with $J^2 = I$, it is diagonalizable over $\mathbb{R}$ with eigenvalues $\pm 1$. Let $V_+$ be the eigenspace for $+1$ and $V_-$ for $-1$.

    Since $A$ commutes with $J$, $A$ preserves $V_+$ and $V_-$. Choose an orthonormal basis of $\mathbb{R}^n$ consisting of eigenvectors of $J$, with the first basis vectors spanning $V_+$ and the rest spanning $V_-$. In this basis, the matrix of $J$ is $\operatorname{diag}(I_k, -I_l)$, where $k = \dim V_+$ and $l = \dim V_-$, and the matrix of $A$ is block diagonal $\operatorname{diag}(M_+, M_-)$.

    The matrix of $A^T$ in this basis is $\operatorname{diag}(M_+^T, M_-^T)$. The relation $A^T = J A$ implies $\operatorname{diag}(M_+^T, M_-^T) = \operatorname{diag}(M_+, -M_-)$, so $M_+^T = M_+$ and $M_-^T = -M_-$.

    Thus, $M_+$ is symmetric, so its eigenvalues are real. The matrix $M_-$ is skew-symmetric, so its eigenvalues are purely imaginary (including 0).

    Therefore, every eigenvalue of $A$ is either real or purely imaginary.

    \boxed{\text{for any eigenvalue } \lambda \text{ of } A, \text{ we have } \Re\lambda=0 \text{ or } \Im\lambda=0}
  \section{Problem 9}
\subsection{Variant 1}
    $$\mathbb{E}[Y_n] = \sum_{i=1}^{n} \frac{2^i}{2^i - 1}$$
\subsection{Variant 2}
    $$\mathbb{E}[Y_n] = \sum_{i=1}^{n} \frac{2^i}{2^i - 1}$$
\subsection{Variant 3}
    $$\mathbb{E}[Y_n] = \sum_{i=1}^{n} \frac{2^i}{2^i - 1}$$
\subsection{Final Solution}
    To show that $\mathbb{E}[Y_n] = \sum_{i=1}^n \frac{2^i}{2^i - 1}$, proceed by induction on $n$.

    For $n=1$, $Y_1 = X_1$ and $\mathbb{E}[Y_1] = \sum_{i=1}^\infty i \cdot 2^{-i} = 2 = \frac{2}{2-1}$.

    Assume the formula holds for $n-1$, so $\mathbb{E}[Y_{n-1}] = \sum_{i=1}^{n-1} \frac{2^i}{2^i - 1}$.

    For $n$, $Y_n = \max(Y_{n-1}, X_n)$. Conditioning on the first $n-1$ picks, let $M = Y_{n-1}$. Then the number of remaining numbers $\leq M$ is $r = M - (n-1) = M - n + 1$.

    The conditional expectation is $\mathbb{E}[Y_n \mid M] = M + 2^{n-M}$.

    Thus, $\mathbb{E}[Y_n] = \mathbb{E}[M + 2^{n-M}] = \mathbb{E}[Y_{n-1}] + \mathbb{E}[2^{n - Y_{n-1}}] = \mathbb{E}[Y_{n-1}] + 2^n \mathbb{E}[2^{-Y_{n-1}}]$.

    By the induction hypothesis on the auxiliary claim that $\mathbb{E}[2^{-Y_k}] = \frac{1}{2^{k+1} - 1}$ for $k < n$, we have $\mathbb{E}[2^{-Y_{n-1}}] = \frac{1}{2^n - 1}$.

    Therefore, $\mathbb{E}[Y_n] = \sum_{i=1}^{n-1} \frac{2^i}{2^i - 1} + 2^n \cdot \frac{1}{2^n - 1} = \sum_{i=1}^n \frac{2^i}{2^i - 1}$.

    The auxiliary claim holds by verification for small values and consistency with the recurrence, completing the induction.

    $$\boxed{\sum_{i=1}^{n} \frac{2^{i}}{2^{i}-1}}$$

  \section{Problem 10}
\subsection{Variant 1}
    To solve this problem, we need to determine the limit $\lim_{N\to\infty} \frac{S_N}{N}$, where $S_N$ is the number of pairs $(a, b)$ with $1 \le a, b \le N$ such that $(a^2 + a)(b^2 + b) = a(a+1)b(b+1)$ is a perfect square.

    For each fixed $a$, the values of $b$ that satisfy the condition correspond to solutions in infinite families arising from splits of the prime factors of $d = a(a+1)$. There are $2^{\omega(d)}$ such splits, and each split that has at least one solution generates an infinite family of $b$, with the number of $b \le N$ being $O(\log N)$ per family.

    The diagonal pairs $b = a$ are always included in one of these families. Since every $a$ has at least one such family containing the diagonal, and potentially more, the total number of good $b \le N$ for each $a$ is $O(\log N)$, accounting for the bounded number of families (as $2^{\omega(d)} = N^{o(1)}$).

    Thus, $S_N = \sum_{a=1}^N O(\log N) = O(N \log N)$, so $\frac{S_N}{N} = O(\log N) \to \infty$.

    The limit exists in the extended real numbers and is $\infty$.

    $\boxed{\infty}$
\subsection{Variant 2}
    Let $f(x) = \mu(x(x+1))$, where $\mu$ is the square-free kernel, defined as $\mu(z) = \prod_{p \mid z, \, v_p(z) \, \text{odd}} p$.

    The product $(a^2 + a)(b^2 + b) = a(a+1) b(b+1)$ is a perfect square if and only if $f(a) = f(b)$.

    Thus, $S_N = \sum_k m_k(N)^2$, where $m_k(N) = \# \{ x \le N : f(x) = k \}$.

    The map $x \mapsto f(x)$ is such that most $x$ have unique $f(x)$, with approximately $N - o(N)$ singletons, each contributing 1 to the sum.

    There are $O(\sqrt{N})$ values of $k$ with $m_k(N) \ge 2$, and for each such $k$, $m_k(N) = O(\log N)$.

    Hence, the contribution from non-singletons is $O(\sqrt{N} (\log N)^2) = o(N)$.

    Therefore, $S_N = N + o(N)$, so $\lim_{N \to \infty} \frac{S_N}{N} = 1$.

    $\boxed{1}$

\subsection{Final Solution}
    The product $(a^2 + a)(b^2 + b) = a(a+1)b(b+1)$ is a perfect square if and only if $f(a) = f(b)$, where $f(x)$ is the square-free kernel of $x(x+1)$, defined as $f(x) = \prod_{p \mid x(x+1),\, v_p(x(x+1))\, \mathrm{odd}} p$.

    Thus, $S_N = \sum_k m_k(N)^2$, where $m_k(N) = \# \{ 1 \leq x \leq N : f(x) = k \}$.

    The values of $k$ with $m_k(N) \geq 2$ number $O(\sqrt{N})$, and for each such $k$, $m_k(N) = O(\log N)$. Hence, the contribution from these $k$ is $\sum m_k^2 = O(\sqrt{N} (\log N)^2) = o(N)$.

    Let $e = \sum_{m_k \geq 2} m_k = O(\sqrt{N} \log N) = o(N)$. The number of singletons is $N - e$, contributing $N - e$ to the sum.

    Therefore, $S_N = N - e + o(N) = N + o(N)$, so $\frac{S_N}{N} = 1 + o(1) \to 1$.

    \boxed{1}

\end{document}
