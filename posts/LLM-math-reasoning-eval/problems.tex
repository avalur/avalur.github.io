\documentclass{article}
\usepackage[a4paper, margin=1cm]{geometry}

\usepackage{amsmath}
\usepackage{amsfonts}
\usepackage{amssymb}
\usepackage{amsthm}
\usepackage{enumerate}

\title{IMC 2025}
\author{}
\date{}

\begin{document}

    \maketitle

    \section*{Day 1}

    \subsection*{Problem 1}
    Let $P \in \mathbb{R}[x]$ be a polynomial with real coefficients, and suppose $\deg(P)\ge2$. For every $x\in\mathbb{R}$, let $\ell_x\subset\mathbb{R}^2$ denote the line tangent to the graph of $P$ at the point $(x,P(x))$.

    \begin{enumerate}
        \item[(a)] Suppose that the degree of $P$ is odd. Show that
        \[
            \bigcup_{x\in\mathbb{R}}\ell_x \;=\; \mathbb{R}^2.
        \]
        \item[(b)] Does there exist a polynomial of even degree for which the above equality still holds?
    \end{enumerate}

    \subsection*{Problem 2}
    Let $f:\mathbb{R}\to\mathbb{R}$ be a twice continuously differentiable function, and suppose that
    \[
        \int_{-1}^{1} f(x)\,\mathrm{d}x = 0
        \quad\text{and}\quad
        f(1) = f(-1) = 1.
    \]
    Prove that
    \[
        \int_{-1}^{1} \bigl(f''(x)\bigr)^2 \,\mathrm{d}x \;\ge\; 15,
    \]
    and find all such functions for which equality holds.

    \subsection*{Problem 3}
    Denote by $\mathcal{S}$ the set of all real symmetric $2025\times2025$ matrices of rank~1 whose entries take values $-1$ or $+1$. Let $A,B\in\mathcal{S}$ be matrices chosen independently uniformly at random. Find the probability that $A$ and $B$ commute, i.e., $AB = BA$.

    \subsection*{Problem 4}
    Let $a$ be an even positive integer. Find all real numbers $x$ such that
    \[
        \left\lfloor \sqrt[a]{b^a + x} \cdot b^{a-1} \right\rfloor
        \;=\; b^a + \left\lfloor \frac{x}{a} \right\rfloor
    \]
    holds for every positive integer $b$.
    (Here $\lfloor x\rfloor$ denotes the largest integer that is no greater than $x$.)

    \subsection*{Problem 5}
    For a positive integer $n$, let $[n] = \{1,2,\dots,n\}$. Denote by $S_n$ the set of all bijections from $[n]$ to $[n]$, and let $T_n$ be the set of all maps from $[n]$ to $[n]$. Define the order $\operatorname{ord}(\tau)$ of a map $\tau\in T_n$ as the number of distinct maps in the set
    \[
        \{\tau,\;\tau\circ\tau,\;\tau\circ\tau\circ\tau,\;\dots\},
    \]
    where $\circ$ denotes composition.  Finally, let
    \[
        f(n) \;=\;\max_{\tau\in S_n}\operatorname{ord}(\tau)
        \quad\text{and}\quad
        g(n) \;=\;\max_{\tau\in T_n}\operatorname{ord}(\tau).
    \]
    Prove that
    \[
        g(n) \;<\; f(n) \;+\; n^{0.501}
    \]
    for sufficiently large $n$.

    \section*{Day 2}

    \subsection*{Problem 6}
    Let $f:(0,\infty)\to\mathbb{R}$ be a continuously differentiable function, and let $b>a>0$ be real numbers such that
    \[
        f(a) = f(b) = k.
    \]
    Prove that there exists a point $\xi \in (a,b)$ such that
    \[
        f(\xi) - \xi f'(\xi) = k.
    \]

    \subsection*{Problem 7}
    Let $\mathbb{Z}_{>0}$ be the set of positive integers. Find all nonempty subsets $M \subseteq \mathbb{Z}_{>0}$ satisfying both of the following properties:
    \begin{enumerate}
        \item[(a)] If $x \in M$, then $2x \in M$.
        \item[(b)] If $x,y \in M$ and $x + y$ is even, then $\tfrac{x + y}{2} \in M$.
    \end{enumerate}

    \subsection*{Problem 8}
    For an $n\times n$ real matrix $A\in M_n(\mathbb{R})$, denote by $A^R$ its counter-clockwise $90^\circ$ rotation.  For example,
    \[
        \begin{bmatrix}
            1 & 2 & 3 \\
            4 & 5 & 6 \\
            7 & 8 & 9
        \end{bmatrix}^R
        =
        \begin{bmatrix}
            3 & 6 & 9 \\
            2 & 5 & 8 \\
            1 & 4 & 7
        \end{bmatrix}.
    \]
    Prove that if $A = A^R$ then for any eigenvalue $\lambda$ of $A$, we have $\Re\lambda = 0$ or $\Im\lambda = 0$.

    \subsection*{Problem 9}
    Let $n$ be a positive integer. Consider the following random process which produces a sequence of $n$ distinct positive integers $X_1,X_2,\dots,X_n$. First, $X_1$ is chosen randomly with
    \[
        \mathbb{P}(X_1 = i) = 2^{-i}
        \quad\text{for every positive integer }i.
    \]
    For $1 \le j \le n-1$, having chosen $X_1,\dots,X_j$, arrange the remaining positive integers in increasing order as $n_1 < n_2 < \cdots$, and choose $X_{j+1}$ randomly with
    \[
        \mathbb{P}(X_{j+1} = n_i) = 2^{-i}
        \quad\text{for every positive integer }i.
    \]
    Let $Y_n = \max\{X_1,\dots,X_n\}$. Show that
    \[
        \mathbb{E}[Y_n]
        \;=\;
        \sum_{i=1}^{n}\frac{2^i}{2^i - 1},
    \]
    where $\mathbb{E}[Y_n]$ is the expected value of $Y_n$.

    \subsection*{Problem 10}
    For any positive integer $N$, let $S_N$ be the number of pairs of integers $1 \le a,b \le N$ such that the number
    \[
        (a^2 + a)\,(b^2 + b)
    \]
    is a perfect square. Prove that the limit
    \[
        \lim_{N\to\infty}\frac{S_N}{N}
    \]
    exists and find its value.

\end{document}