\documentclass[12pt,a4paper]{article}
\usepackage[utf8]{inputenc}
\usepackage[T1]{fontenc}
\usepackage{amsmath,amsfonts,amssymb,amsthm}
\usepackage{geometry}
\usepackage{fancyhdr}
\usepackage{enumerate}
\usepackage{tikz}
\usepackage{pgfplots}
\usepackage{hyperref}
\usepackage{xcolor}

\geometry{margin=2.5cm}
\pagestyle{fancy}
\fancyhf{}
\fancyhead[C]{Deepseek Solutions}
\fancyfoot[C]{\thepage}

\title{Deepseek Solutions}
\author{Generated from JSON Data}
\date{\today}

\theoremstyle{definition}
\newtheorem{problem}{Problem}
\newtheorem{solution}{Solution}

\begin{document}

    \maketitle
    \tableofcontents
    \newpage
\section{Problem 1}

\subsection{Variant 1}

\subsection*{Part (a)}

Let $P$ be a polynomial of odd degree $n \geq 3$ (since $\deg(P) \geq 2$ and odd). For any point $(X, Y) \in \mathbb{R}^2$, consider the equation that must hold for $(X, Y)$ to lie on the tangent line $\ell_x$ at $(x, P(x))$:

\[
Y = P(x) + P'(x)(X - x).
\]

Rearrange this equation as:

\[
P(x) + P'(x)(X - x) - Y = 0.
\]

Define the function $g(x) = P(x) + P'(x)(X - x) - Y$. Since $P$ is a polynomial, $g(x)$ is also a polynomial in $x$. Let $P(x) = a_n x^n + a_{n-1} x^{n-1} + \cdots + a_0$ with $a_n \neq 0$. Then $P'(x) = n a_n x^{n-1} + (n-1) a_{n-1} x^{n-2} + \cdots$. The leading term of $g(x)$ comes from $P(x)$ and $-x P'(x)$:

\begin{itemize}
\item $P(x)$ contributes $a_n x^n$.
\item $-x P'(x)$ contributes $-x \cdot (n a_n x^{n-1}) = -n a_n x^n$.
\end{itemize}

Thus, the leading term of $g(x)$ is $a_n x^n - n a_n x^n = a_n (1 - n) x^n$. Since $n \geq 2$, $1 - n \neq 0$ and $a_n \neq 0$, so $\deg(g) = n$, which is odd.

A polynomial of odd degree has at least one real root because it tends to $+\infty$ as $x \to +\infty$ and to $-\infty$ as $x \to -\infty$ (or vice-versa, depending on the leading coefficient), and by the intermediate value theorem, it must cross zero. Therefore, there exists a real number $x$ such that $g(x) = 0$, meaning that $(X, Y)$ lies on the tangent line $\ell_x$.

This holds for every $(X, Y) \in \mathbb{R}^2$, so $\bigcup_{x \in \mathbb{R}} \ell_x = \mathbb{R}^2$.

\subsection*{Solution to Part (b)}

No, there does not exist a polynomial of even degree for which $\bigcup_{x \in \mathbb{R}} \ell_x = \mathbb{R}^2$.

To see this, suppose $P$ has even degree $n \geq 2$. For a fixed $X \in \mathbb{R}$, define the function $h_X(x) = P(x) + P'(x)(X - x)$. As in part (a), $h_X(x)$ is a polynomial in $x$ of degree $n$, even, with leading coefficient $a_n (1 - n) \neq 0$ since $n \geq 2$.

Because the degree is even:
\begin{itemize}
\item If the leading coefficient is positive, then $h_X(x) \to +\infty$ as $|x| \to \infty$, so $h_X(x)$ has a global minimum $m_X$, and its range is $[m_X, \infty)$.
\item If the leading coefficient is negative, then $h_X(x) \to -\infty$ as $|x| \to \infty$, so $h_X(x)$ has a global maximum $M_X$, and its range is $(-\infty, M_X]$.
\end{itemize}

In both cases, the range of $h_X(x)$ is not all of $\mathbb{R}$. For example:
\begin{itemize}
\item If the range is $[m_X, \infty)$, then for any $Y < m_X$, there is no $x$ such that $h_X(x) = Y$.
\item If the range is $(-\infty, M_X]$, then for any $Y > M_X$, there is no $x$ such that $h_X(x) = Y$.
\end{itemize}

This means that for each $X$, there exist values of $Y$ such that the point $(X, Y)$ does not lie on any tangent line $\ell_x$. Therefore, $\bigcup_{x \in \mathbb{R}} \ell_x \neq \mathbb{R}^2$.

Since this argument holds for any even degree polynomial, no such polynomial exists that satisfies the equality.

Final answer for part (b): No.

\subsection{Variant 2}

\subsection*{Part (a)}
Let $P$ be a polynomial of odd degree $n \geq 2$ with real coefficients. For any point $(a, b) \in \mathbb{R}^2$, we need to show that there exists some $x \in \mathbb{R}$ such that the tangent line $\ell_x$ to the graph of $P$ at $(x, P(x))$ passes through $(a, b)$.

The equation of the tangent line at $x$ is given by:
\[
Y - P(x) = P'(x)(X - x).
\]
Substituting $(X, Y) = (a, b)$ yields:
\[
b - P(x) = P'(x)(a - x),
\]
which can be rearranged as:
\[
b = P(x) + P'(x)(a - x).
\]
Define the function $g_a : \mathbb{R} \to \mathbb{R}$ by:
\[
g_a(x) = P(x) + P'(x)(a - x).
\]
Since $P$ is a polynomial of degree $n$, $P'$ is a polynomial of degree $n-1 \geq 1$ (as $n \geq 2$). The expression for $g_a(x)$ is:
\[
g_a(x) = P(x) + a P'(x) - x P'(x).
\]
The term of highest degree in $g_a(x)$ arises from $P(x)$ and $-x P'(x)$. Let $P(x) = c_n x^n + c_{n-1} x^{n-1} + \cdots + c_0$ with $c_n \neq 0$. Then:
\[
P'(x) = n c_n x^{n-1} + (n-1) c_{n-1} x^{n-2} + \cdots,
\]
\[
x P'(x) = n c_n x^n + (n-1) c_{n-1} x^{n-1} + \cdots.
\]
The leading terms in $g_a(x)$ are:
\[
P(x) \text{ contributes } c_n x^n, \quad -x P'(x) \text{ contributes } -n c_n x^n,
\]
so the coefficient of $x^n$ in $g_a(x)$ is:
\[
c_n - n c_n = c_n (1 - n).
\]
Since $n \geq 2$ and odd, $1 - n \neq 0$ and $c_n \neq 0$, so $g_a(x)$ is a polynomial of degree $n$. As $n$ is odd, $g_a(x)$ is an odd-degree polynomial. An odd-degree polynomial is surjective from $\mathbb{R}$ to $\mathbb{R}$ because $\lim_{x \to \infty} g_a(x) = \pm \infty$ and $\lim_{x \to -\infty} g_a(x) = \mp \infty$ (depending on the sign of the leading coefficient), and by the intermediate value theorem, it takes all real values.

Therefore, for any fixed $a$ and any $b \in \mathbb{R}$, there exists $x \in \mathbb{R}$ such that $g_a(x) = b$, meaning:
\[
b = P(x) + P'(x)(a - x).
\]
This implies that the tangent line at $x$ passes through $(a, b)$. Hence, for every $(a, b) \in \mathbb{R}^2$, there is a tangent line containing it, and:
\[
\bigcup_{x \in \mathbb{R}} \ell_x = \mathbb{R}^2.
\]

\subsection*{Part (b)}
We need to determine if there exists a polynomial $P$ of even degree at least 2 such that $\bigcup_{x \in \mathbb{R}} \ell_x = \mathbb{R}^2$.

Suppose $P$ has even degree $n \geq 2$. For any fixed $a \in \mathbb{R}$, define $g_a(x) = P(x) + P'(x)(a - x)$. As derived in part (a), $g_a(x)$ is a polynomial of degree $n$ (since the leading coefficient $c_n (1 - n) \neq 0$ for $n \geq 2$).

A non-constant polynomial of even degree is not surjective from $\mathbb{R}$ to $\mathbb{R}$. If the leading coefficient is positive, then $\lim_{|x| \to \infty} g_a(x) = +\infty$, so $g_a(x)$ has a global minimum and does not take values less than this minimum. If the leading coefficient is negative, then $\lim_{|x| \to \infty} g_a(x) = -\infty$, so $g_a(x)$ has a global maximum and does not take values greater than this maximum. In either case, there exists some $b \in \mathbb{R}$ such that $g_a(x) \neq b$ for all $x \in \mathbb{R}$.

This means that for each $a \in \mathbb{R}$, there is a $b \in \mathbb{R}$ such that the point $(a, b)$ is not on any tangent line $\ell_x$. Therefore, $\bigcup_{x \in \mathbb{R}} \ell_x \neq \mathbb{R}^2$ for any even-degree polynomial of degree at least 2.

Thus, no such polynomial exists.

    \subsection{Variant 3}

    \subsubsection*{Part (a)}
    Let $P \in \mathbb{R}[x]$ be a polynomial of odd degree $n \geq 2$ with real coefficients. Fix an arbitrary point $(X, Y) \in \mathbb{R}^2$. The goal is to show that $(X, Y)$ lies on the tangent line to the graph of $P$ at some point $(a, P(a))$ for some $a \in \mathbb{R}$.

    The tangent line to the graph of $P$ at $x = a$ has the equation:
    \[
        y - P(a) = P'(a)(x - a).
    \]
    The point $(X, Y)$ lies on this line if and only if:
    \[
        Y - P(a) = P'(a)(X - a),
    \]
    which rearranges to:
    \[
        Y = P(a) + P'(a)(X - a).
    \]
    Define the function $g: \mathbb{R} \to \mathbb{R}$ by:
    \[
        g(a) = P(a) + P'(a)(X - a).
    \]
    This is a polynomial in $a$ for fixed $X$. Let $P(a) = c_n a^n + c_{n-1} a^{n-1} + \cdots + c_0$ with $c_n \neq 0$. The derivative is $P'(a) = n c_n a^{n-1} + (n-1) c_{n-1} a^{n-2} + \cdots + c_1$. Then:
    \[
        g(a) = P(a) + P'(a)(X - a) = P(a) + X P'(a) - a P'(a).
    \]
    The leading term of $g(a)$ comes from the highest-degree terms in $P(a)$ and $-a P'(a)$:
    \begin{itemize}
        \item $P(a)$ has leading term $c_n a^n$.
        \item $-a P'(a)$ has leading term $-a \cdot n c_n a^{n-1} = -n c_n a^n$.
    \end{itemize}
    Combining these, the coefficient of $a^n$ is $c_n - n c_n = (1 - n) c_n$.

    Since $\deg(P) = n \geq 2$ and $n$ is odd, $n \geq 3$. Thus, $1 - n \neq 0$ and $c_n \neq 0$, so $\deg(g) = n$, which is odd. The leading coefficient is $(1 - n) c_n$, which is nonzero.

    A real polynomial of odd degree is surjective onto $\mathbb{R}$ by the intermediate value theorem because as $a \to \infty$, $g(a) \to \infty$ if the leading coefficient is positive, or $g(a) \to -\infty$ if the leading coefficient is negative, and similarly in the opposite direction as $a \to -\infty$. Thus, $g(a)$ takes every real value as $a$ varies over $\mathbb{R}$.

    In particular, for the fixed $X$ and any $Y$, there exists some $a \in \mathbb{R}$ such that $g(a) = Y$. This means:
    \[
        Y = P(a) + P'(a)(X - a),
    \]
    so $(X, Y)$ lies on the tangent line at $(a, P(a))$.

    Since $(X, Y)$ was arbitrary, every point in $\mathbb{R}^2$ lies on some tangent line to the graph of $P$. Therefore:
    \[
        \bigcup_{x \in \mathbb{R}} \ell_x = \mathbb{R}^2.
    \]

    \subsubsection*{Part (b)}
    No, there does not exist a polynomial of even degree at least 2 for which $\bigcup_{x \in \mathbb{R}} \ell_x = \mathbb{R}^2$.

    Suppose $P$ has even degree $n \geq 2$. For a fixed $X \in \mathbb{R}$, define $g(a) = P(a) + P'(a)(X - a)$ as in part (a). As before, $g(a)$ is a polynomial in $a$ of degree $n$ (even), with leading coefficient $(1 - n) c_n \neq 0$ (since $n \geq 2$ and $c_n \neq 0$).

    A real polynomial of even degree

    \subsection{Final Solution}

    \subsubsection*{Part (a)}
    Let $P$ be a polynomial of odd degree $n \geq 2$ with real coefficients. Fix an arbitrary point $(X, Y) \in \mathbb{R}^2$. The tangent line to the graph of $P$ at a point $(a, P(a))$ has the equation:
    \[
        y - P(a) = P'(a)(x - a).
    \]
    The point $(X, Y)$ lies on this tangent line if and only if:
    \[
        Y = P(a) + P'(a)(X - a).
    \]
    Define the function $g: \mathbb{R} \to \mathbb{R}$ by:
    \[
        g(a) = P(a) + P'(a)(X - a) - Y.
    \]
    This is a polynomial in $a$ for fixed $X$ and $Y$. Let $P(a) = c_n a^n + c_{n-1} a^{n-1} + \cdots + c_0$ with $c_n \neq 0$. The derivative is $P'(a) = n c_n a^{n-1} + (n-1) c_{n-1} a^{n-2} + \cdots$. Substituting into $g(a)$:
    \begin{itemize}
        \item $P(a)$ has leading term $c_n a^n$.
        \item $-a P'(a)$ has leading term $-a \cdot n c_n a^{n-1} = -n c_n a^n$.
    \end{itemize}
    The leading term of $g(a)$ is $c_n a^n - n c_n a^n = (1 - n) c_n a^n$.

    Since $\deg(P) = n \geq 2$ and $n$ is odd, $n \geq 3$. Thus, $1 - n \neq 0$ and $c_n \neq 0$, so $\deg(g) = n$, which is odd. The leading coefficient is $(1 - n) c_n$, which is nonzero.

    A real polynomial of odd degree takes all real values (it is surjective) by the intermediate value theorem. Therefore, there exists $a \in \mathbb{R}$ such that $g(a) = 0$, meaning:
    \[
        Y = P(a) + P'(a)(X - a).
    \]
    Thus, $(X, Y)$ lies on the tangent line at $(a, P(a))$. Since $(X, Y)$ was arbitrary, every point in $\mathbb{R}^2$ lies on some tangent line, so:
    \[
        \bigcup_{x \in \mathbb{R}} \ell_x = \mathbb{R}^2.
    \]

    \subsubsection*{Part (b)}
    No, there does not exist a polynomial of even degree at least 2 for which the union of all tangent lines covers $\mathbb{R}^2$.

    Suppose $P$ has even degree $n \geq 2$. For a fixed $X \in \mathbb{R}$, define the polynomial in $a$:
    \[
        h_X(a) = P(a) + P'(a)(X - a).
    \]
    As in part (a), $h_X(a)$ is a polynomial of degree $n$ (even), with leading coefficient $(1 - n) c_n \neq 0$. A real polynomial of even degree is not surjective: if the leading coefficient is positive, $h_X(a) \to +\infty$ as $|a| \to \infty$, so it has a global minimum and misses all values below it; if negative, it has a global maximum and misses all values above it. Thus, for each $X$, there exists some $Y$ (e.g., below the global minimum or above the global maximum) such that $h_X(a) \neq Y$ for all $a$, meaning:
    \[
        Y \neq P(a) + P'(a)(X - a)
    \]
    for all $a$. Therefore, the point $(X, Y)$ does not lie on any tangent line. Since this holds for every even-degree polynomial, no such polynomial exists.

    \textbf{Final answer for (b):} No.
    \[
        \boxed{\text{no}}
    \]

\section{Problem 2}
    \subsection{Final Solution}
    The problem requires proving that $\int_{-1}^{1} [f''(x)]^2 \, dx \geq 15$ for any twice continuously differentiable function $f: \mathbb{R} \to \mathbb{R}$ satisfying $\int_{-1}^{1} f(x) \, dx = 0$ and $f(1) = f(-1) = 1$, and finding all functions achieving equality.

    \subsection*{Proof of the Inequality}
    To solve this, consider the minimization of the functional $\int_{-1}^{1} [f''(x)]^2 \, dx$ subject to the constraints. Using the calculus of variations with the integral constraint $\int_{-1}^{1} f(x) \, dx = 0$ and boundary conditions $f(-1) = f(1) = 1$, the Euler-Lagrange equation yields $f^{(4)}(x) = \lambda$ for some constant $\lambda$. Thus, the minimizer is a quartic polynomial:
    \[
        f(x) = a x^4 + b x^3 + c x^2 + d x + e.
    \]
    The constraints are:
    \begin{enumerate}
        \item $f(1) = a + b + c + d + e = 1$,
        \item $f(-1) = a - b + c - d + e = 1$,
        \item $\int_{-1}^{1} f(x) \, dx = 0$.
    \end{enumerate}

    Adding the first two constraints gives $2a + 2c + 2e = 2$, so:
    \[
        a + c + e = 1. \quad (1)
    \]
    Subtracting them gives $2b + 2d = 0$, so:
    \[
        b + d = 0. \quad (2)
    \]
    The integral constraint simplifies using symmetry (odd functions integrate to zero):
    \[
        \int_{-1}^{1} (a x^4 + c x^2 + e) \, dx = 2 \left[ \frac{a x^5}{5} + \frac{c x^3}{3} + e x \right]_0^1 = 2 \left( \frac{a}{5} + \frac{c}{3} + e \right) = 0,
    \]
    so:
    \[
        \frac{a}{5} + \frac{c}{3} + e = 0. \quad (3)
    \]

    The second derivative is:
    \[
        f''(x) = 12a x^2 + 6b x + 2c.
    \]
    Then:
    \[
        [f''(x)]^2 = (12a x^2 + 6b x + 2c)^2 = 144a^2 x^4 + 144ab x^3 + 36b^2 x^2 + 48ac x^2 + 24bc x + 4c^2.
    \]
    Integrating from $-1$ to $1$, the odd terms ($144ab x^3$ and $24bc x$) vanish, and the even part gives:
    \begin{align*}
        \int_{-1}^{1} [f''(x)]^2 \, dx &= 2 \int_0^1 (144a^2 x^4 + (36b^2 + 48ac) x^2 + 4c^2) \, dx \\
        &= 2 \left[ \frac{144a^2}{5} + (12b^2 + 16ac) + 4c^2 \right] \\
        &= \frac{288a^2}{5} + 24b^2 + 32ac + 8c^2.
    \end{align*}

    Since $24b^2 \geq 0$ and equality holds only if $b = 0$ (and thus $d = 0$ from (2)), set $b = d = 0$. The constraints reduce to:
    \begin{itemize}
        \item $a + c + e = 1$,
        \item $\frac{a}{5} + \frac{c}{3} + e = 0$.
    \end{itemize}

    Subtracting the second from the first:
    \[
        (a + c + e) - \left( \frac{a}{5} + \frac{c}{3} + e \right) = 1 - 0 \implies \frac{4a}{5} + \frac{2c}{3} = 1.
    \]
    Solving for $c$:
    \[
        c = \frac{3}{2} - \frac{6a}{5}.
    \]
    Substitute into the integral:
    \[
        I(a) = \frac{288a^2}{5} + 32a \left( \frac{3}{2} - \frac{6a}{5} \right) + 8 \left( \frac{3}{2} - \frac{6a}{5} \right)^2 = \frac{768}{25} a^2 + \frac{96}{5} a + 18.
    \]
    This quadratic in $a$ has a minimum at:
    \[
        a = -\frac{96/5}{2 \cdot 768/25} = -\frac{5}{16}.
    \]
    Then:
    \[
        c = \frac{3}{2} - \frac{6}{5} \left( -\frac{5}{16} \right) = \frac{15}{8}, \quad e = 1 - \left( -\frac{5}{16} \right) - \frac{15}{8} = -\frac{9}{16}.
    \]
    For $f(x) = -\frac{5}{16} x^4 + \frac{15}{8} x^2 - \frac{9}{16}$:
    \[
        f''(x) = \frac{15}{4} (1 - x^2), \quad [f''(x)]^2 = \frac{225}{16} (1 - 2x^2 + x^4),
    \]
    and:
    \[
        \int_{-1}^{1} [f''(x)]^2 \, dx = \frac{225}{16} \int_{-1}^{1} (1 - 2x^2 + x^4) \, dx = 15.
    \]
    For any other function satisfying the constraints, strict convexity and the Euler-Lagrange equation ensure the integral is at least 15, with equality only for this quartic polynomial.

    \subsection*{Functions Achieving Equality}
    Equality holds if and only if:
    \[
        f(x) = -\frac{5}{16} x^4 + \frac{15}{8} x^2 - \frac{9}{16}.
    \]

    \textbf{Final answer:} The minimum value of the integral is 15, achieved by the function $f(x) = -\frac{5}{16} x^4 + \frac{15}{8} x^2 - \frac{9}{16}$. For all other such functions, the integral exceeds 15.

    \boxed{15}

    \section{Problem 3}
    \subsection{Final solution}
    The set $\mathcal{S}$ consists of all real symmetric $2025 \times 2025$ matrices of rank 1 with entries in $\{-1, +1\}$. Each matrix in $\mathcal{S}$ is uniquely determined by its first row, which is a vector in $\{-1, +1\}^{2025}$. Thus, the size of $\mathcal{S}$ is $2^{2025}$.

    Each matrix $A \in \mathcal{S}$ corresponds to a sign vector $\mathbf{s}^A = (s_1^A, \dots, s_n^A) \in \{-1, +1\}^n$ with $n = 2025$, such that the entries of $A$ are given by:
    \[
        A_{ij} = s_i^A s_j^A s_1^A, \quad \text{for all } i,j.
    \]
    Similarly, for a matrix $B \in \mathcal{S}$ with sign vector $\mathbf{t}$, the entries are:
    \[
        B_{kl} = t_k t_l t_1, \quad \text{for all } k,l.
    \]

    Matrices $A$ and $B$ commute if $AB = BA$. The product $AB$ has entries:
    \[
        (AB)_{ij} = \sum_{k=1}^n A_{ik} B_{kj} = s_i^A s_1^A t_j t_1 \sum_{k=1}^n s_k^A t_k,
    \]
    and the product $BA$ has entries:
    \[
        (BA)_{ij} = \sum_{k=1}^n B_{ik} A_{kj} = t_i t_1 s_j^A s_1^A \sum_{k=1}^n t_k s_k^A.
    \]

    Let $d = \sum_{k=1}^n s_k^A t_k$, the dot product of $\mathbf{s}^A$ and $\mathbf{t}$. Then:
    \[
        (AB)_{ij} = s_i^A s_1^A t_j t_1 d, \quad (BA)_{ij} = t_i t_1 s_j^A s_1^A d.
    \]

    Equating these gives:
    \[
        s_i^A s_1^A t_j t_1 d = t_i t_1 s_j^A s_1^A d, \quad \text{for all } i,j.
    \]

    This holds if either:
    \begin{itemize}
        \item $d = 0$, or
        \item $s_i^A t_j = t_i s_j^A$ for all $i,j$, which implies $\mathbf{t} = \pm \mathbf{s}^A$ (since setting $j=1$ gives $t_i = c s_i^A$ for $c = t_1 / s_1^A = \pm 1$).
    \end{itemize}

    Moreover, if $\mathbf{t} = \pm \mathbf{s}^A$, then $d = \sum s_k^A (\pm s_k^A) = \pm n = \pm 2025 \neq 0$, so the cases are disjoint.

    Since $\mathbf{s}^A$ and $\mathbf{t}$ are chosen independently and uniformly from $\{-1, +1\}^n$, the total number of pairs $(\mathbf{s}^A, \mathbf{t})$ is $(2^{2025})^2 = 2^{4050}$.

    The favorable pairs are those with $\mathbf{t} = \mathbf{s}^A$ or $\mathbf{t} = -\mathbf{s}^A$. For each $\mathbf{s}^A$, there are two choices for $\mathbf{t}$, so the number of favorable pairs is $2 \cdot 2^{2025} = 2^{2026}$.

    The dot product $d = 0$ is impossible when $n = 2025$ is odd, as the sum of an odd number of $\pm 1$ terms is odd and cannot be zero. Thus, this case contributes nothing.

    The probability is:
    \[
        \frac{2^{2026}}{2^{4050}} = 2^{2026 - 4050} = 2^{-2024} = \frac{1}{2^{2024}}.
    \]

    Each pair of sign vectors corresponds to a unique pair of matrices, so this is the probability that $A$ and $B$ commute.

    Final answer:
    \[
        \boxed{\frac{1}{2^{2024}}}
    \]

\section{Problem 4}
\subsection{Final solution (thinking only)}
        We are given that $a$ is an even positive integer and we need to find all real numbers $x$ such that for every positive integer $b$, the equation
        \[
            \left\lfloor \sqrt[a]{b^a + x} \cdot b^{a-1} \right\rfloor = b^a + \left\lfloor \frac{x}{a} \right\rfloor
        \]
        holds. The domain requires $b^a + x \geq 0$ for all positive integers $b$, and since the smallest $b$ is 1, we have $1 + x \geq 0$, so $x \geq -1$.

        After analysis, the values of $x$ that satisfy the equation for all $b$ are those where $x$ is an integer multiple of $a$, i.e., $x = ak$ for some integer $k$, but this fails for certain cases as shown. Instead, we find that $x$ must satisfy $0 \leq x < a$, but this is inconsistent with earlier examples.

        From the initial trials with small $a$ (like $a=2$) and various $x$, we observed that $x = -1$ and $x = 0$ work for $a=2$, and $x=0.5$ and $x=1.5$ also seem to work. However, a pattern emerges when considering the fractional part.

        Critical insight: Let $d = \left\lfloor \frac{x}{a} \right\rfloor$, so $x = ad + r$ with $0 \leq r < a$. The equation must hold for all $b$, and in particular for $b=1$, we have:
        \[
            \left\lfloor (1 + x)^{1/a} \right\rfloor = 1 + d
        \]
        which implies
        \[
            (1 + d)^a \leq 1 + x < (2 + d)^a.
        \]
        Substituting $x = ad + r$ gives
        \[
            (1 + d)^a \leq 1 + ad + r < (2 + d)^a.
        \]
        Since $0 \leq r < a$, this inequality constrains $r$ for each $d$. Moreover, the equation must hold for other $b$, leading to the condition that $r$ must be such that the expression remains consistent.

        However, further testing for $a=2$ showed that $x$ in intervals $[-1,0)$ and $[0,2)$ work, but not $[3,4)$ because for $x=3$, $b=1$: left side is $\lfloor \sqrt{4} \rfloor = 2$, right side is $1 + \lfloor 1.5 \rfloor = 1 + 1 = 2$, but for $b=2$, left side is $\lfloor 2 \cdot \sqrt{4+3} \rfloor = \lfloor 2\sqrt{7} \rfloor \approx \lfloor 2 \cdot 2.64575 \rfloor = \lfloor 5.2915 \rfloor = 5$, right side is $4 + \lfloor 3/2 \rfloor = 4 + 1 = 5$, and for $b=3$, left side $\lfloor 3 \cdot \sqrt{9+3} \rfloor = \lfloor 3\sqrt{12} \rfloor = \lfloor 6\sqrt{3} \rfloor \approx \lfloor 10.392 \rfloor = 10$, right side $9 + 1 = 10$. So $x=3$ works for $a=2$. Similarly, $x=-1,0,0.5,1,1.5,3$ all work for $a=2$.

        But earlier for $x=2$ with $a=2$, it failed at $b=1$. So what is common? For $a=2$, the values that work are $x \in [-1,0) \cup [0,2) \setminus \{2\}$, but 2 is excluded. However, $[0,2)$ includes numbers up to but not including 2, and $[-1,0)$ is separate. But $x=3$ is outside, and it worked. So perhaps for each $d$, $x$ in $[ad + c_d, ad + c_d')$ for some constants.

        From the $b=1$ condition: $(1+d)^a \leq 1+x < (2+d)^a$ and $ad \leq x < a(d+1)$. The intersection must be non-empty. For example, with $a=2$, $d=1$: $ad = 2$, $a(d+1)=4$, and from $b=1$: $(2)^2 = 4 \leq 1+x < (3)^2=9$, so $3 \leq x <8$, but combined with $2 \leq x <4$, so $3 \leq x <4$. Similarly for $d=0$: $0 \leq x <2$, and $1 \leq 1+x <4$, so $0 \leq x <3$, intersection $[0,2)$. For $d=-1$: $-2 \leq x <0$, and $0 \leq 1+x <1$, so $-1 \leq x <0$. For $d=1$, $x \in [3,4)$. We also need to verify for other $b$.

        When we tested for $a=2$ and $x=3$ (which is in $[3,4)$), it worked for $b=1,2,3$. Similarly, $x=3.5$: $b=1$: left $\lfloor \sqrt{1+3.5} \rfloor = \lfloor \sqrt{4.5} \rfloor \approx \lfloor 2.121 \rfloor =2$, right $1 + \lfloor 3.5/2 \rfloor =1 + \lfloor 1.75 \rfloor =1+1=2$. $b=2$: left $\lfloor 2 \sqrt{4+3.5} \rfloor = \lfloor 2\sqrt{7.5} \rfloor \approx \lfloor 2 \cdot 2.738 \rfloor = \lfloor 5.476 \rfloor =5$, right $4 + \lfloor 3.5/2 \rfloor =4+1=5$. $b=3$: left $\lfloor 3 \sqrt{9+3.5} \rfloor = \lfloor 3\sqrt{12.5} \rfloor \approx \lfloor 3 \cdot 3.5355 \rfloor = \lfloor 10.6065 \rfloor =10$, right $9 + \lfloor 3.5/2 \rfloor =9+1=10$. Good.

\section{Problem 5}
\subsection{Final solution}
    To prove that $g(n) < f(n) + n^{0.501}$ for sufficiently large $n$, we analyze the definitions of $f(n)$ and $g(n)$. Recall that $f(n) = \max_{\sigma \in S_n} \operatorname{ord}(\sigma)$, where $S_n$ is the set of permutations of $[n]$, and $\operatorname{ord}(\sigma)$ is the order of $\sigma$ as a group element, which is the least common multiple (LCM) of the lengths of its disjoint cycles. Similarly, $g(n) = \max_{\tau \in T_n} \operatorname{ord}(\tau)$, where $T_n$ is the set of all functions from $[n]$ to $[n]$, and $\operatorname{ord}(\tau)$ is the number of distinct maps in the set $\{\tau, \tau \circ \tau, \tau \circ \tau \circ \tau, \ldots\}$.

    \subsection*{Step 1: Bounding $g(n)$}
    For any function $\tau \in T_n$, its functional graph consists of components, each being a cycle with trees attached. Define:
    \begin{itemize}
        \item $d$ as the maximum transient length, i.e., the maximum over all points of the number of steps to reach a cycle.
        \item $p$ as the LCM of the lengths of all cycles in the functional graph.
    \end{itemize}

    The order $\operatorname{ord}(\tau)$ satisfies:
    \[
        \operatorname{ord}(\tau) \leq d + p.
    \]
    This bound holds because:
    \begin{itemize}
        \item The sequence $\tau^k$ for $k = 1, 2, \ldots, d$ corresponds to the transient phase and may consist of distinct functions.
        \item For $k \geq d$, $\tau^k$ maps all points to cycles, and the sequence becomes periodic with period $p$, contributing at most $p$ distinct functions.
    \end{itemize}

    \subsection*{Step 2: Relating $d$ and $p$ to the Graph Structure}
    Let $c$ be the sum of the lengths of all cycles in the functional graph, so $c \leq n$ is the number of points in cycles. The number of points not in cycles is $n - c$. Since a tree of height $h$ requires at least $h$ nodes (e.g., a path), the maximum transient length satisfies $d \leq n - c$. Moreover, $p$ is the maximum LCM over cycle lengths summing to $c$, so $p \leq f(c)$, where $f(c)$ is Landau's function for the maximum order of a permutation of $c$ elements.

    Thus:
    \[
        d + p \leq (n - c) + f(c).
    \]
    Since this holds for any $\tau$, maximizing over $\tau$ gives:
    \[
        g(n) \leq \max_{c=1}^n \left( n - c + f(c) \right),
    \]
    where $c$ ranges over the sum of cycle lengths in possible functional graphs.

    \subsection*{Step 3: Analyzing the Expression $\max_c (n - c + f(c))$}
    We need to show that:
    \[
        \max_c (n - c + f(c)) < f(n) + n^{0.501}
    \]
    for sufficiently large $n$. Equivalently, we consider:
    \[
        \max_c \left( (n - c + f(c)) - f(n) \right) < n^{0.501}.
    \]
    Split the maximization over $c$ into two cases:

    \textbf{Case 1:} $c > n - n^{0.501}$

    Here, $n - c < n^{0.501}$. Since $f(c) \leq f(n)$ (as $f$ is increasing), we have:
    \[
        n - c + f(c) - f(n) \leq n - c + 0 < n^{0.501}.
    \]

    \textbf{Case 2:} $c \leq n - n^{0.501}$

    Set $m = n - c \geq n^{0.501}$. Then:
    \[
        n - c + f(c) - f(n) = m + f(n - m) - f(n).
    \]
    We show that this is less than $n^{0.501}$ for large $n$.

    Landau's function satisfies $f(n) = \exp\left((1 + o(1)) \sqrt{n \log n}\right)$ as $n \to \infty$. Thus, for fixed $\epsilon > 0$ and large $n$:
    \[
        f(n) \geq \exp\left((1 - \epsilon) \sqrt{n \log n}\right).
    \]
    For $f(n - m)$ with $m \geq n^{0.501}$:
    \[
        f(n - m) \leq \exp\left((1 + \epsilon) \sqrt{(n - m) \log (n - m)}\right).
    \]
    Asymptotically:
    \[
        \sqrt{(n - m) \log (n - m)} = \sqrt{n \log n} - \Theta\left(n^{0.001} \sqrt{\log n}\right)
    \]
    for $m = n^{0.501}$, so:
    \[
        f(n - m) \leq \exp\left((1 + \epsilon) \sqrt{n \log n} - \Theta\left(n^{0.001} \sqrt{\log n}\right)\right).
    \]
    Now:
    \[
        m + f(n - m) - f(n) \leq m - \left( f(n) - f(n - m) \right).
    \]
    For large $n$, $f(n) - f(n - m) \geq \frac{1}{2} f(n)$ because:
    \begin{itemize}
        \item $f(n) \gg f(n - m)$, and
        \item $\frac{f(n)}{f(n - m)} \to \infty$ since the exponent difference $\Theta\left(n^{0.001} \sqrt{\log n}\right) \to \infty$.
    \end{itemize}

    Moreover, $\frac{1}{2} f(n) > m$ for large $n$ because $f(n) / n^{0.501} \to \infty$. Thus:
    \[
        m - \left( f(n) - f(n - m) \right) \leq m - \frac{1}{2} f(n) < 0 < n^{0.501}.
    \]
    Therefore, for $c \leq n - n^{0.501}$:
    \[
        m + f(n - m) - f(n) < n^{0.501}.
    \]

    \subsection*{Step 4: Conclusion}
    In both cases, for sufficiently large $n$:
    \[
        \max_c \left( n - c + f(c) \right) - f(n) < n^{0.501}.
    \]
    Hence:
    \[
        g(n) \leq \max_c \left( n - c + f(c) \right) < f(n) + n^{0.501},
    \]
    which completes the proof for sufficiently large $n$.

    \textbf{Final answer.}
    The inequality $g(n) < f(n) + n^{0.501}$ holds for sufficiently large $n$.

\section{Problem 6}
\subsection{Variant 1}
    Consider the function $p(x) = f(x) - k$. Since $f(a) = f(b) = k$, it follows that $p(a) = f(a) - k = 0$ and $p(b) = f(b) - k = 0$. The goal is to find $\xi \in (a, b)$ such that $f(\xi) - \xi f'(\xi) = k$, which is equivalent to $p(\xi) - \xi p'(\xi) = 0$ because $p'(x) = f'(x)$.

    Define the function $q(x) = \frac{p(x)}{x}$. Since $a > 0$ and $b > a$, the interval $[a, b]$ is away from zero, and $f$ is continuously differentiable on $(0, \infty)$, so $q$ is continuously differentiable on $[a, b]$. Evaluate $q$ at the endpoints:
    \[
        q(a) = \frac{p(a)}{a} = \frac{0}{a} = 0, \quad q(b) = \frac{p(b)}{b} = \frac{0}{b} = 0.
    \]
    Thus, $q(a) = q(b) = 0$.

    By Rolle's theorem, since $q$ is continuous on $[a, b]$, differentiable on $(a, b)$, and $q(a) = q(b)$, there exists $\xi \in (a, b)$ such that $q'(\xi) = 0$.

    The derivative of $q(x)$ is:
    \[
        q'(x) = \frac{d}{dx} \left( \frac{p(x)}{x} \right) = \frac{x p'(x) - p(x)}{x^2}.
    \]
    Setting $q'(\xi) = 0$ gives:
    \[
        \frac{\xi p'(\xi) - p(\xi)}{\xi^2} = 0,
    \]
    which implies $\xi p'(\xi) - p(\xi) = 0$, or equivalently $p(\xi) = \xi p'(\xi)$.

    Substituting back $p(\xi) = f(\xi) - k$ and $p'(\xi) = f'(\xi)$:
    \[
        f(\xi) - k = \xi f'(\xi),
    \]
    so:
    \[
        f(\xi) - \xi f'(\xi) = k.
    \]
    Thus, there exists $\xi \in (a, b)$ satisfying the equation.

    \textbf{Final answer:} The proof is complete, showing that such a $\xi$ exists.
\subsection{Variant 2}
    To prove the statement, consider the function $g(x) = f(x) - k$. Since $f(a) = f(b) = k$, it follows that $g(a) = 0$ and $g(b) = 0$. The function $g$ is continuously differentiable on $[a, b]$ because $f$ is continuously differentiable.

    Next, define the auxiliary function $h(x) = \frac{g(x)}{x}$. Since $a > 0$, $x \neq 0$ on $[a, b]$, and $g$ is continuously differentiable, $h$ is also continuously differentiable on $[a, b]$. Evaluating $h$ at the endpoints:
    \[
        h(a) = \frac{g(a)}{a} = \frac{0}{a} = 0, \quad h(b) = \frac{g(b)}{b} = \frac{0}{b} = 0.
    \]
    Thus, $h(a) = h(b) = 0$. By Rolle's Theorem, since $h$ is continuous on $[a, b]$ and differentiable on $(a, b)$, there exists a point $\xi \in (a, b)$ such that $h'(\xi) = 0$.

    The derivative of $h$ is computed as follows:
    \[
        h'(x) = \frac{d}{dx} \left( \frac{g(x)}{x} \right) = \frac{x g'(x) - g(x)}{x^2}.
    \]
    Setting $h'(\xi) = 0$:
    \[
        \frac{\xi g'(\xi) - g(\xi)}{\xi^2} = 0.
    \]
    Since $\xi \neq 0$, this implies:
    \[
        \xi g'(\xi) - g(\xi) = 0.
    \]
    Rearranging, we have:
    \[
        g(\xi) = \xi g'(\xi).
    \]
    Substituting $g(x) = f(x) - k$ and $g'(x) = f'(x)$:
    \[
        f(\xi) - k = \xi f'(\xi),
    \]
    which simplifies to:
    \[
        f(\xi) - \xi f'(\xi) = k.
    \]
    Therefore, there exists $\xi \in (a, b)$ satisfying the required equation.

    \boxed{\text{proved}}
\subsection{Final Solution}
    \subsection*{Review of Solution Drafts for Problem 6}

    All three solution drafts correctly prove the statement. The core logic is identical across all drafts and relies on the following steps:
    \begin{enumerate}
        \item \textbf{Define $h(x) = f(x) - k$} (or equivalent notation), so $h(a) = h(b) = 0$.
        \item \textbf{Define a quotient function} $\phi(x) = \frac{h(x)}{x}$. This is continuously differentiable on $[a, b]$ because:
        \begin{itemize}
            \item $h$ is continuously differentiable (since $f$ is).
            \item $x \neq 0$ on $[a, b]$ (as $a > 0$).
            \item The quotient of continuously differentiable functions (with non-vanishing denominator) is continuously differentiable.
        \end{itemize}
        \item \textbf{Evaluate endpoints}: $\phi(a) = \frac{h(a)}{a} = 0$ and $\phi(b) = \frac{h(b)}{b} = 0$, so $\phi(a) = \phi(b) = 0$.
        \item \textbf{Apply Rolle's Theorem}: Since $\phi$ is continuous on $[a, b]$, differentiable on $(a, b)$, and $\phi(a) = \phi(b)$, there exists $\xi \in (a, b)$ such that $\phi'(\xi) = 0$.
        \item \textbf{Compute the derivative}:
        \[
            \phi'(x) = \frac{x h'(x) - h(x)}{x^2}.
        \]
        Setting $\phi'(\xi) = 0$ gives:
        \[
            \xi h'(\xi) - h(\xi) = 0 \implies h(\xi) = \xi h'(\xi).
        \]
        \item \textbf{Substitute back}: $h(\xi) = f(\xi) - k$ and $h'(\xi) = f'(\xi)$, so:
        \[
            f(\xi) - k = \xi f'(\xi) \implies f(\xi) - \xi f'(\xi) = k.
        \]
    \end{enumerate}

    \textbf{Strengths}:
    \begin{itemize}
        \item All drafts use rigorous calculus (Rolle's Theorem) and correctly handle differentiability.
        \item Notation is mathematically sound, and the reasoning is complete.
        \item No logical errors or gaps.
    \end{itemize}

    \textbf{Areas for improvement}:
    \begin{itemize}
        \item \textbf{Solution 1} uses $\varphi$ for the quotient function and $h$ for $f - k$, but later uses $\phi$ inconsistently. Notation should be consistent.
        \item \textbf{Solution 2} has the clearest notation ($p = f - k$, $q = p/x$) and is the most concise.
        \item \textbf{Solution 3} uses $g = f - k$ and $h = g/x$, but reuses $h$ (which was used for $f - k$ in Solution 1), which could cause confusion.
    \end{itemize}

    \textbf{Final Solution (Refined and Concise)}
    Using the clearest notation (from Solution 2):

    \hrule

    \textbf{Problem:}
    Let $f: (0, \infty) \to \mathbb{R}$ be continuously differentiable, and let $b > a > 0$ such that $f(a) = f(b) = k$. Prove there exists $\xi \in (a, b)$ with $f(\xi) - \xi f'(\xi) = k$.

    \textbf{Proof:}
    Define $p(x) = f(x) - k$. Then $p(a) = 0$ and $p(b) = 0$. As $f$ is continuously differentiable, $p$ is continuously differentiable on $[a, b]$. Define $q(x) = \frac{p(x)}{x}$. Since $a > 0$, $x \neq 0$ on $[a, b]$, and $p$ is continuously differentiable, $q$ is continuously differentiable on $[a, b]$. Now,
    \[
        q(a) = \frac{p(a)}{a} = \frac{0}{a} = 0, \quad q(b) = \frac{p(b)}{b} = \frac{0}{b} = 0,
    \]
    so $q(a) = q(b) = 0$. By Rolle's Theorem, there exists $\xi \in (a, b)$ such that $q'(\xi) = 0$. The derivative of $q$ is:
    \[
        q'(x) = \frac{x p'(x) - p(x)}{x^2}.
    \]
    Thus,
    \[
        q'(\xi) = 0 \implies \frac{\xi p'(\xi) - p(\xi)}{\xi^2} = 0 \implies \xi p'(\xi) - p(\xi) = 0,
    \]
    since $\xi \neq 0$. This simplifies to $p(\xi) = \xi p'(\xi)$. Substituting $p(\xi) = f(\xi) - k$ and $p'(\xi) = f'(\xi)$:
    \[
        f(\xi) - k = \xi f'(\xi) \implies f(\xi) - \xi f'(\xi) = k.
    \]
    Therefore, such a $\xi$ exists in $(a, b)$.

    \boxed{\text{proved}}

\section{Problem 7}
\subsection{Final Solution}
    The nonempty subsets $M \subseteq \mathbb{Z}_{>0}$ satisfying properties (a) and (b) are characterized as follows. After careful analysis, the solution sets fall into two distinct categories, each satisfying both properties.

    \subsection*{Final Answer}
    The solution sets $M$ are precisely:
    \begin{enumerate}
        \item \textbf{All positive multiples of a fixed odd positive integer $d$}, i.e., $M = d\mathbb{Z}_{>0}$ for some odd $d \in \mathbb{Z}_{>0}$.
        \begin{itemize}
            \item For example:
            \begin{itemize}
                \item If $d = 1$, $M = \mathbb{Z}_{>0}$ (all positive integers).
                \item If $d = 3$, $M = \{3, 6, 9, 12, \ldots\}$.
            \end{itemize}
        \end{itemize}
        \item \textbf{All integers greater than or equal to a fixed positive integer $m$}, i.e., $M = \{n \in \mathbb{Z}_{>0} \mid n \geq m\}$ for some $m \in \mathbb{Z}_{>0}$.
        \begin{itemize}
            \item For example:
            \begin{itemize}
                \item If $m = 1$, $M = \mathbb{Z}_{>0}$ (all positive integers).
                \item If $m = 2$, $M = \{2, 3, 4, 5, \ldots\}$.
            \end{itemize}
        \end{itemize}
    \end{enumerate}

    These are the only sets that satisfy both properties. Note that when $d = 1$ in the first category or $m = 1$ in the second category, both yield the full set of positive integers. Sets not of these forms, such as the powers of 2 or subsets missing certain elements, fail to satisfy property (b) or (a).

    \textbf{Proof Outline}:
    \begin{itemize}
        \item \textbf{Property (a)}: For $M = d\mathbb{Z}_{>0}$ with $d$ odd, if $x = kd$, then $2x = 2kd \in M$. For $M = \{n \geq m\}$, if $x \geq m$, then $2x \geq 2m \geq m$, so $2x \in M$.
        \item \textbf{Property (b)}: For $M = d\mathbb{Z}_{>0}$ with $d$ odd, if $x = ad$, $y = bd$, and $x+y$ even, then $a + b$ is even (as $d$ is odd), so $\frac{x+y}{2} = \frac{(a+b)d}{2} = cd \in M$ for integer $c$. For $M = \{n \geq m\}$, if $x, y \geq m$ and $x+y$ even, then $\frac{x+y}{2} \geq m$ and is an integer, so $\frac{x+y}{2} \in M$.
        \item \textbf{Exhaustiveness}: Any $M$ must have a minimum element $m$. If $m$ is odd, $M$ must be all multiples of $m$ (and $m$ is odd). If $m$ is even, $M$ must be all integers $\geq m$. Other sets lead to contradictions with properties (a) or (b).
    \end{itemize}

\section{Problem 8}
\subsection{Final Solution}
    The solution drafts contain errors in logic and incomplete reasoning. Below is a corrected, concise, and rigorous proof.

    \subsection*{Problem Statement}
    For an $n \times n$ real matrix $A$, define $A^R$ as its counter-clockwise $90^{\circ}$ rotation, given by $A^R = J_n A^T$, where $J_n$ is the exchange matrix (symmetric with $J_n^2 = I_n$). Prove that if $A = A^R$, then for any eigenvalue $\lambda$ of $A$, $\operatorname{Re} \lambda = 0$ or $\operatorname{Im} \lambda = 0$.

    \subsection*{Proof}
    Given $A = A^R = J_n A^T$, where $J_n$ is real symmetric and satisfies $J_n^2 = I_n$. Transposing both sides yields:
    \[
        A^T = (J_n A^T)^T = A J_n,
    \]
    since $J_n^T = J_n$. Substituting $A = J_n A^T$ into the right side:
    \[
        A^T = J_n (J_n A^T) = (J_n^2) A^T = I_n A^T = A^T,
    \]
    which is consistent. From $A = J_n A^T$ and $A^T = A J_n$, we have:
    \[
        J_n A^T = A \quad \text{and} \quad A^T = J_n A.
    \]
    This implies:
    \[
        A J_n = J_n A,
    \]
    so $A$ commutes with $J_n$. Since $J_n$ is diagonalizable (as a real symmetric matrix), and $A$ commutes with $J_n$, they share a common eigenvector for each eigenvalue. Specifically, for any eigenvalue $\lambda$ of $A$, there exists an eigenvector $w \neq 0$ such that:
    \[
        A w = \lambda w \quad \text{and} \quad J_n w = \mu w,
    \]
    where $\mu = \pm 1$ (since the eigenvalues of $J_n$ are $\pm 1$).

    From $A^T = J_n A$, we compute $A^T w$:
    \[
        A^T w = (J_n A) w = J_n (\lambda w) = \lambda (J_n w) = \lambda \mu w.
    \]
    Consider the quadratic form $w^* A w$. On one hand:
    \[
        w^* A w = w^* (\lambda w) = \lambda (w^* w).
    \]
    On the other hand, since $A$ is real, $A^* = A^T$, and:
    \[
        w^* A^T w = (A^T w)^* w = (\lambda \mu w)^* w = \overline{\lambda \mu} (w^* w) = \bar{\lambda} \mu (w^* w),
    \]
    as $\mu$ is real. Also directly:
    \[
        w^* A^T w = w^* (\lambda \mu w) = \lambda \mu (w^* w).
    \]
    Thus:
    \[
        \lambda \mu (w^* w) = \bar{\lambda} \mu (w^* w).
    \]
    Let $d = w^* w > 0$ (since $w \neq 0$). Then:
    \[
        \lambda \mu d = \bar{\lambda} \mu d \implies \lambda \mu = \bar{\lambda} \mu,
    \]
    as $d \neq 0$. Since $\mu = \pm 1$, we divide by $\mu$:
    \[
        \lambda = \bar{\lambda} \quad \text{if} \quad \mu = 1, \qquad \text{and} \quad \lambda = -\bar{\lambda} \quad \text{if} \quad \mu = -1.
    \]
    \begin{itemize}
        \item If $\mu = 1$, then $\lambda = \bar{\lambda}$, so $\lambda$ is real ($\operatorname{Im} \lambda = 0$).
        \item If $\mu = -1$, then $\lambda = -\bar{\lambda}$, so $\operatorname{Re} \lambda = 0$ (purely imaginary).
    \end{itemize}

    Thus, for any eigenvalue $\lambda$, $\operatorname{Re} \lambda = 0$ or $\operatorname{Im} \lambda = 0$.

    \[\boxed{\text{proved}}\]

\section{Problem 9}
\subsection{Final Solution}
    The random process selects $n$ distinct positive integers sequentially. The first number $X_1$ is chosen with $\mathbb{P}(X_1 = i) = 2^{-i}$ for each positive integer $i$. For $1 \leq j \leq n-1$, after selecting $X_1, \ldots, X_j$, the remaining positive integers are ordered increasingly as $n_1 < n_2 < \cdots$, and $X_{j+1}$ is chosen with $\mathbb{P}(X_{j+1} = n_i) = 2^{-i}$.

    Let $Y_n = \max\{X_1, \ldots, X_n\}$. To find $\mathbb{E}[Y_n]$, use the tail sum formula for expectation of a non-negative integer-valued random variable:
    \begin{equation}
        \mathbb{E}[Y_n] = \sum_{k=1}^{\infty} \mathbb{P}(Y_n \geq k).
    \end{equation}

    Here, $\mathbb{P}(Y_n \geq k)$ is the probability that at least one $X_i$ is at least $k$, which equals $1 - \mathbb{P}(Y_n < k)$, where $\mathbb{P}(Y_n < k)$ is the probability that all selected numbers are in $\{1, 2, \ldots, k-1\}$. Define $S = k-1$, so:
    \begin{equation}
        \mathbb{P}(Y_n < k) = \mathbb{P}(\text{all } X_i \in \{1, 2, \ldots, S\}).
    \end{equation}

    This probability is zero if $S < n$ (i.e., $k \leq n$), as fewer than $n$ numbers exist in the set. For $S \geq n$ (i.e., $k > n$), the probability is computed via the selection process. The key observation is that if the first $j$ selections are in $\{1, 2, \ldots, S\}$, the remaining numbers in $\{1, 2, \ldots, S\} \setminus \{X_1, \ldots, X_j\}$ are the smallest $|S| - j$ numbers in the remaining ordered sequence.

\end{document}
