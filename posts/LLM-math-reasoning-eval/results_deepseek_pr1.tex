\documentclass[12pt,a4paper]{article}
\usepackage[utf8]{inputenc}
\usepackage[T1]{fontenc}
\usepackage{amsmath,amsfonts,amssymb,amsthm}
\usepackage{geometry}
\usepackage{fancyhdr}
\usepackage{enumerate}
\usepackage{tikz}
\usepackage{pgfplots}
\usepackage{hyperref}
\usepackage{xcolor}

\geometry{margin=2.5cm}
\pagestyle{fancy}
\fancyhf{}
\fancyhead[C]{Mathematical Problems Collection}
\fancyfoot[C]{\thepage}

\title{Mathematical Problems Collection}
\author{Generated from JSON Data}
\date{\today}

\theoremstyle{definition}
\newtheorem{problem}{Problem}
\newtheorem{solution}{Solution}

\begin{document}

    \maketitle
    \tableofcontents
    \newpage
\section{Problem 1}

\subsection{Solution to Problem 1, variant 1}

\subsection*{Part (a)}

Let $P$ be a polynomial of odd degree $n \geq 3$ (since $\deg(P) \geq 2$ and odd). For any point $(X, Y) \in \mathbb{R}^2$, consider the equation that must hold for $(X, Y)$ to lie on the tangent line $\ell_x$ at $(x, P(x))$:

\[
Y = P(x) + P'(x)(X - x).
\]

Rearrange this equation as:

\[
P(x) + P'(x)(X - x) - Y = 0.
\]

Define the function $g(x) = P(x) + P'(x)(X - x) - Y$. Since $P$ is a polynomial, $g(x)$ is also a polynomial in $x$. Let $P(x) = a_n x^n + a_{n-1} x^{n-1} + \cdots + a_0$ with $a_n \neq 0$. Then $P'(x) = n a_n x^{n-1} + (n-1) a_{n-1} x^{n-2} + \cdots$. The leading term of $g(x)$ comes from $P(x)$ and $-x P'(x)$:

\begin{itemize}
\item $P(x)$ contributes $a_n x^n$.
\item $-x P'(x)$ contributes $-x \cdot (n a_n x^{n-1}) = -n a_n x^n$.
\end{itemize}

Thus, the leading term of $g(x)$ is $a_n x^n - n a_n x^n = a_n (1 - n) x^n$. Since $n \geq 2$, $1 - n \neq 0$ and $a_n \neq 0$, so $\deg(g) = n$, which is odd.

A polynomial of odd degree has at least one real root because it tends to $+\infty$ as $x \to +\infty$ and to $-\infty$ as $x \to -\infty$ (or vice-versa, depending on the leading coefficient), and by the intermediate value theorem, it must cross zero. Therefore, there exists a real number $x$ such that $g(x) = 0$, meaning that $(X, Y)$ lies on the tangent line $\ell_x$.

This holds for every $(X, Y) \in \mathbb{R}^2$, so $\bigcup_{x \in \mathbb{R}} \ell_x = \mathbb{R}^2$.

\subsection*{Solution to Part (b)}

No, there does not exist a polynomial of even degree for which $\bigcup_{x \in \mathbb{R}} \ell_x = \mathbb{R}^2$.

To see this, suppose $P$ has even degree $n \geq 2$. For a fixed $X \in \mathbb{R}$, define the function $h_X(x) = P(x) + P'(x)(X - x)$. As in part (a), $h_X(x)$ is a polynomial in $x$ of degree $n$, even, with leading coefficient $a_n (1 - n) \neq 0$ since $n \geq 2$.

Because the degree is even:
\begin{itemize}
\item If the leading coefficient is positive, then $h_X(x) \to +\infty$ as $|x| \to \infty$, so $h_X(x)$ has a global minimum $m_X$, and its range is $[m_X, \infty)$.
\item If the leading coefficient is negative, then $h_X(x) \to -\infty$ as $|x| \to \infty$, so $h_X(x)$ has a global maximum $M_X$, and its range is $(-\infty, M_X]$.
\end{itemize}

In both cases, the range of $h_X(x)$ is not all of $\mathbb{R}$. For example:
\begin{itemize}
\item If the range is $[m_X, \infty)$, then for any $Y < m_X$, there is no $x$ such that $h_X(x) = Y$.
\item If the range is $(-\infty, M_X]$, then for any $Y > M_X$, there is no $x$ such that $h_X(x) = Y$.
\end{itemize}

This means that for each $X$, there exist values of $Y$ such that the point $(X, Y)$ does not lie on any tangent line $\ell_x$. Therefore, $\bigcup_{x \in \mathbb{R}} \ell_x \neq \mathbb{R}^2$.

Since this argument holds for any even degree polynomial, no such polynomial exists that satisfies the equality.

Final answer for part (b): No.

\subsection{Solution to Problem 1, variant 2}

\subsection*{Part (a)}
Let $P$ be a polynomial of odd degree $n \geq 2$ with real coefficients. For any point $(a, b) \in \mathbb{R}^2$, we need to show that there exists some $x \in \mathbb{R}$ such that the tangent line $\ell_x$ to the graph of $P$ at $(x, P(x))$ passes through $(a, b)$.

The equation of the tangent line at $x$ is given by:
\[
Y - P(x) = P'(x)(X - x).
\]
Substituting $(X, Y) = (a, b)$ yields:
\[
b - P(x) = P'(x)(a - x),
\]
which can be rearranged as:
\[
b = P(x) + P'(x)(a - x).
\]
Define the function $g_a : \mathbb{R} \to \mathbb{R}$ by:
\[
g_a(x) = P(x) + P'(x)(a - x).
\]
Since $P$ is a polynomial of degree $n$, $P'$ is a polynomial of degree $n-1 \geq 1$ (as $n \geq 2$). The expression for $g_a(x)$ is:
\[
g_a(x) = P(x) + a P'(x) - x P'(x).
\]
The term of highest degree in $g_a(x)$ arises from $P(x)$ and $-x P'(x)$. Let $P(x) = c_n x^n + c_{n-1} x^{n-1} + \cdots + c_0$ with $c_n \neq 0$. Then:
\[
P'(x) = n c_n x^{n-1} + (n-1) c_{n-1} x^{n-2} + \cdots,
\]
\[
x P'(x) = n c_n x^n + (n-1) c_{n-1} x^{n-1} + \cdots.
\]
The leading terms in $g_a(x)$ are:
\[
P(x) \text{ contributes } c_n x^n, \quad -x P'(x) \text{ contributes } -n c_n x^n,
\]
so the coefficient of $x^n$ in $g_a(x)$ is:
\[
c_n - n c_n = c_n (1 - n).
\]
Since $n \geq 2$ and odd, $1 - n \neq 0$ and $c_n \neq 0$, so $g_a(x)$ is a polynomial of degree $n$. As $n$ is odd, $g_a(x)$ is an odd-degree polynomial. An odd-degree polynomial is surjective from $\mathbb{R}$ to $\mathbb{R}$ because $\lim_{x \to \infty} g_a(x) = \pm \infty$ and $\lim_{x \to -\infty} g_a(x) = \mp \infty$ (depending on the sign of the leading coefficient), and by the intermediate value theorem, it takes all real values.

Therefore, for any fixed $a$ and any $b \in \mathbb{R}$, there exists $x \in \mathbb{R}$ such that $g_a(x) = b$, meaning:
\[
b = P(x) + P'(x)(a - x).
\]
This implies that the tangent line at $x$ passes through $(a, b)$. Hence, for every $(a, b) \in \mathbb{R}^2$, there is a tangent line containing it, and:
\[
\bigcup_{x \in \mathbb{R}} \ell_x = \mathbb{R}^2.
\]

\subsection*{Part (b)}
We need to determine if there exists a polynomial $P$ of even degree at least 2 such that $\bigcup_{x \in \mathbb{R}} \ell_x = \mathbb{R}^2$.

Suppose $P$ has even degree $n \geq 2$. For any fixed $a \in \mathbb{R}$, define $g_a(x) = P(x) + P'(x)(a - x)$. As derived in part (a), $g_a(x)$ is a polynomial of degree $n$ (since the leading coefficient $c_n (1 - n) \neq 0$ for $n \geq 2$).

A non-constant polynomial of even degree is not surjective from $\mathbb{R}$ to $\mathbb{R}$. If the leading coefficient is positive, then $\lim_{|x| \to \infty} g_a(x) = +\infty$, so $g_a(x)$ has a global minimum and does not take values less than this minimum. If the leading coefficient is negative, then $\lim_{|x| \to \infty} g_a(x) = -\infty$, so $g_a(x)$ has a global maximum and does not take values greater than this maximum. In either case, there exists some $b \in \mathbb{R}$ such that $g_a(x) \neq b$ for all $x \in \mathbb{R}$.

This means that for each $a \in \mathbb{R}$, there is a $b \in \mathbb{R}$ such that the point $(a, b)$ is not on any tangent line $\ell_x$. Therefore, $\bigcup_{x \in \mathbb{R}} \ell_x \neq \mathbb{R}^2$ for any even-degree polynomial of degree at least 2.

Thus, no such polynomial exists.

    \subsection{Solution for Problem 1, variant 3}

    \subsubsection*{Part (a)}
    Let $P \in \mathbb{R}[x]$ be a polynomial of odd degree $n \geq 2$ with real coefficients. Fix an arbitrary point $(X, Y) \in \mathbb{R}^2$. The goal is to show that $(X, Y)$ lies on the tangent line to the graph of $P$ at some point $(a, P(a))$ for some $a \in \mathbb{R}$.

    The tangent line to the graph of $P$ at $x = a$ has the equation:
    \[
        y - P(a) = P'(a)(x - a).
    \]
    The point $(X, Y)$ lies on this line if and only if:
    \[
        Y - P(a) = P'(a)(X - a),
    \]
    which rearranges to:
    \[
        Y = P(a) + P'(a)(X - a).
    \]
    Define the function $g: \mathbb{R} \to \mathbb{R}$ by:
    \[
        g(a) = P(a) + P'(a)(X - a).
    \]
    This is a polynomial in $a$ for fixed $X$. Let $P(a) = c_n a^n + c_{n-1} a^{n-1} + \cdots + c_0$ with $c_n \neq 0$. The derivative is $P'(a) = n c_n a^{n-1} + (n-1) c_{n-1} a^{n-2} + \cdots + c_1$. Then:
    \[
        g(a) = P(a) + P'(a)(X - a) = P(a) + X P'(a) - a P'(a).
    \]
    The leading term of $g(a)$ comes from the highest-degree terms in $P(a)$ and $-a P'(a)$:
    \begin{itemize}
        \item $P(a)$ has leading term $c_n a^n$.
        \item $-a P'(a)$ has leading term $-a \cdot n c_n a^{n-1} = -n c_n a^n$.
    \end{itemize}
    Combining these, the coefficient of $a^n$ is $c_n - n c_n = (1 - n) c_n$.

    Since $\deg(P) = n \geq 2$ and $n$ is odd, $n \geq 3$. Thus, $1 - n \neq 0$ and $c_n \neq 0$, so $\deg(g) = n$, which is odd. The leading coefficient is $(1 - n) c_n$, which is nonzero.

    A real polynomial of odd degree is surjective onto $\mathbb{R}$ by the intermediate value theorem because as $a \to \infty$, $g(a) \to \infty$ if the leading coefficient is positive, or $g(a) \to -\infty$ if the leading coefficient is negative, and similarly in the opposite direction as $a \to -\infty$. Thus, $g(a)$ takes every real value as $a$ varies over $\mathbb{R}$.

    In particular, for the fixed $X$ and any $Y$, there exists some $a \in \mathbb{R}$ such that $g(a) = Y$. This means:
    \[
        Y = P(a) + P'(a)(X - a),
    \]
    so $(X, Y)$ lies on the tangent line at $(a, P(a))$.

    Since $(X, Y)$ was arbitrary, every point in $\mathbb{R}^2$ lies on some tangent line to the graph of $P$. Therefore:
    \[
        \bigcup_{x \in \mathbb{R}} \ell_x = \mathbb{R}^2.
    \]

    \subsubsection*{Part (b)}
    No, there does not exist a polynomial of even degree at least 2 for which $\bigcup_{x \in \mathbb{R}} \ell_x = \mathbb{R}^2$.

    Suppose $P$ has even degree $n \geq 2$. For a fixed $X \in \mathbb{R}$, define $g(a) = P(a) + P'(a)(X - a)$ as in part (a). As before, $g(a)$ is a polynomial in $a$ of degree $n$ (even), with leading coefficient $(1 - n) c_n \neq 0$ (since $n \geq 2$ and $c_n \neq 0$).

    A real polynomial of even degree

    \subsection{Final Solution to the Problem 1}

    \subsubsection*{Part (a)}
    Let $P$ be a polynomial of odd degree $n \geq 2$ with real coefficients. Fix an arbitrary point $(X, Y) \in \mathbb{R}^2$. The tangent line to the graph of $P$ at a point $(a, P(a))$ has the equation:
    \[
        y - P(a) = P'(a)(x - a).
    \]
    The point $(X, Y)$ lies on this tangent line if and only if:
    \[
        Y = P(a) + P'(a)(X - a).
    \]
    Define the function $g: \mathbb{R} \to \mathbb{R}$ by:
    \[
        g(a) = P(a) + P'(a)(X - a) - Y.
    \]
    This is a polynomial in $a$ for fixed $X$ and $Y$. Let $P(a) = c_n a^n + c_{n-1} a^{n-1} + \cdots + c_0$ with $c_n \neq 0$. The derivative is $P'(a) = n c_n a^{n-1} + (n-1) c_{n-1} a^{n-2} + \cdots$. Substituting into $g(a)$:
    \begin{itemize}
        \item $P(a)$ has leading term $c_n a^n$.
        \item $-a P'(a)$ has leading term $-a \cdot n c_n a^{n-1} = -n c_n a^n$.
    \end{itemize}
    The leading term of $g(a)$ is $c_n a^n - n c_n a^n = (1 - n) c_n a^n$.

    Since $\deg(P) = n \geq 2$ and $n$ is odd, $n \geq 3$. Thus, $1 - n \neq 0$ and $c_n \neq 0$, so $\deg(g) = n$, which is odd. The leading coefficient is $(1 - n) c_n$, which is nonzero.

    A real polynomial of odd degree takes all real values (it is surjective) by the intermediate value theorem. Therefore, there exists $a \in \mathbb{R}$ such that $g(a) = 0$, meaning:
    \[
        Y = P(a) + P'(a)(X - a).
    \]
    Thus, $(X, Y)$ lies on the tangent line at $(a, P(a))$. Since $(X, Y)$ was arbitrary, every point in $\mathbb{R}^2$ lies on some tangent line, so:
    \[
        \bigcup_{x \in \mathbb{R}} \ell_x = \mathbb{R}^2.
    \]

    \subsubsection*{Part (b)}
    No, there does not exist a polynomial of even degree at least 2 for which the union of all tangent lines covers $\mathbb{R}^2$.

    Suppose $P$ has even degree $n \geq 2$. For a fixed $X \in \mathbb{R}$, define the polynomial in $a$:
    \[
        h_X(a) = P(a) + P'(a)(X - a).
    \]
    As in part (a), $h_X(a)$ is a polynomial of degree $n$ (even), with leading coefficient $(1 - n) c_n \neq 0$. A real polynomial of even degree is not surjective: if the leading coefficient is positive, $h_X(a) \to +\infty$ as $|a| \to \infty$, so it has a global minimum and misses all values below it; if negative, it has a global maximum and misses all values above it. Thus, for each $X$, there exists some $Y$ (e.g., below the global minimum or above the global maximum) such that $h_X(a) \neq Y$ for all $a$, meaning:
    \[
        Y \neq P(a) + P'(a)(X - a)
    \]
    for all $a$. Therefore, the point $(X, Y)$ does not lie on any tangent line. Since this holds for every even-degree polynomial, no such polynomial exists.

    \textbf{Final answer for (b):} No.
    \[
        \boxed{\text{no}}
    \]

\section{Problem 2}
    The problem requires proving that $\int_{-1}^{1} [f''(x)]^2 \, dx \geq 15$ for any twice continuously differentiable function $f: \mathbb{R} \to \mathbb{R}$ satisfying $\int_{-1}^{1} f(x) \, dx = 0$ and $f(1) = f(-1) = 1$, and finding all functions achieving equality.

    \subsection*{Proof of the Inequality}
    To solve this, consider the minimization of the functional $\int_{-1}^{1} [f''(x)]^2 \, dx$ subject to the constraints. Using the calculus of variations with the integral constraint $\int_{-1}^{1} f(x) \, dx = 0$ and boundary conditions $f(-1) = f(1) = 1$, the Euler-Lagrange equation yields $f^{(4)}(x) = \lambda$ for some constant $\lambda$. Thus, the minimizer is a quartic polynomial:
    \[
        f(x) = a x^4 + b x^3 + c x^2 + d x + e.
    \]
    The constraints are:
    \begin{enumerate}
        \item $f(1) = a + b + c + d + e = 1$,
        \item $f(-1) = a - b + c - d + e = 1$,
        \item $\int_{-1}^{1} f(x) \, dx = 0$.
    \end{enumerate}

    Adding the first two constraints gives $2a + 2c + 2e = 2$, so:
    \[
        a + c + e = 1. \quad (1)
    \]
    Subtracting them gives $2b + 2d = 0$, so:
    \[
        b + d = 0. \quad (2)
    \]
    The integral constraint simplifies using symmetry (odd functions integrate to zero):
    \[
        \int_{-1}^{1} (a x^4 + c x^2 + e) \, dx = 2 \left[ \frac{a x^5}{5} + \frac{c x^3}{3} + e x \right]_0^1 = 2 \left( \frac{a}{5} + \frac{c}{3} + e \right) = 0,
    \]
    so:
    \[
        \frac{a}{5} + \frac{c}{3} + e = 0. \quad (3)
    \]

    The second derivative is:
    \[
        f''(x) = 12a x^2 + 6b x + 2c.
    \]
    Then:
    \[
        [f''(x)]^2 = (12a x^2 + 6b x + 2c)^2 = 144a^2 x^4 + 144ab x^3 + 36b^2 x^2 + 48ac x^2 + 24bc x + 4c^2.
    \]
    Integrating from $-1$ to $1$, the odd terms ($144ab x^3$ and $24bc x$) vanish, and the even part gives:
    \begin{align*}
        \int_{-1}^{1} [f''(x)]^2 \, dx &= 2 \int_0^1 (144a^2 x^4 + (36b^2 + 48ac) x^2 + 4c^2) \, dx \\
        &= 2 \left[ \frac{144a^2}{5} + (12b^2 + 16ac) + 4c^2 \right] \\
        &= \frac{288a^2}{5} + 24b^2 + 32ac + 8c^2.
    \end{align*}

    Since $24b^2 \geq 0$ and equality holds only if $b = 0$ (and thus $d = 0$ from (2)), set $b = d = 0$. The constraints reduce to:
    \begin{itemize}
        \item $a + c + e = 1$,
        \item $\frac{a}{5} + \frac{c}{3} + e = 0$.
    \end{itemize}

    Subtracting the second from the first:
    \[
        (a + c + e) - \left( \frac{a}{5} + \frac{c}{3} + e \right) = 1 - 0 \implies \frac{4a}{5} + \frac{2c}{3} = 1.
    \]
    Solving for $c$:
    \[
        c = \frac{3}{2} - \frac{6a}{5}.
    \]
    Substitute into the integral:
    \[
        I(a) = \frac{288a^2}{5} + 32a \left( \frac{3}{2} - \frac{6a}{5} \right) + 8 \left( \frac{3}{2} - \frac{6a}{5} \right)^2 = \frac{768}{25} a^2 + \frac{96}{5} a + 18.
    \]
    This quadratic in $a$ has a minimum at:
    \[
        a = -\frac{96/5}{2 \cdot 768/25} = -\frac{5}{16}.
    \]
    Then:
    \[
        c = \frac{3}{2} - \frac{6}{5} \left( -\frac{5}{16} \right) = \frac{15}{8}, \quad e = 1 - \left( -\frac{5}{16} \right) - \frac{15}{8} = -\frac{9}{16}.
    \]
    For $f(x) = -\frac{5}{16} x^4 + \frac{15}{8} x^2 - \frac{9}{16}$:
    \[
        f''(x) = \frac{15}{4} (1 - x^2), \quad [f''(x)]^2 = \frac{225}{16} (1 - 2x^2 + x^4),
    \]
    and:
    \[
        \int_{-1}^{1} [f''(x)]^2 \, dx = \frac{225}{16} \int_{-1}^{1} (1 - 2x^2 + x^4) \, dx = 15.
    \]
    For any other function satisfying the constraints, strict convexity and the Euler-Lagrange equation ensure the integral is at least 15, with equality only for this quartic polynomial.

    \subsection*{Functions Achieving Equality}
    Equality holds if and only if:
    \[
        f(x) = -\frac{5}{16} x^4 + \frac{15}{8} x^2 - \frac{9}{16}.
    \]

    \textbf{Final answer:} The minimum value of the integral is 15, achieved by the function $f(x) = -\frac{5}{16} x^4 + \frac{15}{8} x^2 - \frac{9}{16}$. For all other such functions, the integral exceeds 15.

    \boxed{15}
\end{document}
