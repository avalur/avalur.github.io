\documentclass[a4paper,12pt]{article}
\usepackage{amssymb,amsfonts,amsmath}
\usepackage[english]{babel}
\usepackage{latexsym}
\usepackage{epsfig}

% =======================================================================
% Margins --- save forests
\NeedsTeXFormat{LaTeX2e}
\oddsidemargin -10 pt       %   Left margin on odd-numbered pages.
\evensidemargin 10 pt       %   Left margin on even-numbered pages.
\marginparwidth 1 in        %   Width of marginal notes.
\oddsidemargin -1.5 true cm %   Note that \oddsidemargin = \evensidemargin
\evensidemargin -1.5 true cm
\marginparwidth 0.75 in
\textwidth 7.5 true in % Width of text line.
\textheight 26.0 true cm
\topmargin -2.5 true cm

% =======================================================================
% Environments problem, solution, remark

\newcount\probcnt
\newenvironment{problem}[1]{%
  \global\advance\probcnt1%
  \goodbreak\medskip\par\noindent\textbf{Problem~\the\probcnt%
    \if{#1}\empty\else~(#1)\fi.}~}%
{%
  \goodbreak
}

\newenvironment{solution}[1][]{%
  \goodbreak\smallskip\par\noindent\textbf{Solution{\if#1\empty\else~#1\fi}.}~}%
{%
  \goodbreak
}

\newenvironment{remark}[1][]{
  \goodbreak\smallskip\par
  \small
  \noindent\textbf{Remark{\if#1\empty\else~#1\fi}.}~%
}{%
  \goodbreak
  \normalsize
}

\newenvironment{lemma}[1][]{
  \goodbreak\smallskip\par
  \noindent\textit{Lemma{\if#1\empty\else~#1\fi}.}~%
}{%
  \goodbreak\smallskip
}

\newenvironment{proof}[1][]{
  \goodbreak\par
  \noindent\textit{Proof{\if#1\empty\else~#1\fi}.}~%
}{%
  \goodbreak\smallskip
}


% =======================================================================
%%% Some common macros
\newcommand{\RR}{{\mathbb{R}}}
\newcommand{\ZZ}{{\mathbb{Z}}}
\newcommand{\CC}{{\mathbb{C}}}
\newcommand{\QQ}{{\mathbb{Q}}}
\newcommand{\NN}{{\mathbb{N}}}
\newcommand{\tr}{\rm tr}
\newcommand{\ds}{\displaystyle}
\newcommand{\dx}{\mathrm{d}x}
\newcommand{\dy}{\mathrm{d}y}
\newcommand{\dz}{\mathrm{d}z}
\newcommand{\dt}{\mathrm{d}t}
\newcommand{\du}{\mathrm{d}u}
\newcommand{\GL}{\operatorname{GL}}
\newcommand{\rk}{\operatorname{rk}}

% == TITLES ==========================================================
\begin{document}
\begin{center}
  {\Large\textbf{Proposed problems for the
  International Mathematical Competition 2024}}
\end{center}

% =======================================================================
%%% DOCUMENT_BEGIN
% Do not change the document above this point
% =======================================================================

% =======================================================================
% My macros
% =======================================================================
% Insert your macros here

% \def\MyMacro{...}


% =======================================================================

\begin{problem}{Alex Avdiushenko, Neapolis University Paphos, Cyprus}
  Let the set $S$ contain $n^2+n-1$ elements.
  All $n$-element subsets of $S$ are divided into two classes.
  Prove that among them there exist $n$ pairwise disjoint subsets that belong to the same class.
  \begin{remark}
    Relatively hard problem, which can be third or even fourth.
  \end{remark}
\end{problem}

% -----------------------------------------------------------------------

\begin{solution}
We prove by induction the stronger statement that if the set $S$ consists of
$(n+1)m-1$ elements and all $n$-element subsets of $S$ are divided into two classes,
then there exists at least $m$ pairwise disjoint subsets that belong to the same class.
The base case $m=1$ is trivial.

We now carry out the induction step from $m-1$ to $m$.
Assume all the subsets belong to one class. Then, there is nothing to prove.

Otherwise, let's choose two $n$-element subsets $A$ and $B$ from different classes
such that their intersection has the maximal cardinality.
Note that $|A \cap B| = n-1$.
Indeed, if $|A \cap B| = k < n-1$, then consider the set $C$, derived from $B$ by replacing one element,
not belonging to the intersection $A \cap B$, with an element from the set $A \setminus B$.
Then $|A \cap C| = k+1$ and $|B \cap C| = n-1$ and,
either $A$ and $C$ or $B$ and $C$ will be from different classes.

Now, remove $A \cup B$ from $S$, leaving a set of $(n+1)(m-1)-1$ elements,
to which the induction hypothesis applies,
i.e., such that $m-1$ non-intersecting sets from this same class can be found.
Accordingly, in the set $S$, either $A$ or $B$ will belong to this class.
\end{solution}

\begin{problem}{Alex Avdiushenko, Neapolis University Paphos, Cyprus}
Let $R$ be a ring such that if $x^3 = 0$ then $x = 0$.
Moreover, for any elements $a, b \in R$, the equation $(ab)^2 = a^2 b^2$ holds.
Prove that the ring is commutative, i.e. for any two elements $a, b \in R$, $ab = ba$.
\end{problem}

\begin{solution}

  Firstly, $x^2 = 0 \Rightarrow x^3 = 0 \Rightarrow x = 0$.

  Secondly, $ab = 0 \Leftrightarrow ba = 0$, since $ab = 0 \Rightarrow b(ab)a= 0
  \Rightarrow (ba)^2 = 0 \Rightarrow ba = 0$.

  Thirdly, $a(ba-ab)b = abab-aabb=0 \Rightarrow (ba-ab)ba=0$ and analogously
  $b(ba-ab)a = 0 \Rightarrow (ba-ab)ab=0$.

  Fourthly and lastly,
  $(ab-ba)^2 = abab - abba - baab + baba = -(ba-ab)ab + (ba-ab)ba = 0$ and $ab=ba$.

\end{solution}

\begin{problem}{Alex Avdiushenko, Neapolis University Paphos, Cyprus}
Find the greatest real number \( m \) and the smallest \( M \) such
that for any positive numbers \( a, b, c, d, e \), the following inequality holds:

\[ m < \frac{a}{a+b} + \frac{b}{b+c} + \frac{c}{c+d} + \frac{d}{d+e} + \frac{e}{e+a} < M \]
  \begin{remark}
    An easy problem, which can be first one.
  \end{remark}

\end{problem}

\begin{solution}

\textbf{Answer:} \( m = 1, M = 4 \).

\textbf{Examples:} \( (1, \varepsilon^4, \varepsilon^3, \varepsilon^2, \varepsilon) \)
and \( (0, 5^n, 4^n, 3^n, 2^n ) \).

The lower bound can be estimated by replacing the denominator of each fraction with the sum of all the numbers. The upper bound is more complicated:

\begin{eqnarray*}
\frac{a}{a+b} + \frac{b}{b+c} + \frac{c}{c+d} + \frac{d}{d+e} + \frac{e}{e+a} =
\frac{1}{1 + \frac{b}{a}} + \frac{1}{1 + \frac{c}{b}} + \frac{1}{1 + \frac{d}{c}} + \frac{1}{1 + \frac{e}{d}} + \frac{1}{1 + \frac{a}{e}} = \\ =
\frac{1}{1 + x_1} + \frac{1}{1 + x_2} + \frac{1}{1 + x_3} + \frac{1}{1 + x_4} + \frac{1}{1 + x_5}
\end{eqnarray*}

where \( \Prod x_i = 1 \).
Let's prove that among the five numbers \( \{x_i\} \),
there are two whose product is not less than 1.
It is obvious that if there are two numbers, each not less than 1,
then their product will also be not less than 1.
It is also clear that there will be at least one number not less than 1.
If four of the numbers \( \{x_i\} \) are not greater than 1,
then the product of any three of these four numbers will also be not greater than 1,
which means the product of the remaining two will be at least 1.
Without loss of generality, we can assume \( x_1x_2 \geq 1 \). Then,

\begin{eqnarray*}
\frac{1}{1+x_1} + \frac{1}{1+x_2} + \frac{1}{1+x_3} + \frac{1}{1+x_4} + \frac{1}{1+x_5}
= \frac{1 + x_2 + x_1 + 1}{1 + x_2 + x_1 + x_1x_2} + \frac{1}{1+x_3} + \frac{1}{1+x_4} + \frac{1}{1+x_5} \leq \\
\leq 1 + 1 + 1 + 1 = 4
\end{eqnarray*}

\begin{remark}
  Using cyclic replacement of \( a \to b, b \to c, c \to d, d \to e, e \to a \),
  it can be understood that \( m + M = 5 \), and then the second part of the reasoning is not needed.
\end{remark}

\end{solution}


\begin{problem}{Alex Avdiushenko, Neapolis University Paphos, Cyprus}
For which positive integer numbers \( n \) does there exist a square matrix
of order \( n \) with elements 0 and 1, such that its square is a matrix of all ones?
\end{problem}

\begin{solution}

\textbf{Answer:} For exact squares.

\textbf{Solution:} Let \( U \) be a matrix of all ones.
Consider the equality \( A^3 = AU = UA \).
On one side of this matrix, the number of ones in the rows of matrix \( A \) is equal
to the number of ones in the columns of matrix \( A \).
It follows that in matrix \( A \), there is the same number of ones
in each row and in each column, denoted by \( k \).

Now consider the equality \( A^4 = AAU = U^2 \),
from which we conclude that \( n = k^2 \), where \( n \) is the dimension of the matrices.

It remains to construct examples of matrices \( A \) of order \( k^2 \),
the square of which is equal to \( U \). To do this, take \( k \) matrices
of order \( k \) (the first is the identity matrix,
and each subsequent one is obtained from the previous one by cyclically shifting
all ones to the right) and form them from the blocks \( k \times k \).
Next, simply repeat these \( k \) first rows \( (k-1) \) times without changes.
The square of the resulting matrix \( A \) will be equal to \( U \).
\end{solution}

% =======================================================================
% Do not change the document below this point
%%% DOCUMENT_END
% =======================================================================

\end{document}

