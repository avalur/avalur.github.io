\documentclass{article}
\usepackage[utf8]{inputenc}
\usepackage[T1]{fontenc}
\usepackage{amsmath}
\usepackage{amsfonts}
\usepackage{amssymb}
\usepackage{enumitem}

\usepackage[margin=0.7in]{geometry} % Adjust the margins of document here.

\usepackage{tikz}                                          % Для простых рисунков в документе
\usetikzlibrary{matrix,arrows,decorations.pathmorphing,shapes.geometric,calc,snakes,backgrounds,arrows.meta}
\usepackage{xcolor}

\begin{document}
\pagestyle{plain}

\section*{8th IMC 2001, July 19 -- July 25, Prague, Czech Republic, First day}

\subsection*{Problem 1}

Let \( n \) be a positive integer. Consider an \( n \times n \) matrix with entries \( 1, 2, \ldots, n^2 \) written in order starting top left and moving along each row in turn left–to–right. We choose \( n \) entries of the matrix such that exactly one entry is chosen in each row and each column. What are the possible values of the sum of the selected entries?

\textbf{Solution.} The choice corresponds to a permutation \( \sigma \in S_n \), with selected entries at positions \( (j, \sigma(j)) \). The sum is:
\[
\sum_{j=1}^{n} \left( n(j - 1) + \sigma(j) \right)
= n \sum_{j=1}^{n} (j - 1) + \sum_{j=1}^{n} \sigma(j)
= n \cdot \frac{n(n - 1)}{2} + \frac{n(n + 1)}{2} = \frac{n(n^2 + 1)}{2}
\]
Thus, the sum is independent of \( \sigma \).

\subsection*{Problem 2}

Let \( r, s, t \) be positive integers which are pairwise relatively prime. If \( a \) and \( b \) are elements of a commutative multiplicative group with unity element \( e \), and \( a^r = b^s = (ab)^t = e \), prove that \( a = b = e \).

Does the same conclusion hold if \( a \) and \( b \) are elements of an arbitrary non-commutative group?

\textbf{Solution.}
\begin{enumerate}
\item Since \( ab = ba \), and \( us + vt = 1 \) for some integers \( u, v \), then:
\[
ab = (ab)^{us + vt} = (ab)^{us} \cdot (ab)^t^v = a^{us} b^{us} \cdot e = a^{us} b^{us}
\]
So \( ab = a^{us} \), and:
\[
b^r = (ab)^r = a^{usr} = (a^r)^{us} = e
\Rightarrow b = (b^r)^x (b^s)^y = e
\Rightarrow b = e, \quad \text{similarly } a = e
\]
\item \textbf{Counterexample:} Let \( a = (123) \), \( b = (34567) \in S_7 \), then \( a^3 = b^5 = (ab)^7 = e \), but \( a \ne e \), \( b \ne e \).
\end{enumerate}

\subsection*{Problem 3}

Find
\[
\lim_{t \to 1^-} (1 - t) \sum_{n=1}^{\infty} \frac{t^n}{1 + t^n}
\]

\textbf{Solution.}
Let \( h = -\ln t \), so as \( t \to 1^- \), \( h \to 0^+ \):
\[
\sum_{n=1}^\infty \frac{t^n}{1 + t^n}
= \sum_{n=1}^\infty \frac{1}{1 + e^{nh}} \quad \Rightarrow
(1 - t) \sum \to h \sum \to \int_0^\infty \frac{dx}{1 + e^x} = \ln 2
\]

\subsection*{Problem 4}

Let \( k \in \mathbb{N} \). Let \( p(x) \) be a polynomial of degree \( n \) with coefficients in \( \{-1, 0, 1\} \), and divisible by \( (x - 1)^k \). Let \( q \) be prime such that
\[
\frac{q}{\ln q} < \frac{k}{\ln(n+1)}
\]
Prove that all complex \( q \)th roots of unity are roots of \( p(x) \).

\textbf{Solution.}
Let \( p(x) = (x - 1)^k r(x) \), and let \( \varepsilon_j = e^{2\pi i j/q} \). Suppose none of \( \varepsilon_j \) are roots of \( r(x) \). Then:
\[
\left| \prod_{j=1}^{q-1} p(\varepsilon_j) \right| \geq \left| \prod_{j=1}^{q-1} (1 - \varepsilon_j)^k \right| = q^k
\]
But on the other hand:
\[
\left| \prod_{j=1}^{q-1} p(\varepsilon_j) \right| \leq (n + 1)^{q - 1}
\Rightarrow (n+1)^{q-1} \geq q^k \Rightarrow \text{Contradiction}
\]

\subsection*{Problem 5}

Let \( A \) be an \( n \times n \) complex matrix such that \( A \ne \lambda I \) for any \( \lambda \in \mathbb{C} \). Prove that \( A \) is similar to a matrix with at most one non-zero entry on the main diagonal.

\textbf{Solution.}
\emph{Base case:} \( n = 1 \) is trivial. For \( n = 2 \), we analyze all possible cases of matrix entries and use similarity transformations to achieve the required form. For general \( n \), apply induction and block structure:
\[
A = \begin{bmatrix}
A' & * \\
* & \beta
\end{bmatrix}
\Rightarrow \text{Transform } A' \text{ and apply the step recursively}
\]
Even in edge cases (e.g., \( n = 3 \)), a suitable similarity transformation reduces \( A \) to desired form by using trace and matrix similarity invariance.

\subsection*{Problem 6}

Suppose differentiable functions \( a(x), b(x), f(x), g(x) : \mathbb{R} \to \mathbb{R} \) satisfy:

\begin{itemize}
  \item \( f(x) \geq 0 \), \( f'(x) \geq 0 \), \( g(x) > 0 \), \( g'(x) > 0 \)
  \item \( \lim_{x \to \infty} a(x) = A > 0 \), \( \lim b(x) = B > 0 \)
  \item \( \lim f(x) = \lim g(x) = \infty \)
  \item \( \frac{f'}{g'} + a(x) \cdot \frac{f}{g} = b(x) \)
\end{itemize}

Prove:
\[
\lim_{x \to \infty} \frac{f(x)}{g(x)} = \frac{B}{A + 1}
\]

\textbf{Solution.}
Let \( h(x) = f(x) g(x)^A \), then:
\[
h'(x) = f'(x) g(x)^A + A f(x) g(x)^{A-1} g'(x)
\Rightarrow \frac{h'}{(g^{A+1})'} = \frac{f'(x)}{g'(x)} + A \cdot \frac{f(x)}{g(x)}
\]
Using original equation:
\[
\frac{h'}{(g^{A+1})'} = b(x) - (a(x) - A)\frac{f(x)}{g(x)}
\Rightarrow \lim_{x \to \infty} \frac{h'}{(g^{A+1})'} = B
\Rightarrow \lim \frac{f(x)}{g(x)} = \frac{B}{A + 1}
\]

\end{document}
