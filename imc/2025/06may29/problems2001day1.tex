\documentclass{article}
\usepackage[utf8]{inputenc}
\usepackage[T1]{fontenc}
\usepackage{amsmath}
\usepackage{amsfonts}
\usepackage{amssymb}
\usepackage{enumitem}

\usepackage[margin=0.7in]{geometry} % Adjust the margins of document here.

\usepackage{tikz}                                          % Для простых рисунков в документе
\usetikzlibrary{matrix,arrows,decorations.pathmorphing,shapes.geometric,calc,snakes,backgrounds,arrows.meta}
\usepackage{xcolor}

\begin{document}
\pagestyle{plain}

\section*{8th IMC 2001, July 19 -- July 25, Prague, Czech Republic, First day}

\subsection*{Problem 1}
Let \( n \) be a positive integer. Consider an \( n \times n \) matrix with entries \( 1, 2, \ldots, n^2 \) written in order starting top left and moving along each row in turn left–to–right. We choose \( n \) entries of the matrix such that exactly one entry is chosen in each row and each column. What are the possible values of the sum of the selected entries?

\subsection*{Problem 2}
Let \( r, s, t \) be positive integers which are pairwise relatively prime. If \( a \) and \( b \) are elements of a commutative multiplicative group with unity element \( e \), and \( a^r = b^s = (ab)^t = e \), prove that \( a = b = e \).

Does the same conclusion hold if \( a \) and \( b \) are elements of an arbitrary non-commutative group?

\subsection*{Problem 3}
Find
\[
\lim_{t \to 1^-} (1 - t) \sum_{n=1}^{\infty} \frac{t^n}{1 + t^n}
\]

\subsection*{Problem 4}
Let \( k \in \mathbb{N} \). Let \( p(x) \) be a polynomial of degree \( n \) with coefficients in \( \{-1, 0, 1\} \), and divisible by \( (x - 1)^k \). Let \( q \) be prime such that
\[
\frac{q}{\ln q} < \frac{k}{\ln(n+1)}
\]
Prove that all complex \( q \)th roots of unity are roots of \( p(x) \).

\subsection*{Problem 5}
Let \( A \) be an \( n \times n \) complex matrix such that \( A \ne \lambda I \)
for any \( \lambda \in \mathbb{C} \). Prove that \( A \) is similar to a matrix
with at most one non-zero entry on the main diagonal.


\subsection*{Problem 6}
Suppose differentiable functions \( a(x), b(x), f(x), g(x) : \mathbb{R} \to \mathbb{R} \) satisfy:

\begin{itemize}
  \item \( f(x) \geq 0 \), \( f'(x) \geq 0 \), \( g(x) > 0 \), \( g'(x) > 0 \)
  \item \( \lim_{x \to \infty} a(x) = A > 0 \), \( \lim b(x) = B > 0 \)
  \item \( \lim f(x) = \lim g(x) = \infty \)
  \item \( \frac{f'}{g'} + a(x) \cdot \frac{f}{g} = b(x) \)
\end{itemize}

Prove:
\[
\lim_{x \to \infty} \frac{f(x)}{g(x)} = \frac{B}{A + 1}
\]
\end{document}
