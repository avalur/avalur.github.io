\documentclass[a4paper,12pt]{article}
\usepackage{amssymb,amsfonts,amsmath}
\usepackage[english]{babel}
\usepackage{latexsym}
\usepackage{epsfig}

% =======================================================================
% Margins --- save forests
\NeedsTeXFormat{LaTeX2e}
\oddsidemargin -10 pt       %   Left margin on odd-numbered pages.
\evensidemargin 10 pt       %   Left margin on even-numbered pages.
\marginparwidth 1 in        %   Width of marginal notes.
\oddsidemargin -1.5 true cm %   Note that \oddsidemargin = \evensidemargin
\evensidemargin -1.5 true cm
\marginparwidth 0.75 in
\textwidth 7.5 true in % Width of text line.
\textheight 26.0 true cm
\topmargin -2.5 true cm

% =======================================================================
% Environments problem, solution, remark

\newcount\probcnt
\newenvironment{problem}[1]{%
  \global\advance\probcnt1%
  \goodbreak\medskip\par\noindent\textbf{Problem~\the\probcnt%
    \if{#1}\empty\else~(#1)\fi.}~}%
{%
  \goodbreak
}

\newenvironment{solution}[1][]{%
  \goodbreak\smallskip\par\noindent\textbf{Solution{\if#1\empty\else~#1\fi}.}~}%
{%
  \goodbreak
}

\newenvironment{remark}[1][]{
  \goodbreak\smallskip\par
  \small
  \noindent\textbf{Remark{\if#1\empty\else~#1\fi}.}~%
}{%
  \goodbreak
  \normalsize
}

\newenvironment{lemma}[1][]{
  \goodbreak\smallskip\par
  \noindent\textit{Lemma{\if#1\empty\else~#1\fi}.}~%
}{%
  \goodbreak\smallskip
}

\newenvironment{proof}[1][]{
  \goodbreak\par
  \noindent\textit{Proof{\if#1\empty\else~#1\fi}.}~%
}{%
  \goodbreak\smallskip
}


% =======================================================================
%%% Some common macros
\newcommand{\RR}{{\mathbb{R}}}
\newcommand{\ZZ}{{\mathbb{Z}}}
\newcommand{\CC}{{\mathbb{C}}}
\newcommand{\QQ}{{\mathbb{Q}}}
\newcommand{\NN}{{\mathbb{N}}}
\newcommand{\tr}{\rm tr}
\newcommand{\ds}{\displaystyle}
\newcommand{\dx}{\mathrm{d}x}
\newcommand{\dy}{\mathrm{d}y}
\newcommand{\dz}{\mathrm{d}z}
\newcommand{\dt}{\mathrm{d}t}
\newcommand{\du}{\mathrm{d}u}
\newcommand{\GL}{\operatorname{GL}}
\newcommand{\rk}{\operatorname{rk}}

% == TITLES ==========================================================
\begin{document}
\begin{center}
  {\Large\textbf{Proposed problems for the IMC 2025}}
\end{center}

% =======================================================================
%%% DOCUMENT_BEGIN
% Do not change the document above this point
% =======================================================================

% =======================================================================
% My macros
% =======================================================================
% Insert your macros here

% \def\MyMacro{...}


% =======================================================================

\begin{problem}{Alex Avdiushenko, Neapolis University Paphos, Cyprus}
  Find all strictly monotonic functions \(f:\mathbb{R}\to\mathbb{R}\) such that
\[
f(x)\,f^{-1}(x) = x^2.
\]
\end{problem}

% -----------------------------------------------------------------------

\begin{solution}
  \begin{enumerate}
    \item[\textbf{1.}] \textit{Value at the origin.}\;
          Substituting \(x=0\) into equation gives
          \[
            f(0)\,f^{-1}(0)=0 .
          \]
          If \(f(0)\neq0\) then \(f^{-1}(0)=0\); applying \(f\) to both sides forces
          \(f(0)=0\) --- a contradiction. Hence
          \[
            f(0)=0 \tag{1}
          \]

    \item[\textbf{2.}] It is clear that either there exists an \(a\) such that \(f(a)=ka\) with \(k\neq1\),
    or else \(\forall x\, f(x)=x\).

    Substituting \(x=ka\) into the given equation yields \( f(ka)=k^2a \).
    Proceeding by induction, one easily shows that
    \[
      f\bigl(k^na\bigr)=k^{n+1}a
      \qquad(n\in\mathbb{Z}_{\ge0}).
    \]

    Now take a number \(x\) lying between \(a\) and \(f(a) = ka\) and set \(f(x)=lx\).
    By monotonicity, the value \(f^{n}(x)\) lies between \(f^{n}(a)\) and \(f^{n+1}(a)\).
    In other words, the number \(l^{n}x\) is always situated between \(k^na\) and \(k^{n+1}a\),
    whence it follows that \(l=k\).

    \item[\textbf{3.}] Because \(f\) is a strictly monotonic bijection, every real
          number lies between some (forward or backward) iterate of \(a\) and \(f(a) = ka\);
    thus
          \[
            f(x) \equiv k\,x
          \]

    \item[\textbf{4.}] \textit{Verification.}\;
          For \(f(x)=k\,x\) we have \(f^{-1}(x)=x/k\), hence
          \[
            f(x)\,f^{-1}(x)=\bigl(kx\bigr)\Bigl(\frac{x}{k}\Bigr)=x^{2}.
          \]

          If \(k>0\) the function is strictly increasing; if \(k<0\) it is
          strictly decreasing, fulfilling the monotonicity requirement.
  \end{enumerate}

\end{solution}

% =======================================================================

\begin{problem}{Alex Avdiushenko, Neapolis University Paphos, Cyprus}
  Find all continuous functions \(f:\mathbb{R}\to\mathbb{R}\) such that,
  for every \(x,y\in\mathbb{R}\),
  \[
  (x-y)\,f(x+y) \;=\; (x+y)\,\bigl(f(x)-f(y)\bigr).
  \]

\end{problem}

% -----------------------------------------------------------------------

\begin{solution}
\textbf{Answer:} \(ax^{2}+bx\).

  \begin{enumerate}
  \item \emph{$f(0)=0$.}  Put $y=0$ in the equation:
        \[
        x\,f(x)=x\bigl(f(x)-f(0)\bigr)\;\Longrightarrow\;f(0)=0.
        \]

  \item Consider the three special pairs $(1,z)$, $(1,z+1)$, $(2,z)$:

  Fix $z\in\mathbb{R}\setminus\{0,1,2\}$.
  Using original equation with the pairs listed, we successively get
  \begin{align*}
  (1,z):\;& f(1+z)=\frac{1+z}{1-z}\,f(1)-\frac{1+z}{1-z}\,f(z),\\[4pt]
  (1,z+1):\;& f(z+2)=-\frac{2+z}{z}\,f(1)+\frac{2+z}{z}\,f(z+1),\\[4pt]
  (2,z):\;& f(z+2)=\frac{2+z}{2-z}\,f(2)-\frac{2+z}{2-z}\,f(z).
  \end{align*}
  Substituting the first line into the second yields
  \begin{align*}
  f(z+2) = \frac{2+z}{z} \left(-f(1) + \frac{1+z}{1-z}\,f(1)-\frac{1+z}{1-z}\,f(z) \right) = \\[4pt]
  = \frac{2+z}{z} \left(\frac{2z}{1-z}\,f(1)  - \frac{1+z}{1-z}\,f(z) \right) = \\[4pt]
  = \frac{2(2+z)}{z(1-z)}\,f(1) - \frac{(2+z)(1+z)}{z(1-z)}\,f(z) \tag{3}
  \end{align*}

  \item Using the third line (for \((2, z)\)) and (3), we get
  \begin{align*}
  f(z+2) = \frac{2+z}{2-z}\,f(2)-\frac{2+z}{2-z}\,f(z) =
  \frac{2(2+z)}{z(1-z)}\,f(1) - \frac{(2+z)(1+z)}{z(1-z)}\,f(z) \\[4pt]
  \text{we reduce \((2+z)\)} \\[4pt]
  \frac{1}{2-z}\,f(2)-\frac{1}{2-z}\,f(z) =
  \frac{2}{z(1-z)}\,f(1) - \frac{1+z}{z(1-z)}\,f(z) \\[4pt]
  \text{and group \(f(z)\)} \\[4pt]
  \left( \frac{1+z}{z(1-z)} -\frac{1}{2-z} \right)\,f(z) =
  \frac{2}{z(1-z)}\,f(1) - \frac{1}{2-z}\,f(2) \\[4pt]
  \text{and multiply \(z(1-z)(2-z)\)} \\[4pt]
  \left( (1+z)(2-z) - z(1-z) \right)\,f(z) = 2(2-z)\,f(1) - z(1-z)\,f(2) \\[4pt]
  2 f(z) = 2(2-z)\,f(1) - z(1-z)\,f(2) \\[4pt]
    f(z) = 2\,f(1) - z \left( f(1) + \frac{f(2)}{2} \right) + \frac{z^2}{2}\,f(2)
  \end{align*}

  \item That is, the function can only be quadratic, and \(f(0) = 0\),
  and therefore \( f(x) = ax^2 + bx\), which, as can and should be verified,
  satisfies the original equation.
  \end{enumerate}

  \remark{In a similar way it can be calculated that \(f(z+2) - 2f(z+1) + f(z) = 0\),
  which is discrete analogue of ``second derivative = 0``.}
\end{solution}

\begin{problem}{Alex Avdiushenko, Neapolis University Paphos, Cyprus}
Let $P,Q,R\in O(3)$ be real orthogonal $3\times 3$ matrices,
i.e.\ $P^{\mathsf T}P=Q^{\mathsf T}Q=R^{\mathsf T}R=I_3$.

Show that the matrix equation
\[
  P + Q \;=\; R
\]

has no solutions in $O(3)$.

\end{problem}

\begin{solution}
Assume, for contradiction, that $P,Q,R\in O(3)$ satisfy $P+Q=R$.

\smallskip
\textbf{1.  Reduce to one orthogonal matrix.}
Set
\[
S:=P^{\mathsf T}Q\in O(3),\qquad
Q=PS,\qquad
R=P(I_3+S).
\]

\textbf{2.  Orthogonality of $R$.}
Because $R$ is orthogonal,
\[
I_3=R^{\mathsf T}R=(I_3+S)^{\mathsf T}(I_3+S)
         =I_3+S^{\mathsf T}+S+S^{\mathsf T}S
         =I_3+S^{\mathsf T}+S+I_3,
\]
whence
\[
S+S^{\mathsf T}=-I_3
\]

\textbf{3.  Minimal polynomial of $S$.}
Multiplying the last identity on the left by $S$ gives
\[
S^2+S+I_3=0.
\]
Hence every eigenvalue $\lambda$ of $S$ satisfies $\lambda^2+\lambda+1=0$, i.e.
\[
\lambda=\exp\!\bigl(\pm 2\pi i/3\bigr)\quad(\text{non-real})
\]

\textbf{4.  Dimension parity contradiction.}
For a real matrix, non-real eigenvalues occur in complex-conjugate pairs; thus a real $3\times3$ matrix with only non-real eigenvalues cannot exist (the dimension is odd).
Therefore such an $S\in O(3)$ does not exist.

\textbf{5.  Conclusion.}
Since $S$ cannot exist, neither can a triple $(P,Q,R)\in O(3)^3$ with $P+Q=R$.
Hence the equation admits \emph{no} solutions in $O(3)$.

\remark{
The same argument generalises to any dimension $n$.
If an orthogonal matrix $S\in O(n)$ satisfies $S+S^{\mathsf T}=-I_n$, then it obeys the polynomial identity $S^{2}+S+I_n=0$; hence its eigenvalues are the two complex cubic roots of unity
\[
\lambda=\exp\!\bigl(\pm 2\pi i/3\bigr),
\]
which are non-real.
Because complex eigenvalues of a real matrix occur in conjugate pairs, such an $S$ can exist only when the dimension $n$ is \emph{even}.

Consequently, the $n=2$ construction based on a $120^\circ$ rotation extends to every even dimension by taking block-diagonal (direct sum) copies of the $2\times2$ block, whereas for every odd $n$ no orthogonal matrices $P,Q,R$ satisfy $P+Q=R$.
}
\end{solution}

% =======================================================================
% Do not change the document below this point
%%% DOCUMENT_END
% =======================================================================

\end{document}

