\documentclass{article}
\usepackage[utf8]{inputenc}
\usepackage[T1]{fontenc}
\usepackage{amsmath}
\usepackage{amsfonts}
\usepackage{amssymb}
\usepackage{enumitem}

\usepackage[margin=0.7in]{geometry} % Adjust the margins of document here.

\usepackage{tikz}                                          % Для простых рисунков в документе
\usetikzlibrary{matrix,arrows,decorations.pathmorphing,shapes.geometric,calc,snakes,backgrounds,arrows.meta}
\usepackage{xcolor}

\begin{document}
\pagestyle{plain}

\section*{Second Olympiad for NUP team selection}

\begin{center}
May 2025
\end{center}

\textbf{Problem 1.} (10 points)
Find all functions $f : \mathbb{R} \to \mathbb{R}$ such that for any real numbers $a < b$, the image $f([a,b])$ is a closed interval of length $b - a$. \\

\textit{Solution.} The functions $f(x) = x + c$ and $f(x) = -x + c$ with some constant $c$ obviously satisfy the condition of the problem. We will prove now that these are the only functions with the desired property.

Let $f$ be such a function. Then $f$ clearly satisfies $|f(x) - f(y)| \leq |x - y|$ for all $x, y$; therefore, $f$ is continuous. Given $x, y$ with $x < y$, let $a, b \in [x, y]$ be such that $f(a)$ is the maximum and $f(b)$ is the minimum of $f$ on $[x, y]$. Then $f([x, y]) = [f(b), f(a)]$; hence
\[
y - x = f(a) - f(b) \leq |a - b| \leq y - x
\]
This implies $\{a, b\} = \{x, y\}$, and therefore $f$ is a monotone function. Suppose $f$ is increasing. Then $f(x) - f(y) = x - y$ implies $f(x) - x = f(y) - y$, which says that $f(x) = x + c$ for some constant $c$. Similarly, the case of a decreasing function $f$ leads to $f(x) = -x + c$ for some constant $c$.\\

\textbf{Problem 2.} (10 points)
Let $A$ be an $n \times n$ real matrix such that $3A^3 = A^2 + A + I$ ($I$ is the identity matrix). Show that the sequence $A^k$ converges to an idempotent matrix. (A matrix $B$ is called idempotent if $B^2 = B$.) \\

\textit{Solution.} The minimal polynomial of $A$ is a divisor of $3x^3 - x^2 - x - 1$.
This polynomial has three different roots.
This implies that $A$ is diagonalizable: $A = C^{-1}DC$ where $D$ is a diagonal matrix.
The eigenvalues of the matrices $A$ and $D$ are all roots of polynomial $3x^3 - x^2 - x - 1$.
One of the three roots is $1$, the remaining two roots have smaller absolute value than $1$.
Hence, the diagonal elements of $D^k$, which are the $k$-th powers of the eigenvalues, tend to either $0$ or $1$ and the limit $M = \lim D^k$ is idempotent. Then $\lim A^k = C^{-1}MC$ is idempotent as well.\\

\textbf{Problem 3.} (10 points)
Let $a_1, a_2, \ldots, a_{51}$ be non-zero elements of a field. We simultaneously replace each element with the sum of the 50 remaining ones. In this way we get a sequence $b_1, \ldots, b_{51}$. If this new sequence is a permutation of the original one, what can be the characteristic of the field? (The characteristic of a field is $p$, if $p$ is the smallest positive integer such that
\[
\underbrace{x + x + \cdots + x}_{p} = 0
\]
for any element $x$ of the field. If there exists no such $p$, the characteristic is 0.) \\

\textit{Solution.} Let $S = a_1 + a_2 + \cdots + a_{51}$. Then $b_1 + b_2 + \cdots + b_{51} = 50S$. Since $b_1, b_2, \ldots, b_{51}$ is a permutation of $a_1, a_2, \cdots, a_{51}$, we get $50S = S$, so $49S = 0$. Assume that the characteristic of the field is not equal to 7. Then $49S = 0$ implies that $S = 0$. Therefore $b_i = -a_i$ for $i = 1, 2, \ldots, 51$. On the other hand, $b_i = a_{\varphi(i)}$ where $\varphi \in S_{51}$. Therefore, if the characteristic is not 2, the sequence $a_1, a_2, \cdots, a_{51}$ can be partitioned into pairs $\{a_i, a_{\varphi(i)}\}$ of additive inverses. But this is impossible, since 51 is an odd number. It follows that the characteristic of the field is 7 or 2.

The characteristic can be either 2 or 7. For the case of 7, $x_1 = \cdots = x_{51} = 1$ is a possible choice. For the case of 2, any elements can be chosen such that $S = 0$, since then $b_i = -a_i = a_i$. \\

\textbf{Problem 4.} (10 points)
Find all differentiable functions $f : (0, \infty) \to \mathbb{R}$ such that
\[
f(b) - f(a) = (b - a) f'\left( \sqrt{ab} \right) \quad \text{for all} \quad a, b > 0. \tag{1}
\]

\textit{Solution.} First, we show that it $f$ is infinitely many times differentiable.
By substituting $a = \frac{1}{2}t$ and $b = 2t$ in (1),
\[
f'(t) = \frac{f(2t) - f\left(\frac{1}{2}t\right)}{\frac{3}{2}t}. \tag{2}
\]
Inductively, if $f$ is $k$ times differentiable
then the right-hand side of (2) is $k$ times differentiable, so the $f'(t)$ on the left-hand side is $k$ times differentiable as well; hence $f$ is $k + 1$ times differentiable.

Now substitute $b = e^{h}t$ and $a = e^{-h}t$ in (1), differentiate three times with respect to $h$ then take limits with $h \to 0$:
\[
f(e^{h}t) - f(e^{-h}t) - (e^{h}t - e^{-h}t)f'(t) = 0
\]
\[
\left( \frac{\partial}{\partial h} \right)^3 \left( f(e^{h}t) - f(e^{-h}t) - (e^{h}t - e^{-h}t)f'(t) \right) = 0
\]
\[
e^{3ht} t^3 f'''(e^{ht}) + 3e^{2ht} t^2 f''(e^{ht}) + e^{ht} t f'(e^{ht}) + e^{-3ht} t^3 f'''(e^{-ht}) + 3e^{-2ht} t^2 f''(e^{-ht}) + e^{-ht} t f'(e^{-ht})
\]
\[
- (e^{ht} + e^{-ht}) f'(t) = 0
\]
\[
2t^3 f'''(t) + 6t^2 f''(t) = 0
\]
\[
t f'''(t) + 3 f''(t) = 0
\]
\[
(t f(t))''' = 0.
\]

Consequently, $t f(t)$ is at most a quadratic polynomial of $t$, and therefore
\[
f(t) = C_1 t + \frac{C_2}{t} + C_3 \tag{3}
\]
with some constants $C_1, C_2$, and $C_3$.

It is easy to verify that all functions of the form (3) satisfy the equation (1).\\

\textbf{Problem 5.} (10 points)
Let $f(x)$ be a polynomial with real coefficients of degree $n$.
Suppose that \( \frac{f(k) - f(m)}{k - m} \)
is an integer for all integers $0 \leq k < m \leq n$. Prove that $a - b$ divides $f(a) - f(b)$ for all pairs of distinct integers $a$ and $b$. \\

\vspace{1em}
\textit{Solution 1.} We need the following

\textbf{Lemma.} Denote the least common multiple of $1, 2, \ldots, k$ by $L(k)$, and define
\[
h_k(x) = L(k) \cdot \binom{x}{k} \quad (k = 1, 2, \ldots).
\]
Then the polynomial $h_k(x)$ satisfies the condition, i.e. $a - b$ divides $h_k(a) - h_k(b)$ for all pairs of distinct integers $a, b$.

\textit{Proof.} It is known that
\[
\binom{a}{k} = \sum_{j=0}^{k} \binom{a - b}{j} \binom{b}{k - j}.
\]
(This formula can be proved by comparing the coefficient of $x^k$ in $(1 + x)^a$ and $(1 + x)^{a-b}(1 + x)^b$.) From here we get:
\[
h_k(a) - h_k(b) = L(k) \left( \binom{a}{k} - \binom{b}{k} \right)
= L(k) \sum_{j=1}^{k} \binom{a - b}{j} \binom{b}{k - j}
= (a - b) \sum_{j=1}^{k} \frac{L(k)}{j} \binom{a - b - 1}{j - 1} \binom{b}{k - j}.
\]

On the right-hand side all fractions $\frac{L(k)}{j}$ are integers, so the right-hand side is a multiple of $(a - b)$. The lemma is proved.

Expand the polynomial $f$ in the basis $1$, $\binom{x}{1}$, $\binom{x}{2}, \ldots$ as
\begin{equation}
f(x) = A_0 + A_1 \binom{x}{1} + A_2 \binom{x}{2} + \cdots + A_n \binom{x}{n}. \tag{1}
\end{equation}

We prove by induction on $j$ that $A_j$ is a multiple of $L(j)$ for $1 \leq j \leq n$. (In particular, $A_j$ is an integer for $j \geq 1$.) Assume that $L(j)$ divides $A_j$ for $1 \leq j \leq m - 1$. Substituting $m$ and some $k \in \{0, 1, \ldots, m - 1\}$ in (1),
\[
\frac{f(m) - f(k)}{m - k} = \sum_{j=1}^{m-1} \frac{A_j}{L(j)} \cdot \frac{h_j(m) - h_j(k)}{m - k} + \frac{A_m}{m - k}.
\]

Since all other terms are integers, the last term $\frac{A_m}{m - k}$ is also an integer. This holds for all $0 \leq k < m$, so $A_m$ is an integer that is divisible by $L(m)$.

Hence, $A_j$ is a multiple of $L(j)$ for every $1 \leq j \leq n$. By the lemma this implies the original statement.

\textit{Solution 2.} The statement of the problem follows immediately from the following claim, applied to the polynomial
\[
g(x, y) = \frac{f(x) - f(y)}{x - y}.
\]

\textbf{Claim.} Let $g(x, y)$ be a real polynomial of two variables with total degree less than $n$. Suppose that $g(k, m)$ is an integer whenever $0 \leq k < m \leq n$ are integers. Then $g(k, m)$ is an integer for every pair $k, m$ of integers.

\textit{Proof.} Apply induction on $n$. If $n = 1$ then $g$ is a constant. This constant can be read from $g(0, 1)$ which is an integer, so the claim is true.

Now suppose that $n \geq 2$ and the claim holds for $n - 1$. Consider the polynomials
\[
g_1(x, y) = g(x + 1, y + 1) - g(x, y + 1) \quad \text{and} \quad g_2(x, y) = g(x, y + 1) - g(x, y). \tag{1}
\]

For every pair $0 \leq k < m \leq n - 1$ of integers, the numbers $g(k, m)$, $g(k, m + 1)$ and $g(k + 1, m + 1)$ are all integers, so $g_1(k, m)$ and $g_2(k, m)$ are integers, too. Moreover, in (1) the maximal degree terms of $g$ cancel out, so $\deg g_1, \deg g_2 < \deg g$. Hence, we can apply the induction hypothesis to the polynomials $g_1$ and $g_2$ and we thus have $g_1(k, m), g_2(k, m) \in \mathbb{Z}$ for all $k, m \in \mathbb{Z}$.

In view of (1), for all $k, m \in \mathbb{Z}$, we have that

\begin{itemize}
    \item[(a)] $g(0, 1) \in \mathbb{Z}$;
    \item[(b)] $g(k, m) \in \mathbb{Z}$ if and only if $g(k + 1, m + 1) \in \mathbb{Z}$;
    \item[(c)] $g(k, m) \in \mathbb{Z}$ if and only if $g(k, m + 1) \in \mathbb{Z}$.
\end{itemize}

For arbitrary integers $k, m$, apply (b) $|k|$ times then apply (c) $|m - k - 1|$ times as
\[
g(k, m) \in \mathbb{Z} \Leftrightarrow \dots \Leftrightarrow g(0, m - k) \in \mathbb{Z} \Leftrightarrow \dots \Leftrightarrow g(0, 1) \in \mathbb{Z}.
\]

Hence, $g(k, m) \in \mathbb{Z}$. The claim has been proved.

\end{document}
