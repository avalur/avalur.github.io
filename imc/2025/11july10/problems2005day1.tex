\documentclass{article}
\usepackage[utf8]{inputenc}
\usepackage[T1]{fontenc}
\usepackage{amsmath}
\usepackage{amsfonts}
\usepackage{amssymb}
\usepackage{enumitem}

\usepackage[margin=0.7in]{geometry} % Adjust the margins of document here.

\usepackage{tikz}                                          % Для простых рисунков в документе
\usetikzlibrary{matrix,arrows,decorations.pathmorphing,shapes.geometric,calc,snakes,backgrounds,arrows.meta}
\usepackage{xcolor}

\begin{document}
\pagestyle{plain}

\section*{12th IMC 2005, July 22 -- 28, Blagoevgrad}

\section*{First Day}

\subsection*{Problem 1}
Let \( A \) be the \( n \times n \) matrix, whose \( (i,j)^{\text{th}} \) entry is \( i + j \) for all \( i, j = 1, 2, \ldots, n \). What is the rank of \( A \)?

\subsection*{Problem 2}
For an integer \( n \geq 3 \) consider the sets
\[
S_n = \{(x_1, x_2, \ldots, x_n) : \forall i \; x_i \in \{0,1,2\} \}
\]
\[
A_n = \{(x_1, x_2, \ldots, x_n) \in S_n : \forall i \leq n - 2 \; |\{x_i, x_{i+1}, x_{i+2}\}| \neq 1 \}
\]
and
\[
B_n = \{(x_1, x_2, \ldots, x_n) \in S_n : \forall i \leq n - 1 \; (x_i = x_{i+1} \Rightarrow x_i \neq 0)\}.
\]

Prove that \( |A_{n+1}| = 3 \cdot |B_n| \).\\
(\(|A|\) denotes the number of elements of the set \(A\).)

\subsection*{Problem 3}
Let \( f : \mathbb{R} \to [0, \infty) \) be a continuously differentiable function. Prove that
\[
\left| \int_0^1 f^3(x)\,dx - f^2(0) \int_0^1 f(x)\,dx \right|
\leq \max_{0 \leq x \leq 1} |f'(x)| \left( \int_0^1 f(x)\,dx \right)^2.
\]

\subsection*{Problem 4}
Find all polynomials \( P(x) = a_n x^n + a_{n-1} x^{n-1} + \ldots + a_1 x + a_0 \) (\( a_n \neq 0 \)) satisfying the following two conditions:
\begin{itemize}
    \item[(i)] \( (a_0, a_1, \ldots, a_n) \) is a permutation of the numbers \( (0, 1, \ldots, n) \)
    \item[(ii)] All roots of \( P(x) \) are rational numbers.
\end{itemize}

\subsection*{Problem 5}
Let \( f : (0, \infty) \to \mathbb{R} \) be a twice continuously differentiable function such that
\[
\left| f''(x) + 2x f'(x) + (x^2 + 1) f(x) \right| \leq 1
\]
for all \( x \). Prove that \( \displaystyle \lim_{x \to \infty} f(x) = 0 \).

\subsection*{Problem 6}
Given a group \( G \), denote by \( G(m) \) the subgroup generated by the \( m^{\text{th}} \) powers of elements of \( G \). If \( G(m) \) and \( G(n) \) are commutative, prove that \( G(\gcd(m,n)) \) is also commutative.

(\( \gcd(m,n) \) denotes the greatest common divisor of \( m \) and \( n \).)

\end{document}
