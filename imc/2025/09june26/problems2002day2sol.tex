\documentclass{article}
\usepackage[utf8]{inputenc}
\usepackage[T1]{fontenc}
\usepackage{amsmath}
\usepackage{amsfonts}
\usepackage{amssymb}
\usepackage{enumitem}

\usepackage[margin=0.7in]{geometry} % Adjust the margins of document here.

\usepackage{tikz}                                          % Для простых рисунков в документе
\usetikzlibrary{matrix,arrows,decorations.pathmorphing,shapes.geometric,calc,snakes,backgrounds,arrows.meta}
\usepackage{xcolor}

\begin{document}
\pagestyle{plain}

\section*{9th IMC 2002, July 19 -- 25, Warsaw, Poland}

\section*{Second Day}

\subsection*{Problem 1}
Compute the determinant of the $n \times n$ matrix $A = [a_{ij}]$, where
\[
a_{ij} =
    \begin{cases}
        (-1)^{|i-j|}, & \text{if } i \ne j, \\
        2, & \text{if } i = j.
    \end{cases}
\]

\textbf{Solution.}
Add the second row to the first one, then add the third row to the second one, and so on, until adding the $n$th row to the $(n-1)$th. The determinant remains unchanged, and the matrix becomes:

\[
\begin{vmatrix}
2 & -1 & +1 & \cdots & \pm 1 & \mp 1 \\
-1 & 2 & -1 & \cdots & \mp 1 & \pm 1 \\
+1 & -1 & 2 & \cdots & \pm 1 & \mp 1 \\
\vdots & \vdots & \vdots & \ddots & \vdots & \vdots \\
\mp 1 & \pm 1 & \mp 1 & \cdots & 2 & -1 \\
\pm 1 & \mp 1 & \pm 1 & \cdots & -1 & 2 \\
\end{vmatrix}
=
\begin{vmatrix}
1 & 1 & 0 & \cdots & 0 & 0 \\
0 & 1 & 1 & \cdots & 0 & 0 \\
0 & 0 & 1 & \cdots & 0 & 0 \\
\vdots & \vdots & \vdots & \ddots & \vdots & \vdots \\
0 & 0 & 0 & \cdots & 1 & 1 \\
\pm 1 & \mp 1 & \pm 1 & \cdots & -1 & 2 \\
\end{vmatrix}
\]

Now subtract the first column from the second, then subtract the new second from the third, and so on until subtracting the $(n-1)$th from the $n$th column. The resulting matrix is:

\[
\begin{vmatrix}
1 & 0 & 0 & \cdots & 0 & 0 \\
0 & 1 & 0 & \cdots & 0 & 0 \\
\vdots & \vdots & \vdots & \ddots & \vdots & \vdots \\
0 & 0 & 0 & \cdots & 1 & 0 \\
0 & 0 & 0 & \cdots & 0 & n+1 \\
\end{vmatrix} = n + 1.
\]

\subsection*{Problem 2}

Two hundred students participated in a mathematical contest.
They had 6 problems to solve. It is known that each problem was correctly solved by at least 120 participants.
Prove that there must be two participants such that every problem was solved by
at least one of these two students.

\textbf{Solution.}
For each pair of students, consider the set of problems not solved by either of them. There are $\binom{200}{2} = 19900$ such pairs.

For each problem, at most 80 students did not solve it. Thus for each problem, at most $\binom{80}{2} = 3160$ pairs of students both failed to solve it. So 6 problems contribute at most $6 \cdot 3160 = 18960$ such sets.

Hence at least $19900 - 18960 = 940$ pairs solved all problems between them. Therefore, at least one such pair exists.

\subsection*{Problem 3}

For each $n \ge 1$ define
\[
    a_n = \sum_{k=0}^\infty \frac{k^n}{k!}, \qquad
    b_n = \sum_{k=0}^\infty \frac{(-1)^k k^n}{k!}.
\]
Show that $a_n \cdot b_n$ is an integer.

\textbf{Solution.}
We show by induction that $a_n/e$ and $b_n e$ are integers.

For $n = 0$, $a_0 = e$, $b_0 = 1/e$.

Assume $a_0, \dotsc, a_n$ and $b_0, \dotsc, b_n$ satisfy the condition. Then:
\[
a_{n+1} = \sum_{k=0}^\infty \frac{(k+1)^n}{k!} = \sum_{m=0}^n \binom{n}{m} a_m,
\]
and similarly:
\[
b_{n+1} = -\sum_{m=0}^n \binom{n}{m} b_m.
\]
So both sequences satisfy integer recurrence relations, implying $a_n b_n$ is rational, and in fact an integer.

\subsection*{Problem 4}
In the tetrahedron $OABC$, let $\angle BOC = \alpha$, $\angle COA = \beta$, $\angle AOB = \gamma$.
Let $\sigma$ be the angle between the faces $OAB$ and $OAC$,
and $\tau$ the angle between faces $OBA$ and $OBC$. Prove that
\[
    \gamma > \beta \cos \sigma + \alpha \cos \tau.
\]

\textbf{Solution.}
Assume $OA = OB = OC = 1$. The spherical areas of triangle sections $AOB, BOC, COA$ are $\frac{1}{2}\gamma, \frac{1}{2}\alpha, \frac{1}{2}\beta$.

Project arcs onto plane $OAB$. The signed areas of the projections of sectors $AOC$ and $BOC$ are $\frac{1}{2} \beta \cos \sigma$ and $\frac{1}{2} \alpha \cos \tau$. Their sum is less than the area of sector $AOB$, i.e., $\frac{1}{2} \gamma$. Multiply by 2 to get the result.

\subsection*{Problem 5}

Let $A$ be an $n \times n$ complex matrix with $n > 1$. Prove that:
\[
    A\bar{A} = I_n \iff \exists S \in \mathrm{GL}_n(\mathbb{C}) \text{ such that } A = S \bar{S}^{-1}.
\]
(\mathrm{GL}_n(\mathbb{C}) denotes the set of all complex invertible matrices \(n \times n\).)

\textbf{Solution.}
$(\Leftarrow)$ is trivial: $A = S \bar{S}^{-1} \Rightarrow AA = I$.

$(\Rightarrow)$: Pick $w \in \mathbb{C} \setminus \{0\}$, let $S = wA + w I$. Then:
\[
AS = A(wA + wI) = w I + w A = S.
\]
If $S$ is not invertible, then $1/w S = A - (w/w)I$ is singular $\Rightarrow w/w$ is eigenvalue of $A$. But $w/w$ can be any complex number on the unit circle. Hence there exists such $w$ for which $S$ is invertible.

\subsection*{Problem 6}

Let $f : \mathbb{R}^n \to \mathbb{R}$ be convex, with gradient $\nabla f$ defined everywhere and Lipschitz continuous:
\[
\exists L > 0 \quad \text{s.t.} \quad \forall x_1, x_2 \in \mathbb{R}^n,\quad \|\nabla f(x_1) - \nabla f(x_2)\| \le L \|x_1 - x_2\|.
\]
Prove:
\[
\|\nabla f(x_1) - \nabla f(x_2)\|^2 \le L \langle \nabla f(x_1) - \nabla f(x_2), x_1 - x_2 \rangle.
\]

\textbf{Solution.}
Define
\[
g(x) = f(x) - f(x_1) - \langle \nabla f(x_1), x - x_1 \rangle.
\]
Then $g$ is convex, differentiable, $g(x_1) = 0$, $\nabla g(x_1) = 0$. Define:
\[
y_0 = x_2 - \frac{1}{L} \nabla g(x_2), \quad y(t) = y_0 + t(x_2 - y_0).
\]
Using convexity and integration:
\[
g(x_2) = g(y_0) + \int_0^1 \langle \nabla g(y(t)), x_2 - y_0 \rangle dt \ge \frac{1}{2L} \|\nabla g(x_2)\|^2.
\]
Substituting back:
\[
f(x_2) - f(x_1) - \langle \nabla f(x_1), x_2 - x_1 \rangle \ge \frac{1}{2L} \|\nabla f(x_2) - \nabla f(x_1)\|^2.
\]
Exchanging $x_1 \leftrightarrow x_2$ and averaging yields the result.

\end{document}
