\documentclass{article}
\usepackage[utf8]{inputenc}
\usepackage[T1]{fontenc}
\usepackage{amsmath}
\usepackage{amsfonts}
\usepackage{amssymb}
\usepackage{enumitem}

\usepackage[margin=0.7in]{geometry} % Adjust the margins of document here.

\usepackage{tikz}                                          % Для простых рисунков в документе
\usetikzlibrary{matrix,arrows,decorations.pathmorphing,shapes.geometric,calc,snakes,backgrounds,arrows.meta}
\usepackage{xcolor}

\begin{document}
\pagestyle{plain}

\section*{9th IMC 2002, July 19 -- 25, Warsaw, Poland}

\section*{Second Day}

\subsection*{Problem 1}
Compute the determinant of the $n \times n$ matrix $A = [a_{ij}]$, where
\[
a_{ij} =
    \begin{cases}
        (-1)^{|i-j|}, & \text{if } i \ne j, \\
        2, & \text{if } i = j.
    \end{cases}
\]

\subsection*{Problem 2}

Two hundred students participated in a mathematical contest.
They had 6 problems to solve. It is known that each problem was correctly solved by at least 120 participants.
Prove that there must be two participants such that every problem was solved by
at least one of these two students.

\subsection*{Problem 3}

For each $n \ge 1$ define
\[
    a_n = \sum_{k=0}^\infty \frac{k^n}{k!}, \qquad
    b_n = \sum_{k=0}^\infty \frac{(-1)^k k^n}{k!}.
\]
Show that $a_n \cdot b_n$ is an integer.


\subsection*{Problem 4}
In the tetrahedron $OABC$, let $\angle BOC = \alpha$, $\angle COA = \beta$, $\angle AOB = \gamma$.
Let $\sigma$ be the angle between the faces $OAB$ and $OAC$,
and $\tau$ the angle between faces $OBA$ and $OBC$. Prove that
\[
    \gamma > \beta \cos \sigma + \alpha \cos \tau.
\]

\subsection*{Problem 5}

Let $A$ be an $n \times n$ complex matrix with $n > 1$. Prove that:
\[
    A\bar{A} = I_n \iff \exists S \in \mathrm{GL}_n(\mathbb{C}) \text{ such that } A = S \bar{S}^{-1}.
\]
(\(\mathrm{GL}_n(\mathbb{C})\) denotes the set of all \(n \times n\) complex invertible matrices.)

\subsection*{Problem 6}

Let $f : \mathbb{R}^n \to \mathbb{R}$ be convex, with gradient $\nabla f$ defined everywhere and Lipschitz continuous:
\[
\exists L > 0 \quad \text{s.t.} \quad \forall x_1, x_2 \in \mathbb{R}^n,\quad \|\nabla f(x_1) - \nabla f(x_2)\| \le L \|x_1 - x_2\|.
\]
Prove:
\[
\|\nabla f(x_1) - \nabla f(x_2)\|^2 \le L \langle \nabla f(x_1) - \nabla f(x_2), x_1 - x_2 \rangle.
\]

\end{document}
